\chapter{数学期望}
\begin{introduction}
  \item Intro to Prob\quad 2.4
  \item Prob$\&$Stat\quad2.2 2.3 2.7
\end{introduction}
\section{引导问题:期望的来源}
\begin{instance}[(分赌本问题)]
甲、乙两人赌技不相上下。两人进行赌博,各出赌注50法郎,每局中无平局。他们约定:谁先赢三局,则赢得全部赌本100法郎。当甲赢了二局,乙赢了一局,因故要中止赌博。现问这100法郎如何分才算公平?

\begin{enumerate}
\item 甲得$\frac{2}{3}$,乙得$\frac{1}{3}$。

这个方案看似是合理的。但是,如果甲赢了一局,乙赢了零局,那么这种分法的结果是:甲得 100法郎,乙得0法郎,这样是否合理?
\item 设想再赌下去,则甲最终所得$X$为一个随机变量,其可能取值为0或100,再赌两局必可结束。接下来两局的结果为$\{\text{甲甲, 甲乙, 乙甲, 乙乙}\}$。这四种情况中有三种情况:甲赢了100法郎。只有一种情况:乙赢下100法郎,因为两人赌技不相上下,所以,在两局后甲的收益是一个随机变量,其分布列为
 \begin{table}[h]
        \centering
\begin{tabular}{c|cc}
$X$ & 100           & 0             \\ \hline
$P$ & $\frac{3}{4}$ & $\frac{1}{4}$
\end{tabular}
\end{table}

于是,甲的“期望”收益应为
$$
100\times \frac{3}{4} + 0 \times \frac{1}{4} = 75\text{ (法郎)}
$$
相应地,乙的“期望”收益应为25(法郎)。
\end{enumerate}    
\end{instance}



\section{数学期望}


\begin{definition}{数学期望}\label{def:expectation}
\begin{enumerate}
\item 假设$X$为一个离散随机变量,其分布列记为
        $p_{i}=P\left(X=x_{i}\right) \quad i=1.2, \cdots, n, \cdots$。
    如果 $\sum_{i=1}^{\infty}\left|x_{i}\right| p_{i}<\infty$,则称 
    $$E(x)=\sum_{i=1}^{\infty} x_{i} p\left(x_{i}\right)$$
        为随机变量$X$的(数学)期望或均值。
\item 假设$X$为一个连续随机变量,其密度函数$p(x)$。如果$\int_{-\infty}^{+\infty}|x| p(x) d x<\infty$,则称 $$E(x)=\int_{-\infty}^{+\infty} x p(x) d x$$
为随机变量$X$的(数学)期望或均值。
\end{enumerate}
\end{definition}
\begin{remark}
    如果$\sum_{i=1}^{\infty}\left|x_{i}\right| p_{i}$或$\int_{-\infty}^{+\infty}|x| p(x)$不收敛,则称$X$的期望不存在。
\end{remark}
\begin{example}
设$X$服从区间$(a,b)$上的均匀分布,求$E(X)$.
\end{example}
\begin{solution}
因为$X$的密度函数为
$$
p(x) = \left\{
\begin{aligned}
  &  \frac{1}{b-a}, & a<x<b,\\
  &  0, &  \text{其他}.
\end{aligned}
\right.
$$
所以,
$$
E(X) = \int_{a}^b x \frac{1}{b-a}\text{d} x = \frac{1}{b-a} \left.\frac{x^2}{2}\right|_a^b = \frac{a+b}{2}.
$$
\end{solution}

\begin{example}\label{ex:chap06_discrete_rv_expectation}
如果$X$的分布列为
 \begin{table}[ht]
        \centering
\begin{tabular}{c|ccc}
$X$ & -1          & 0  & 1           \\ \hline
$P$ & $\frac{1}{3}$ & $\frac{1}{3}$ & $\frac{1}{3}$
\end{tabular}
\end{table}

那么$E(X) = 0$。而$X^2$的分布列为
\begin{table}[ht]
        \centering
\begin{tabular}{c|cc}
$X$ &            0  & 1           \\ \hline
$P$ &  $\frac{1}{3}$ & $\frac{2}{3}$
\end{tabular}
\end{table}

那么$E(X^2) = \frac{2}{3}$。
\end{example}

\section{数学期望的性质}
通过例\ref{ex:chap06_discrete_rv_expectation},我们在计算随机变量函数的期望时有统一的公式。

\begin{theorem}\label{thm:chap06_expectation_rv_funciton}
若随机变量$X$的分布用分布列$p(x_i)$或密度函数$p(x)$表示,则$X$的某一函数$g(X)$的数学期望为
$$E(g(x))=\left\{\begin{aligned}
&\sum_{i} g\left(x_{i}\right) p\left(x_{i}\right), &\quad \text{在离散场合}, \\
&\int_{-\infty}^{+\infty} g(x) p(x) d x, &\quad \text{在连续场合}. 
\end{aligned}\right.$$
\end{theorem}

\begin{property}
\begin{enumerate}
\item 若$c$是常数,则$E(c) = c$;
 \item 对任意常数$a$,有$E(aX) = aE(X)$;
 \item 对任意的两个函数$g_1(x)$和$g_2(x)$,有
 $$ 
 E(g_1(X)\pm g_2(X)) = E(g_1(X)) \pm E(g_2(X)).
 $$
\end{enumerate}
\end{property}
\begin{note}
    \vspace{5cm}
\end{note}
\begin{problem}
    已知$X$的期望$E(X)$,如何计算$X$的线性变换$aX+b$的期望呢?
\end{problem}



\section{方差与标准差}
通过定理\ref{thm:chap06_expectation_rv_funciton},我们可以计算随机变量$X$函数的期望。其中,有一个特殊的函数,即$(X-E(X))^2$,它的期望也是分布的重要特征,我们单独给一个定义——方差。

\begin{definition}{方差与标准差}\label{def:variance}
若随机变量$X^{2}$的数学期望$E(X^{2})$存在,则称偏差平方$$E(X-E(X))^{2}$$为随机变量$X$的方差,记为\begin{eqnarray*}
\text{Var}(X) &=&E(X-E(X))^{2}\\
&=& \left\{
\begin{aligned}
   & \sum_{i}\left(x_{i}-E(x)\right)^{2} p\left(x_{i}\right), \qquad \mbox{在离散场合}\\
   &\int_{-\infty}^{+\infty}(x-E(X))^{2} p(x) d x, \quad \mbox{在连续场合}
\end{aligned}
\right.
\end{eqnarray*}
称方差的平方根$\sqrt{\operatorname{Var}(X)}$为随机变量$X$的标准差,记为$\sigma(x)$或$\sigma_x$。
\end{definition}

\begin{property}
    \begin{enumerate}
        \item $\text{Var}(X) = E(X^2) - (E(X))^2$;
        \item 常数的方差为零,即若$c$为常数,则$\text{Var}(c) = 0$;
        \item 若$a,b$为常数,则$\text{Var}(aX+b) = a^2 \text{Var}(X)$.
    \end{enumerate}
\end{property}
\begin{note}
    \vspace{5cm}
\end{note}




这里介绍三个常见随机变量期望和方差的计算方法。
\begin{example}
    如果$X\sim b(n,p)$,那么$E(X) = np$且$\text{Var}(X) = np(1-p)$。
\end{example}
\begin{solution}
$X$的分布列为
    $$
    P(X=k) = C_n^k p^k (1-p)^{n-k},k = 0,1,2,\cdots,n.
    $$
于是,$X$的期望为
\begin{eqnarray*}
    E(X) &=& \sum_{k=0}^n k P(X=k)\\
    &=& \sum_{k=0}^n k \cdot C_n^k p^k (1-p)^{n-k}\\
    &=& \sum_{k=1}^n k\cdot \frac{n!}{k!(n-k)!} p^k (1-p)^{n-k}\\
 &=& \sum_{k=1}^n \frac{n!}{(k-1)!(n-k)!} p^k (1-p)^{n-k}\\
 &=& np\cdot \sum_{k=1}^n \frac{(n-1)!}{(k-1)!(n-k)!} p^{k-1} (1-p)^{n-k}\\
 &\overset{k'=k-1}{=}& np\cdot \sum_{k'=0}^{n-1} \frac{(n-1)!}{(k')!(n-1-k')!} p^{k'} (1-p)^{n-1-k'}\\
 &=& np,
\end{eqnarray*}
因为$X$的方差$\text{Var}(X) = E(X^2) - (E(X))^2$,所以,需要计算$X^2$的期望。而$X^2$的期望为
\begin{eqnarray*}
    E(X) &=& \sum_{k=0}^n k^2 P(X=k)\\
    &=& \sum_{k=0}^n k^2\cdot  C_n^k p^k (1-p)^{n-k}\\
    &=& \sum_{k=1}^n k^2\cdot \frac{n!}{k!(n-k)!} p^k (1-p)^{n-k}\\
 &=& \sum_{k=1}^n k\cdot \frac{n!}{(k-1)!(n-k)!} p^k (1-p)^{n-k}\\
 &=& np\cdot \sum_{k=1}^n k\cdot  \frac{(n-1)!}{(k-1)!(n-k)!} p^{k-1} (1-p)^{n-k}\\
 &\overset{k'=k-1}{=}& np\cdot \sum_{k'=0}^{n-1} (k'+1) \frac{(n-1)!}{(k')!(n-1-k')!} p^{k'} (1-p)^{n-1-k'}\\
 &=& np\cdot ((n-1)p+1) \\
 &=& n(n-1)p^2 + np,
\end{eqnarray*}
于是,
$$\text{Var}(X) = E(X^2) - (E (X))^2 = n(n-1)p^2 + np - (np)^2 = np - np^2 = np(1-p).$$
\end{solution}

\begin{example}
    如果$X\sim N(\mu,\sigma^2)$,那么$E(X) = \mu$且$\text{Var}(X) = \sigma^2$。
\end{example}
\begin{solution}
$X$的密度函数为
    $$
    p(x) = \frac{1}{\sqrt{2\pi \sigma^2}} \exp\left\{ - \frac{1}{2\sigma^2} (x-\mu)^2\right\},x\in R.
    $$
于是,$X$的期望为
\begin{eqnarray*}
    E(X) &=&  \int_{0}^{\infty} x p(x) \text{d}x\\
 &=&     \int_{-\infty}^{\infty} x \frac{1}{\sqrt{2\pi \sigma^2}} \exp\left\{ - \frac{1}{2\sigma^2} (x-\mu)^2\right\} \text{d}x\\
    &\overset{z = x-\mu}{=}&  \int_{-\infty}^{\infty} (z+\mu) \frac{1}{\sqrt{2\pi \sigma^2}} \exp\left\{ - \frac{1}{2\sigma^2} (z)^2\right\} \text{d} z\\
    &= & \int_{-\infty}^{\infty} z \frac{1}{\sqrt{2\pi \sigma^2}} \exp\left\{ - \frac{1}{2\sigma^2} (z)^2\right\} \text{d} z + \mu\int_{-\infty}^{\infty}  \frac{1}{\sqrt{2\pi \sigma^2}} \exp\left\{ - \frac{1}{2\sigma^2} (z)^2\right\} \text{d} z\\
    &=& \mu.
\end{eqnarray*}
而$X$的方差为
\begin{eqnarray*}
    \text{Var}(X) &=& E(X-E(X))^2\\
    &=&\int_{-\infty}^{\infty} (x-\mu)^2 \frac{1}{\sqrt{2\pi \sigma^2}} \exp\left\{ - \frac{1}{2\sigma^2} (x-\mu)^2\right\} \text{d} x \\
    &\overset{z = x-\mu}{=}& \int_{-\infty}^{\infty} z^2 \frac{1}{\sqrt{2\pi \sigma^2}} \exp\left\{ - \frac{1}{2\sigma^2} z^2\right\} \text{d} z\\
&=&  \frac{1}{\sqrt{2\pi \sigma^2}} \int_{-\infty}^{\infty} -{\sigma^2} z \text{d} \exp\left\{ - \frac{1}{2\sigma^2} z^2\right\} \\
&=& \frac{1}{\sqrt{2\pi \sigma^2}} \cdot \left(\left.-{\sigma^2} z\exp\left\{ - \frac{1}{2\sigma^2} z^2\right\} \right|_{-infty}^{\infty}+  \int_{-\infty}^{\infty}  {\sigma^2}\exp\left\{ - \frac{1}{2\sigma^2} z^2\right\} \text{d}z \right)\\
&=&  \int_{-\infty}^{\infty} {\sigma^2} \frac{1}{\sqrt{2\pi \sigma^2}} \exp\left\{ - \frac{1}{2\sigma^2} z^2\right\} \text{d}z \\
&=& \sigma^2.
\end{eqnarray*}
\end{solution}

\begin{example}
    如果$X\sim Ga(\alpha,\lambda)$,那么$E(X) = \frac{\alpha}{\lambda}$且$\text{Var}(X) = \frac{\alpha}{\lambda^2}$。
\end{example}
\begin{solution}
$X$的密度函数为
    $$
    p(x) = \frac{\lambda^\alpha}{\Gamma(\alpha)} x^{\alpha-1} \exp\{-\lambda x\},x>0.
    $$
这里我们先考虑$X^k$的期望
\begin{eqnarray*}
    E(X^k) &=&  \int_{0}^{\infty} x^k p(x) \text{d}x = \int_{0}^{\infty} x^k \frac{\lambda^\alpha}{\Gamma(\alpha)} x^{\alpha-1} \exp\{-\lambda x\}\text{d}x\\
    &=& \int_{0}^{\infty}  \frac{\lambda^\alpha}{\Gamma(\alpha)} x^{\alpha+k -1} \exp\{-\lambda x\}\text{d}x\\
    &=& \int_{0}^{\infty} \frac{(\alpha+k-1)\times \cdots \times (\alpha)}{\lambda^k}\cdot \frac{\lambda^{\alpha+k}}{\Gamma(\alpha+k)} x^{\alpha+k -1} \exp\{-\lambda x\}\text{d}x\\
    &=&\frac{(\alpha+k-1)\times \cdots \times (\alpha)}{\lambda^k}.
\end{eqnarray*}
于是,$X$的期望为
$$E(X) = \frac{\alpha}{\lambda}.$$
而
$X$的方差为
$$
\text{Var}(X) = E(X^2) - (E X)^2 = \frac{(\alpha+1)\alpha}{\lambda^2} -  \frac{\alpha^2}{\lambda^2}  =  \frac{\alpha}{\lambda^2}.
$$
\end{solution}

\begin{remark}
    在计算期望时,一个最常用的方法是“合成概率函数”。
\end{remark}

\section{$k$阶矩}

\begin{definition}\label{def:moment}
设$X$为随机变量,$k$为正整数。如果以下的数学期望都存在,则
\begin{enumerate}
    \item 称
$$
\mu_k = E(X^k)
$$
为$X$的$k$阶原点矩。
\item 称
$$
\nu_k = E((X-E(X))^k)
$$
为$X$的$k$阶中心矩。
\end{enumerate}
\end{definition}

\begin{remark}
    \begin{enumerate}
        \item $\mu_1$是数学期望;
        \item $\nu_2$是方差;
        \item $k$阶矩存在时,$k-1$阶矩也存在,从而低于$k$的各阶矩都存在。
        \item 中心矩和原点矩的关系
        $$
        \nu_k = E(X-E(X))^k = E(X-\mu_1)^k = \sum_{i=0}^k C_k^i \mu_i (-\mu_1)^{k-i}.
        $$
    \end{enumerate}
\end{remark} 

基于矩,统计学中也关注以下三个特征数,供学生课后自行了解。

\subsection{变异系数}
\begin{definition}\label{def:cv}
设随机变量$X$的二阶矩存在,则称比值
$$
C_v(X) = \frac{\sqrt{Var(X)}}{E(X)} = \frac{\sigma(X)}{E(X)}.
$$
为$X$的变异系数。
\end{definition}
\begin{remark}
    变异系数是一个无量纲的量,从而消除量纲对波动的影响。
\end{remark}
\subsection{偏度系数}
\begin{definition}\label{def:skewness}
设随机变量$X$的前三阶矩阵存在,则比值
$$\beta_{s}=\frac{\nu_{3}}{\nu_{2}^{{3}/{2}}}=\frac{E(X-E(X))^{3}}{(\operatorname{Var}(X))^{{3}/{2}}}$$
称为$X$或分布的偏度系数,简称偏度。
    
\end{definition}
\begin{remark}
    \begin{enumerate}
    \item 偏度$\beta_s$是描述分布偏离对称程度的一个特征数;
        \item $\beta_{s}<0$时,称左偏,又称负偏;
        \item $\beta_{s}>0$时,称右偏,又称正偏;
        \item 对于连续型随机变量,当密度函数$p(x)$关于其数学期望对称,则其三阶中心矩$\nu_3$必为0,从而$\beta_s =0$。
    \end{enumerate}
\end{remark}
\begin{itemize}
    \item 当密度$p(x)$关于数学期望堆成时,$V_{3}=0 \Rightarrow \beta_{s}=0$。
    \item $\beta_{s}<0$时,称左偏。
    \item $\beta_{s}>0$时,称右偏。
\end{itemize}


\subsection{峰度系数}
\begin{definition}\label{def:kurtosis}
设随机变量$X$的前四阶存在,则
$$\beta_{k}=\frac{\nu_{4}}{\nu_{2}^{2}}-3=\frac{E(X-E(X))^{4}}{(\operatorname{Var}(X))^{2}}-3 $$
称为$X$或分布的峰度系数,简称峰度。
\end{definition}
\begin{remark}
    \begin{enumerate}
        \item 峰度是描述分布是否存在“尖峰后尾”现象的一个特征数;
        \item 正态分布$N(\mu,\sigma^2)$中$\nu_2 = \sigma^2$且$\nu_4 = 3\sigma^4$(请同学们课后证明)。于是,$\beta_k = 3\sigma^4 / \sigma^4 - 3 =0$。这表明了峰度系数$\beta_k$是相对于正态分布而言的超出量。
        \item $\beta_{k}>0$表示“厚尾”分布;
         \item $\beta_{k}<0$表示“薄尾”分布;
         \item 偏度和峰度都是描述分布形状的特征数。
    \end{enumerate}
\end{remark}


\section{基于矩的概率不等式}
\begin{theorem}{Markov不等式}\label{thm:chap06_Markov_Inequality}
    设$X$是一个非负随机变量。如果$a>0$,那么
	$$P(X>a ) \leq \frac{E(X)}{a}.$$
\end{theorem}
\begin{proof}
     (仅考虑连续情况)设$X$是一个连续随机变量,其密度函数为$p(x)$。于是,$X$的期望是
     \begin{eqnarray*}
        E(X) &=& \int_{0}^{\infty} xp(x)\text{d} x \\
        &=& \int_{0}^{a} xp(x)\text{d} x +\int_{a}^{\infty} xp(x)\text{d} x \\
        &\geq&  \int_{a}^{\infty} xp(x)\text{d} x 
        \\
        &\geq&  \int_{a}^{\infty} ap(x)\text{d} x \\
        &=& a P(X\geq a)
     \end{eqnarray*}
因此,定理得证,对离散随机变量亦可类似进行证明。
\end{proof}

\begin{theorem}{Chebyshev不等式}\label{thm:chap06_Chebyshev_Inequality}
    设随机变量$X$的数学期望和方差都存在,则对任意常数$\varepsilon >0$,有
	$$P(|x-E x| \geqslant \varepsilon) \leq \frac{\text{Var}(X)}{\varepsilon^{2}}.$$
\end{theorem}
\begin{remark}
事件${|x-E x| \geqslant \varepsilon}$称为大偏差,其概率$P(|x-E x| \geqslant \varepsilon)$称为大偏差发生概率。
\end{remark}
\begin{proof}
    (仅考虑连续情况)其密度函数$p(x)$,记$E(x)=\mu$,
	\begin{eqnarray*}
	    P(|X-\mu| \geqslant \varepsilon) &=&\int_{\{x:|x-\mu| \geqslant \varepsilon\}} p(x) d x \\
& \leqslant &\int_{\{x:|x-\mu| \geqslant \varepsilon\}} \frac{(x-\mu)^{2}}{\varepsilon^{2}} p(x) d x \\
& \leqslant& \frac{1}{\varepsilon^{2}} \int_{-\infty}^{+\infty}(x-\mu)^{2} p(x) d x \\
&=&\frac{\text{Var}(X)}{\varepsilon^{2}}.
	\end{eqnarray*}
\end{proof}

\section{信息量}
在信息论中,由美国克劳德·艾尔伍德·香农(Claude Elwood Shannon)院士提出了一个概念——信息熵,也称为香农熵。对于一个离散的随机变量,其概率分布列
$$
P(X=x_i) = p_i,i=1,2,\cdots,n.
$$
则信息熵的定义为
$$
H(X) = -\sum_{i=1}^n p_i \log_2(p_i).
$$
从形式上来看,信息熵的定义是随机变量$X$函数的期望。香农熵可以用于度量随机变量的不确定性。而如今,在机器学习中,对于信息熵有更一般的定义。
\begin{definition}[(信息熵)]
    对于一个随机变量$X$,其概率分布列或概率密度函数为$p(x)$,则称
    $$
    H(X) = - E(\ln(p(X)))
    $$
    为信息熵。
\end{definition}

\begin{example}
    对于一个二点分布随机变量$X$,$P(X=1)=p$。则信息熵为
    \begin{eqnarray*}
        H(X) = - ((1-p) \ln(1-p) + p \ln(p)).
    \end{eqnarray*}
\end{example}

除了香农熵之外,费歇尔信息量是统计学中一个基本概念。
\begin{definition}
    对于一个随机变量$X$,其概率分布列或概率密度函数为$p(x;\theta), \theta \in \Theta$满足下列条件:
    \begin{itemize}
        \item 参数空间$\Theta$是直线上的一个开区间;
        \item 支撑$S = \{x:p(x;\theta)>0\}$与$\theta$无关;
        \item 导数$\frac{\partial }{\partial \theta}p(x;\theta)$对一切$\theta \in \Theta$都存在;
        \item 对$p(x;\theta)$,积分与微分运算可交换次序,即
        $$
        \frac{\partial }{\partial \theta} \int_{-\infty}^{\infty} p(x;\theta)\text{d} x =  \int_{-\infty}^{\infty} \frac{\partial }{\partial \theta}p(x;\theta)\text{d} x;
        $$
        \item 期望$E\left(\frac{\partial }{\partial \theta} \ln p(x;\theta)  \right)^2$存在;
    \end{itemize}
    则称
    $$
    I(\theta) = E\left(\frac{\partial }{\partial \theta} \ln p(X;\theta)  \right)^2
    $$
    为随机变量$X$的费歇尔信息量。
\end{definition}

\begin{example}
    随机变量$X$服从泊松分布$P(\lambda)$,即其分布列为
    $$
    p(x;\lambda) = \frac{\lambda^x}{x!}e^{-\lambda}, x=0,1,2,\cdots
    $$
    可验证泊松分布的分布列满足条件。于是,我们可计算
    \begin{enumerate}
        \item 求对数,即
        $$
        \ln p(x;\lambda) = x\ln \lambda - \lambda - \ln(x!).
        $$
        \item 求导数,即
        $$
        \frac{\partial }{\partial \lambda}  \ln p(x;\lambda) = \frac{x}{\lambda} - 1
        $$
        \item 求期望,即
$$
I(\lambda ) = E\left( \frac{X}{\lambda} - 1 \right)^2 = E\left( \frac{(X-\lambda)^2}{\lambda^2}  \right)^2 = \frac{1}{\lambda}.
$$
    \end{enumerate}
\end{example}

\begin{example}
    随机变量$X$服从指数分布$Exp(1/\theta)$,即其密度函数为
    $$
    p(x;\theta) = \frac{1}{\theta} \exp\left\{-\frac{x}{\theta}\right\}, x>0
    $$
    可验证指数分布的分布列满足条件。于是,我们可计算
    \begin{enumerate}
        \item 求对数,即
        $$
        \ln p(x;\theta) = -\ln(\theta) - \frac{x}{\theta}.
        $$
        \item 求导数,即
        $$
        \frac{\partial }{\partial \theta}  \ln p(x;\theta) = -\frac{1}{\theta} + \frac{x}{\theta^2}
        $$
        \item 求期望,即
$$
I(\lambda ) = E\left( -\frac{1}{\theta} + \frac{X}{\theta^2} \right)^2 = \frac{1}{\theta^2}.
$$
    \end{enumerate}
\end{example}


\section{习题}

    \begin{enumerate}
        \item 设随机变量$X$的密度函数为
$$ p(x) = \begin{cases}
a + bx^2, & 0\leq x\leq 1\\
0, & \text{其他}.
\end{cases}
$$
如果$E(X) = 2/3$,求$a$和$b$。

\item 设随机变量$X$的分布函数为
$$
	    F(x) = 1- e^{-x^2},x > 0
$$
试求$E(X)$和$\text{Var}(X)$。

\item 设随机变量$X \sim U(a,b)$,对$k = 1,2,3,4$,求:
\begin{enumerate}
    \item $\mu_k = E(X^k)$与$\upsilon_k = E(X - E(X))^k$;
    \item 偏度系数和峰度系数。
\end{enumerate}

\item 设随机变量$X$服从双参数韦布尔分布,其分布函数为
$$
	    F(x) = 1-\exp \left \{-\left (\frac{x}{\eta} \right )^m \right \}, x>0,
$$
其中$\eta > 0,m > 0$.试写出该分布的$p$分位数$x_p$的表达式,且求出当$m = 1.5,\eta = 1000$时的$x_{0.1},x_{0.5},x_{0.8}$的值。
 
\begin{exercise} 请根据常见分布的分布列或密度函数,计算其期望和方差。
	\begin{table}[ht]
			\centering
   \caption{常见分布的分布列(密度函数)及其期望与方差}
			\begin{tabular}{llc c}
				\hline
				分布 & 分布列/密度函数& 期望& 方差\\
				\hline
				二项分布$b(n,p)$ & $P(X=x) = \frac{n!}{x!(n-x)!} p^x (1-p)^{n-x}, x = 0,1, \cdots,n$ & $np$&$np(1-p)$\\
			\hline
				泊松分布$P(\lambda)$ &$P(X=x) = \frac{\lambda^x}{x!} e^{-\lambda }, x = 0,1, \cdots$  & $\lambda$& $\lambda$\\
					\hline
				负二项分布$Nb(r,p)$ &$P(X=x) = \frac{(x-1)!}{(r-1)!(x-r)!} (1-p)^{x-r}p^r, x = r,r+1, \cdots$ & $\frac{r}{p}$& $\frac{r(1-p)}{p^2}$\\
				\hline
				正态分布$N(\mu,\sigma^2)$ &$p(x) = \frac{1}{\sqrt{2\pi \sigma^2}} \exp\left\{-\frac{(x-\mu)^2}{2\sigma^2}\right\}, -\infty <x <\infty$ &$\mu$&$\sigma^2$\\
				\hline
				均匀分布$U(a,b)$ &$p(x) = \frac{1}{b-a}, a<x<b$ & $\frac{a+b}{2}$ & $\frac{(b-a)^2}{12}$\\
				\hline
				伽马分布$Ga(\alpha,\lambda)$ &$p(x) = \frac{\lambda^\alpha}{\Gamma(\alpha)} x^{\alpha -1} \exp\{-\lambda x\}, x \geq  0$ $\frac{\alpha}{\lambda}$&$\frac{\alpha}{\lambda^2}$\\
				\hline
				 贝塔分布$Be(a,b)$ &$p(x) = \frac{\Gamma(a+b)}{\Gamma(a)\Gamma(b)} x^{a-1} (1-x)^{b-1}, 0<x<1$ &$\frac{a}{a+b}$ & $\frac{ab}{(a+b)^2(a+b+1)}$ \\
				\hline
			\end{tabular}
		\end{table}
\end{exercise}
   \end{enumerate}



    