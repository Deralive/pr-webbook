\chapter{频率学派的常见点估计方法}
\begin{introduction}
  \item Prob $\&$ Stat\quad 6.2 6.3
\end{introduction}
从本讲开始,我们介绍统计学中最关键的两个任务——估计和检验。有以下的四个观点:
\begin{itemize}
    \item 点估计是灵魂;
    \item 区间估计是工具;
    \item 假设检验是核心;
    \item 统计量是基础。
\end{itemize}

\section{引导问题}
在参数模型中,假定总体分布$X\sim p(x;\bm{\theta}),\bm{\theta}\in \bm{\Theta}$。这里$\bm{\theta}$是总体分布中的未知参数,可以是一维的,也可以是多维的。
\begin{example}
考虑以下两个总体分布:
\begin{enumerate}
\item 若$X \sim P(\lambda)$。这里总体分布存在一个泊松分布族,即
    $$
    \mathcal{P} = \left\{\frac{\lambda^x}{x!}e^{-\lambda}: \lambda > 0\right\},
    $$
    其中参数是$\bm{\theta} = \lambda, \bm{\Theta} = R^{+}$。一旦$\lambda = \lambda_0$确定后,总体分布就唯一确定。
    \item 若$X \sim N(\mu,\sigma^2)$。这里总体分布存在一个正态分布族,即
    $$
    \mathcal{P} = \left\{\frac{1}{\sqrt{2\pi \sigma^2}}\exp\left\{-\frac{1}{2\sigma^2}(x-\mu)^2\right\}: \mu \in R, \sigma^2 > 0\right\},
    $$
    其中参数是$\bm{\theta} = (\mu,\sigma^2)', \bm{\Theta} = R\times R^{+}$。一旦$(\mu,\sigma^2) = (\mu_0,\sigma_0^2)$确定后,总体分布就唯一确定。
\end{enumerate}
\end{example}

实际中,参数模型中的参数都是未知的,而我们需要通过样本$x_1,x_2,\cdots,x_n$来推断$\bm{\theta}$(或其函数$g(\bm{\theta})$)。

\begin{definition}{估计}
设$x_1,x_2,\cdots,x_n$是来自总体的一个样本,称用于估计未知参数$\bm{\theta}$的统计量
$$
\hat{\bm{\theta}} = \hat{\bm{\theta}}(x_1,x_2,\cdots,x_n)
$$
为$\bm{\theta}$的估计量,或称为$\bm{\theta}$的点估计,简称估计。
\end{definition}


在本讲中,我们讲介绍三种频率学派的点估计思想:替换、拟合、似然及其对应的估计方法。

\section{替换思想}
\subsection{矩法}
替换的本质思想是:利用经验分布函数$F_n(x)$来替换总体分布函数$F(x)$。这里因为我们假定了参数模型,由参数唯一能够决定分布函数,于是记为$F_{\theta}(x)$。比较一下这两个“分布”函数是不同的。
\begin{itemize}
    \item $F_n(x)$:基于样本可计算而得,是完全已知的;
\item $F_{\theta}(x)$:一定存在未知的信息,完全未知或部分未知;
\end{itemize}

替换原理,是有卡尔·皮尔逊教授于1900年提出的,也被称为\textbf{矩法}。简单可以概括为
\begin{itemize}
    \item 用样本矩替换总体矩(既可以是原点矩,也可以是中心矩);
    \item 用样本矩的函数去替换相应的总体矩的函数;
\end{itemize}
这个替换原理也可以应用于分布未知的场合,对参数作出估计,即
\begin{itemize}
    \item 用样本均值$\bar{x}$估计总体均值$E(X)$;
    \item 用样本方差$s^2$估计总体方差$\text{Var}(X)$;
    \item 用事件$A$出现的频率估计事件$A$发生的概率;
    \item 用样本的$p$分位数估计总体的$p$分位数。
\end{itemize}
\subsection{矩估计}
基于替换原理,以下我们给出得到矩估计的具体步骤:
 \begin{enumerate}
        \item 设总体具有已知的概率函数$p(x;\theta_{1},\theta_{2},\cdots,\theta_{k}),(\theta_{1},\cdots,\theta_{k})\in \Theta$是未知参数(向量);
        \item 得到样本$x_{1},x_{2},\cdots,x_{n}$;
        \item 假设总体分布的$k$阶原点矩$\mu_{k}$存在(对所有的$0<j\leq k$,各低阶矩$\mu_{j}$均存在);
        \item 若假设$\theta_{1},\theta_{2},\cdots,\theta_{k}$能表示成总体矩$\mu_{1},\mu_{2},\cdots,\mu_{k}$的函数,即$$\theta_{j}=\theta_{j}(\mu_{1},\mu_{2},\cdots,\mu_{k})$$则可给出诸$\theta_{j}$的矩估计$$\hat{\theta}_{j}=\theta_{j}(a_{1},a_{2},\cdots,a_{k}),j=1,2,\cdots,k$$其中,$a_{j}=\frac{1}{n}\sum_{i=1}^{n}x_{i}^{j}$,是$j$阶样本原点矩,$j=1,2,\cdots,k.$
        \item 另外,我们要估计参数的函数$\eta  =\eta (\theta _{1} ,\theta _{2} ,\cdots ,\theta _{k})$,则$\eta$的矩估计为
        $$\hat{\eta  } =\eta (\hat{\theta}_{1} ,\hat{\theta }_{2} ,\cdots ,\hat{\theta}_{k}).$$
    \end{enumerate}

\begin{example}
设总体为指数分布,即
$$
p(x;\lambda) = \lambda e^{-\lambda}, x\geq 0.
$$
$x_1,x_2,\cdots,x_n$是样本。

一方面,注意到$E(X) = 1/\lambda$,即$\lambda= 1/E(X)$。故$\lambda$的矩估计为
    $$
    \hat{\lambda}_1 = \frac{1}{\bar{x}}.
    $$

另一方面,由于$\text{Var}(X) = 1/\lambda^2$,即$\lambda = 1/\sqrt{\text{Var}(X)}$。故$\lambda$的矩估计为
    $$
    \hat{\lambda}_2 = \frac{1}{s}.
    $$
\end{example}
\begin{remark}
    这表明了矩估计不唯一,但通常应该尽量采用\textbf{低阶矩}给出未知参数的估计。
\end{remark}

\begin{example}
    设一个实验有三种可能结果,其发生的概率分别为$$p_{1}=\theta^{2},\quad p_{2}=2\theta(1-\theta) ,\quad p_{3}=(1-\theta)^{2}$$
    现在估了$n$次试验,观测到三种结果发生的次数分别为$n_{1},n_{2},n_{3}$。如何估计$\theta$?
\end{example}
\begin{solution}
    设该实验的三种结果分别是$a_1,a_2,a_3$。于是,分布列如表\ref{tab:lecture18_1}。
    \begin{table}[ht]
        \centering
        \caption{三种结果的实验}\label{tab:lecture18_1}
    \begin{tabular}{cccc}
    \toprule
    结果 & $a_{1}$ & $a_{2}$ & $a_{3}$\\
    \midrule
    概率 &  $\theta^{2}$&$2\theta(1-\theta)$  &$(1-\theta)^{2}$ \\
    \midrule
    频率 &  $\frac{n_{1}}{n}$& $\frac{n_{2}}{n}$ & $\frac{n_{3}}{n}$\\
    \bottomrule
    \end{tabular}
    \end{table}
为估计$\theta$,我们可以构建三个矩估计,即
\begin{enumerate}
    \item 可以用$n_1/n$来估计$\theta^2$,则
    $$\hat{\theta}_1 = \sqrt{\frac{n_1}{n}}.$$
    \item 可以用$n_3/n$来估计$(1-\theta)^2$,则
    $$\hat{\theta}_2 = 1 - \sqrt{\frac{n_3}{n}}.$$
   \item 可以用$n_1/n + n_2/(2n)$来估计$\theta$,则
    $$\hat{\theta}_3 = \frac{2n_1 + n_2}{2n}.$$
\end{enumerate}
\end{solution}
\begin{problem}
    都是针对$\theta$的估计,三种估计$\hat{\theta}_1,\hat{\theta}_2,\hat{\theta}_3$有无好坏差异?
\end{problem}
\section{拟合思想}
拟合思想是目前大部分机器学习模型,乃至深度学习模型,参数(或权重)的学习本质上就是给定一个复杂的模型框架,使其尽可能贴合数据,这就是拟合。在数理统计中,最小二乘估计是利用拟合思想来构造的估计方法。在后续的课程中,我们将详细地介绍最小二乘估计。这里我们介绍一个简单的例子。
\begin{example}
总体分布$X\sim N(\mu,1)$,而$x_1,x_2,\cdots,x_n$是样本。我们想要估计$\mu$。因为$\mu$是总体分布的期望,每一个样本$x_1,x_2,\cdots,x_n$都在$\mu$附近波动。我们想要找到一个实数$c$与这些样本最\textbf{接近}。于是,我们需要定义一个损失函数
$$
l(c) = \sum_{i=1}^n g(|x_i - c|) = \sum_{i=1}^n |x_i - c|^{k} =\left\{
\begin{aligned}
&\sum_{i=1}^n |x_i - c|^2 , &\text{如果$k=2$};\\
&\sum_{i=1}^n |x_i - c| , &\text{如果$k=1$}.
\end{aligned}
\right.
$$
拟合思想促使我们所得到的估计为
$$
\hat{\mu} = \arg\min_{c} l(c).
$$
当$k=2$时,则损失函数定义为$l(c) = \sum_{i=1}^n (x_i-c)^2$。我们需要求$l(c)$的最小值点,则对$l(c)$求导,即
$$
\frac{\partial l(c)}{\partial c} = -2 \sum_{i=1}^n (x_i-c).
$$
令$\frac{\partial l(c)}{\partial c} = 0$。所以,
$$
c = \frac{1}{n} \sum_{i=1}^n x_i = \bar{x}.
$$
因此,我们通常称$\bar{x}$为$\mu$的最小二乘估计。
\end{example}
\begin{problem}
 当$k=1$时,则损失函数定义为$l(c) = \sum_{i=1}^n |x_i-c|$。通过最小化$l(c)$可以得到最小一乘估计。这个最小一乘估计是什么?
\end{problem}
\begin{note}
    \vspace{5cm}
\end{note}

\section{似然思想}
\begin{problem}
 考虑有两个盒子各有100个球,记为A盒和B盒。A盒中有99个白球,1个黑球;B盒中有1个白球,99个黑球。我们从某个盒子中抽出了一个球,发现这个球是白球。请问:我们是从哪个盒子中抽出球的?   
\end{problem}
\begin{remark}
    似然思想的本质是“以成败论英雄”。
\end{remark}

最大似然估计,顾名思义就是使得似然函数的最大值点。
\begin{problem}
    什么是似然函数?
\end{problem}

\begin{definition}{似然函数}
    设总体分布的概率分布列或概率密度函数为$p(x;\theta)$,其中$\theta$是未知参数。而$x_1,x_2,\cdots,x_n$是样本。$x_1,x_2,\cdots,x_n$的联合分布列或联合密度函数为
    $$
    p(x_1,x_2,\cdots,x_n;\theta) = \prod_{i=1}^n p(x_i;\theta).
    $$
    对于未知参数$\theta$而言,称$p(x_1,x_2,\cdots,x_n;\theta)$为似然函数,记为$L(\theta)$.
\end{definition}
既然我们明确了,似然函数本质上就是样本的联合概率(质量/密度)函数。将其看成参数的函数,这个函数就是似然函数。于是,我们给出最大似然估计明确的定义。
\begin{definition}{最大似然估计}
    设$\theta$是待估参数,$L(\theta)$是似然函数。如果统计量$\hat{\theta} = \hat{\theta}(x_1,x_2,\cdots,x_n)$满足
    $$
    L(\hat{\theta}) = \sup_{\theta\in \overline{\Theta}} L(\theta).
    $$
    那么称$\hat{\theta}$是$\theta$的最大似然估计,记为MLE(maximum likelihood estimate)。
\end{definition}
\begin{remark}
这里$\overline{\Theta}$包括 $\theta$的定义域$\Theta$及其边界。
\end{remark}
\begin{problem}
    最大似然估计怎么求?
\end{problem}
求最大似然估计的一般步骤为
\begin{enumerate}
    \item 求对数似然函数,即$l(\theta) = \ln L(\theta)$.
    \item 求对数似然函数的驻点$\hat{\theta}$,即$\hat{\theta}$满足
    $$
    \frac{\partial l(\theta)}{\partial \theta} = 0.
    $$
    \item 验证$\hat{\theta}$是$L(\theta)$或$l(\theta)$的最大值点。
\end{enumerate}
\begin{remark}
    从上述步骤来看,就是找似然函数或者对数似然函数的最大值点,这才是本质问题。
\end{remark}

\begin{example}
    设总体分布$X\sim b(1,\theta)$,而$x_1,x_2,\cdots,x_n$是样本。于是,$x_1,x_2,\cdots,x_n$的联合分布列为
    $$
    p(x_1,x_2,\cdots,x_n) = (\theta)^{\sum_{i=1}^n x_i} (1-\theta)^{n-\sum_{i=1}^n x_i}
    $$
    这也是似然函数,记为$L(\theta)$。然后,对数似然函数为
    $$
    l(\theta) = \ln L(\theta) = \sum_{i=1}^n x_i \ln (\theta) + (n-\sum_{i=1}^n x_i) \ln (1-\theta).
    $$
    对$l(\theta)$关于$\theta$求导,即
    $$
    0=\frac{\partial l(\theta)}{\partial \theta} = \frac{\sum_{i=1}^n x_i}{\theta} - \frac{n-\sum_{i=1}^n x_i}{1-\theta}.
    $$
    由此可得
    $$
    \hat{\theta} = \frac{1}{n} \sum_{i=1}^n x_i = \bar{x}.
    $$
    我们还可以验证
    $$
   \left. \frac{\partial^2 l(\theta)}{\partial \theta^2}\right|_{\theta = \hat{\theta}} = \left.-\frac{\sum_{i=1}^n x_i}{\theta^2} - \frac{n-\sum_{i=1}^n x_i}{(1-\theta)^2} \right|_{\theta = \hat{\theta}} < 0
    $$
    所以,$\hat{\theta}$是$l(\theta)$最大值点。因此,$\theta$的最大似然估计为$\bar{x}$.
\end{example}
    

\begin{example}
 设该实验的三种结果分别是$a_1,a_2,a_3$。于是,分布列如表\ref{tab:lecture18_2}。
这里我们介绍$\theta$的最大似然估计。

 
    \begin{table}[ht]
        \centering
        \caption{三种结果的实验}\label{tab:lecture18_2}
    \begin{tabular}{cccc}
    \toprule
    结果 & $a_{1}$ & $a_{2}$ & $a_{3}$\\
    \midrule
    概率 &  $\theta^{2}$&$2\theta(1-\theta)$  &$(1-\theta)^{2}$ \\
    \midrule
    频率 &  $\frac{n_{1}}{n}$& $\frac{n_{2}}{n}$ & $\frac{n_{3}}{n}$\\
    \bottomrule
    \end{tabular}
    \end{table}

首先,似然函数为
$$
L(\theta) = \frac{n!}{n_1!n_2!n_3!} (\theta^2)^{n_1} (2\theta (1-\theta))^{n_2}((1-\theta)^2)^{n_3} = \frac{n!}{n_1!n_2!n_3!} 2^{n_2}\theta^{2n_1+n_2}(1-\theta)^{n_2+2n_3}.
$$
而对数似然函数为
$$
l(\theta) = \ln L(\theta) = \ln \left(\frac{n!}{n_1!n_2!n_3!} 2^{n_2} \right) + (2n_1+n_2) \ln (\theta) + (n_2+2n_3) \ln (1-\theta).
$$
然后对$l(\theta)$关于$\theta$求导,即
$$
\frac{\partial l(\theta)}{\partial \theta} = \frac{2n_1+n_2}{\theta}  - \frac{n_2+2n_3}{1-\theta}.
$$
可以解得
$$
\hat{\theta} = \frac{2n_1 + n_2}{2n_1 + n_2 + n_2 + 2n_3} = \frac{2n_1 + n_2}{2n}.
$$
\end{example}
\begin{remark}
“求导”是求最大似然估计最常用的方法,但并不是在所有场合求导都有效。
\end{remark}

\begin{example}
    设总体分布$U(0,\theta)$,而$x_1,x_2,\cdots,x_n$是样本。
    $\theta$的似然函数为
\begin{eqnarray*}
    L(\theta) &=& \prod_{i=1}^n \frac{1}{\theta} I(0 < x_i < \theta)\\
    &=&\frac{1}{\theta^n} I( 0 < x_1,x_2,\cdots,x_n < \theta)\\
    &=&\frac{1}{\theta^n} I( 0 < x_{(1)}\leq x_{(n)} < \theta)
\end{eqnarray*}
要使得$L(\theta)$达到最大。对于示性函数取值应为1,而$1/(\theta^n)$是$\theta$的减函数。所以,$\theta$的取值应尽可能小,而$\theta$要大于$x_{(n)}$,所以$\Theta = (x_{(n)},\infty)$,而$\overline{\Theta} = [x_{(n)},\infty)$。因此,$\theta$的最大似然估计为
$$
\hat{\theta} = x_{(n)}.
$$
\end{example}
以上的例子都是一维参数的情况,以下我们举一个二维参数的例子。
\begin{example}
设总体分布$X\sim N(\mu,\sigma^2)$,$\theta = (\mu,\sigma^2)$是二维参数。而$x_1,x_2,\cdots,x_n$是样本。于是,似然函数为
$$
L(\mu,\sigma^2) = \prod_{i=1}^n p(x_i) = (2\pi \sigma^2)^{-n/2} \exp\left\{ - \frac{1}{2\sigma^2} \sum_{i=1}^n (x_i-\mu)^2\right\}.
$$
而其对数似然函数为
$$
l(\mu,\sigma^2) = \ln L(\mu,\sigma^2) = -(n/2)\ln (2\pi ) - (n/2) \ln (\sigma^2) -  \frac{1}{2\sigma^2} \sum_{i=1}^n (x_i-\mu)^2.
$$
对$l(\mu,\sigma^2)$分别关于$\mu$和$\sigma^2$求导,即
\begin{eqnarray*}
    \frac{\partial l(\mu,\sigma^2)}{\partial \mu} &=& \frac{1}{\sigma^2} \sum_{i=1}^n (x_i - \mu)  = 0\\
    \frac{\partial l(\mu,\sigma^2)}{\partial \sigma^2} &=& -\frac{n}{2\sigma^2} + \frac{1}{2(\sigma^2)^2} \sum_{i=1}^n (x_i-\mu)^2 =0
\end{eqnarray*}
由此解得
$$
\left\{
\begin{aligned}
   & \hat{\mu} = \bar{x},\\
   & \hat{\sigma}^2 = s_n^2.
\end{aligned}
\right.
$$
\end{example}
\begin{problem}
    如何求$\sigma$的最大似然估计?
\end{problem}
    
\begin{property}{(最大似然估计的不变性)}
    如果$\hat{\theta}$是$\theta$的最大似然估计, 则对任一函数$g(\theta)$,$g(\hat{\theta})$是其最大似然估计。
\end{property}

\section{EM算法(选修)}
最大似然估计是一种非常有效的参数估计方法,但是当分布里有多余参数或数据为截尾或缺失时,求最大似然估计就变得困难。EM算法由Dempster等人于1977年提出的,最初可以解决缺失数据场景下的最大似然估计求解问题。EM算法的核心想法是分两步骤:E步求期望;M步求最值。这里我们仅用一个例子来演示一下如何利用EM算法来求最大似然估计。
\begin{example}
    设一次试验可能有四个结果,其发生的概率分别为$\frac{1}{2}-\frac{\theta}{4}, \frac{1-\theta}{4},\frac{1+\theta}{4},\frac{\theta}{4}$,其中$\theta \in (0,1)$。现进行了197次试验,四种结果发生的次数分别为75,18,70,34。求$\theta$的最大似然估计。
\end{example}
\begin{solution}
设$y_1,y_2,y_3,y_4$表示四种结果发生的次数。这个总体分布是一个多项分布,故其似然函数为
\begin{eqnarray*}
L(y_1,y_2,y_3,y_4;\theta) &=& \frac{(y_1+y_2+y_3+y_4)!}{y_1!y_2!y_3!y_4!} \left(\frac{1}{2}-\frac{\theta}{4}\right)^{y_1} \left(\frac{1-\theta}{4}\right)^{y_{2}}\left(\frac{1+\theta}{4}\right)^{y_{3}}\left(\frac{\theta}{4}\right)^{y_{4}}\\
&\propto& (2-\theta)^{y_1} (1-\theta)^{y_2} (1+\theta)^{y_3} \theta^{y_4}.
\end{eqnarray*}
其对数似然函数为
$$
l(\theta) \propto y_1 \ln(2-\theta) + y_2 \ln (1-\theta) + y_3 \ln(1+\theta) + y_4 \ln \theta.
$$
我们可以对$l(\theta)$关于$\theta$求导,即
$$
\frac{\partial l(\theta)}{\partial \theta} = - \frac{y_1}{2-\theta} - \frac{y_2}{1-\theta} + \frac{y_3}{1+\theta} + \frac{y_4}{\theta}.
$$
方程$\frac{\partial l(\theta)}{\partial \theta} = 0$的求解是相对复杂的。于是,这里介绍EM算法的一般形式。我们引入两个潜变量$z_1,z_2$。假设第一种结果可以分为两部分,其发生的概率分别为$\frac{1-\theta}{4}$和$\frac{1}{4}$,令$z_1$和$y_1-z_1$分别表示落入这两部分的次数。再假设第三种结果分为两部分,其发生的概率分别为$\frac{\theta}{4}$和$\frac{1}{4}$,令$z_2$和$y_3-z_2$分别表示落入这两部分的次数。这里$z_1,z_2$是我们虚构的,它们具体的数值是不可观测的。

数据$(y_1,y_2,y_3,y_4,z_1,z_2)$,称为完全数据,而数据$(y_1,y_2,y_3,y_4)$,称为不完全数据。考虑我们可以观测到完全数据,其似然函数为
\begin{eqnarray*}
    &&L(\theta;y_1,y_2,y_3,y_4,z_1,z_2) \\
    &=& \frac{n!}{z_1!(y_1-z_1)!y_2!z_2!(y_3-z_2)!y_4!} \left(\frac{1}{4}\right)^{y_{1}-z_{1}}\left(\frac{1-\theta}{4}\right)^{z_{1}+y_{2}}\left(\frac{1}{4}\right)^{y_{3}-z_{2}}\left(\frac{\theta}{4}\right)^{z_{2}+y_{4}} \\
    &\propto& (1-\theta)^{z_1+y_2} (\theta)^{z_2+y_4}.
\end{eqnarray*}
其对数似然函数为
$$
l(\theta;y_1,y_2,y_3,y_4,z_1,z_2) = (z_1+y_2)\ln (1-\theta) + (z_2 + y_4) \ln \theta.
$$
基于此,可以解得其最大似然估计为
$$
\hat{\theta} = \frac{z_2 + y_4}{z_1 + y_2 + z_2 + y_4}.
$$
遗憾的是,我们并没有观测到$z_1,z_2$。但是我们知道,当$y_1,y_2,y_3,y_4$及$\theta$已知时,$$
z_1 \sim b\left(y_1, \frac{1-\theta}{2-\theta}\right),\quad z_{2} \sim b\left(y_3,\frac{\theta}{1+\theta}\right).
$$
EM算法具体如下:
\begin{itemize}
    \item E步:在观测数据$y_1,y_2,y_3,y_4$和已知$\theta = \theta^{(i)}$的条件下,我们求基于完全数据的对数似然函数的期望
    \begin{eqnarray*}
        Q(\theta | y_1,y_2,y_3,y_4,\theta^{(i)}) &=& E_{z_1,z_2} l(\theta;y_1,y_2,y_3,y_4,z_1,z_2) \\
        &=& (E_{\theta^{(i)}}(z_1|y_1) + y_2)\ln (1-\theta) + (E_{\theta^{(i)}}(z_2|y_3) + y_4) \ln(\theta)\\
        &=& \left( \frac{1-\theta^{(i)}}{2-\theta^{(i)}} y_1 + y_2 \right) \ln (1-\theta)  + \left( \frac{\theta^{(i)}}{1+\theta^{(i)}} y_3 + y_4 \right) \ln \theta.
    \end{eqnarray*}
    \item M步:求$ Q(\theta | y_1,y_2,y_3,y_4,\theta^{(i)})$关于$\theta$的最大值$\theta^{(i+1)}$,即
    $$
    \theta^{(i+1)} = \arg\max_{\theta} Q(\theta | y_1,y_2,y_3,y_4,\theta^{(i)})
    $$
    也就是说,对$Q(\theta | y_1,y_2,y_3,y_4,\theta^{(i)}) $关于$\theta$求导,即
    $$
    \frac{\frac{\theta^{(i)}}{1+\theta^{(i)}} y_3 + y_4 }{\theta^{(i+1)}} - \frac{\frac{1-\theta^{(i)}}{2-\theta^{(i)}} y_1 + y_2 }{1-\theta^{(i+1)}} =0
    $$
    由此可得
    $$
    \theta^{(i+1)} = \frac{\frac{\theta^{(i)}}{1+\theta^{(i)}} y_3 + y_4 }{ \frac{\theta^{(i)}}{1+\theta^{(i)}} y_3 + y_4 
 + \frac{1-\theta^{(i)}}{2-\theta^{(i)}} y_1 + y_2}.
    $$
    
\end{itemize}
\end{solution}

\begin{problem}
    EM算法是一个迭代算法。我们仍有以下三个问题:
    \begin{itemize}
    \item EM算法在什么条件下是收敛?
    \item EM算法如何才能停止?
    \item 如何选择EM算法的初始值?
    \end{itemize}
\end{problem}
\begin{note}
    \vspace{5cm}
\end{note}

\section{习题}
\begin{enumerate}
    \item 1. 设总体分布列/密度函数如下,$x_1,x_2,\cdots,x_n$是样本,试求未知参数的矩估计:
\begin{enumerate}
    \item $P(X = x) = (x-1)\theta^{2} (1-\theta)^{x-2},x=2,3,\cdots, 0<\theta<1$.
\item  $p(x;\theta) = (\theta+1) x^{\theta}, 0<x<1, \theta>0$.
\end{enumerate}

\item  设总体为$N(\mu,1)$,现对该总体观测$n$次,发现有$k$次观测值为正,使用频率替换方法求$\mu$的估计。

\item  设总体的概率密度函数如下,$x_1,x_2,\cdots,x_n$是样本,试求未知参数的最大似然估计:
\begin{enumerate}
    \item $p(x;\theta) = \frac{1}{2\theta} e^{-|x|/\theta}, \theta> 0$;
\item $p(x;\theta_1,\theta_2) = \frac{1}{\theta_2-\theta_1}, \theta_1 < x < \theta_2$
\end{enumerate}

\item (选做)
    众所周知,双胞胎可分为同卵双胞胎与异卵双胞胎。在一项针对双胞胎的研究中,研究者关心的是一对双胞胎是同卵双胞胎的概率,记为$p$,而且研究者也关心一个孩子是男孩的概率,记为$q$. 在这项研究中,研究者招募到了$n$对双胞胎(包括龙凤胎),其中$n_1$是两个男孩的双胞胎,$n_2$是两个女孩的双胞胎,$n_3= n-(n_1+n_2)$是龙凤胎(一个男孩一个女孩)。另外,研究者虽然知道不同性别的双胞胎一定不是同卵双胞胎,但并不知道其中哪些相同性别的双胞胎为同卵双胞胎。
    \begin{enumerate}
        \item 利用EM算法,写出$\bm{\theta} = (p,q)$的最大似然估计的形式。(提示:构造合适的潜变量,定义似然函数,写出EM算法中的E步和M步)
        \item 在$n = 1000, n_1 = 432, n_2 = 232, n_3 = 336$时,利用Python计算出$\bm{\theta}$的最大似然估计。
    \end{enumerate}


\end{enumerate}

