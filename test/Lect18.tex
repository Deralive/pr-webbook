\chapter{充分统计量}
\begin{introduction}
	\item Prob $\&$ Stat\quad 5.5
\end{introduction}
\section{引导问题}
统计量是样本的一种函数,是对样本中所蕴含信息的一种压缩。类似于图片的有损压缩和无损压缩,怎样的统计量是对样本的一种无损压缩?


\section{定义}
为了阐述清楚充分统计量,我们先讲以下这个简单的例子。
\begin{example}
	30秒口算是小学数学课课前一项测试。在表中是四位同学的20题的结果。
	
	\begin{table}[h]
		\centering 
		\begin{tabular}{ c cccc}
			\hline
			题号 & 学生1 & 学生2 & 学生3 & 学生4\\ 
			\hline
			1 & $\surd$ & $\times$ & $\surd$ & $\times$ \\
			2 & $\times$ & $\times$ & $\surd$ & $\times$ \\
			3 & $\surd$ & $\surd$ & $\surd$ & $\surd$ \\
			4 & $\surd$ & $\surd$ & $\surd$ & $\surd$ \\
			5 & $\surd$ & $\surd$ & $\times$ & $\surd$ \\
			6 & $\surd$ & $\surd$ & $\surd$ & $\surd$ \\
			7 & $\surd$ & $\surd$ & $\surd$ & $\surd$ \\
			8 & $\times$ & $\surd$ & $\surd$ & $\surd$ \\
			9 & $\surd$ & $\surd$ & $\times$ & $\surd$ \\
			10 & $\surd$ & $\surd$ & $\surd$ & $\surd$ \\
			11 & $\surd$ & $\surd$ & $\surd$ & $\surd$ \\
			12 & $\surd$ & $\surd$ & $\surd$ & $\times$ \\
			13 & $\surd$ & $\times$ & $\surd$ & $\surd$ \\
			14 & $\times$ & $\surd$ & $\surd$ & $\surd$ \\
			15 & $\surd$ & $\surd$ & $\times$ & $\surd$ \\
			16 & $\surd$ & $\surd$ & $\surd$ & $\surd$ \\
			17 & $\times$ & $\surd$ & $\surd$ & $\times$ \\
			18 & $\surd$ & $\surd$ & $\surd$ & $\surd$ \\
			19 & $\surd$ & $\surd$ & $\surd$ & $\surd$ \\
			20 & $\surd$ & $\times$ & $\times$ & $\surd$ \\
			\hline
			总分 & 16 & 16 & 16 & 16 \\
			\hline
		\end{tabular}
	\end{table}
 通过上述数据,你的直观感受是什么?
 
我们重新定义一下这个问题。记$x_{i,j}$为第$j$个学生在第$i$道题目的结果,若正确,$x_{i,j}$取值为$1$;否则为$0$,$i=1,2,\cdots,20,j=1,2,3,4$。
    以学生1为例,可观测到一组样本
        $$\{1,0,1,1,1,
             1,1,0,1,1,
             1,1,1,0,1,
             1,0,1,1,1\}.$$
    以学生2为例,可以观测到另一组样本
     $$\{0,0,1,1,1,
         1,1,1,1,1,
          1,1,0,1,1,
           1,1,1,1,0\}.$$
      这里的1或者0都是$x_{i,j}$具体的取值。也就是说,对于学生$j$,可以得到一组样本
      $\{x_{1,j},x_{2,j},\cdots,x_{20,j}\}$,而总分
      $$
      \sum_{i=1}^{20} x_{i,j}
      $$
      是样本的统计量。显然,给定样本,我们可以知道统计量的信息(根据样本我们能够计算统计量),但是根据统计量,我们无法得知样本的所有信息(根据总分,我们无法反推出学生$j$在每道题是否回答正确的结果)。因此,在统计量对样本加工或者简化的过程中,信息被丢失了。
      \end{example}
  \begin{problem}
 在构建统计量时,哪些信息可以丢失?哪些信息不可以丢失?
 \end{problem}
在参数模型中,我们使用参数化的概率质量函数或概率密度函数来刻画样本的信息。为了不区分概率质量函数和概率密度函数,我们这里统称为\textbf{概率函数},记为$p(x;\bm{\theta})$,其中$\bm{\theta}$是未知参数。一旦参数确定,参数模型是唯一确定的。因此,参数模型中的未知参数包含的就是“有用信息”。
如果我们希望统计量不损失信息,本质上就要求了统计量囊括了未知参数的一切信息。  
      
\begin{example} 
  对于每一个学生,其以一定概率能够回答正确任意一道口算题,假定其总体分布为$b(1,p)$。每一道口算可以看作独立同分布的样本。于是,样本的符号可记为
  $$x_1,x_2,\cdots,x_n$$
  其中,这里样本量$n$取20。这里用$n$表示只是为了更一般的情况。
  
  对于样本来说,其联合概率(质量)函数为
  \begin{eqnarray*}
  	P(X_1 = x_1,X_2=x_2,\cdots,X_n=x_n) &=& p_{(X_1,X_2,\cdots,X_n)}(x_1,x_2,\cdots,x_n) \\
  	&=& \prod_{i=1}^n p_{X_i}(x_i) \\
  	&=& \prod_{i=1}^n P(X_i = x_i)\\ &=&\prod_{i=1}^n p^{x_i}(1-p)^{1-x_i}\\
  	&=& p^{\sum_{i=1}^n x_i} (1-p)^{n - \sum_{i=1}^n x_i}. 
  \end{eqnarray*}
这是样本所有的信息,可以看出其依赖于未知参数$p$。

接下来,我们来考虑统计量$$
T = \sum_{i=1}^n X_i
$$
的分布。
根据所学习到的知识,$T$的分布为$b(n,p)$,其概率(质量)函数为
$$
P(T = t) = p_T(t) = C_{n}^t p^{t} (1-p)^{n-t}.
$$

然后,在已知统计量的分布条件下,我们来考虑样本中剩余信息的分布,这里我们用条件概率(质量)函数来表示,即
\begin{eqnarray*}
	P(X_1=x_1,X_2 = x_2,\cdots,X_n = x_n | T = t) &=& \frac{P((X_1=x_1,X_2 = x_2,\cdots,X_n = x_n ,T = t)}{P(T=t)}\\
	&=&  \left\{
	\begin{aligned}
		&\frac{1}{C_{n}^t}, & \text{如果$t = \sum_{i=1}^n x_i$}.\\
		&0 , &  \text{如果$t \neq \sum_{i=1}^n x_i$}.
		\end{aligned}
	\right.
\end{eqnarray*}
我们发现这个条件分布竟然与未知参数$p$无关,这意味着,这个统计量已经囊括了样本中所有有用的信息,也就是数理统计里所定义的“充分的”。
\end{example}


\begin{definition}
设$X \sim F(x;\theta),x_1,\cdots,x_n$是来自某个总体的样本,总体分布函数为$F(x;\theta)$,统计量$T=T(x_1,x_2,\cdots,x_n)$。
如果在给定$T$的取值后,$x_1,x_2,\cdots,x_n$的条件分布与$\theta$无关,则称$T$为$\theta$的充分统计量。
\end{definition}

接下来,我们来看另一个例子。
\begin{example}
设$x_1,x_2,\cdots,x_n$是来自正态分布$N(\mu,1)$的样本,$T = \bar{x}$,则$T$是充分的。
\end{example}
\begin{proof}
我们可知,$x_1,x_2,\cdots,x_n$的联合密度函数为
\begin{eqnarray*}
    p(x_1,x_2,\cdots,x_n;\mu) &=&  \prod_{i=1}^n p(x_i) \\
    &=& \prod_{i=1}^n \frac{1}{\sqrt{2\pi }} \exp\left\{-\frac{1}{2}(x_i - \mu)^2\right\}.
\end{eqnarray*}
因为统计量
$$
T = \bar{x} \sim N(\mu,\frac{1}{n}),
$$
所以,$T$的密度函数为
$$
p_T(t) = \frac{1}{\sqrt{2\pi/n}}\exp\left\{-\frac{1}{2\sigma^2/n} (t-\mu)^2\right\}.
$$
根据定义来判断一个统计量是否是充分的,需要考虑一个条件概率(密度)函数,即
$$
p(x_1,x_2,\cdots,x_n|t) = \frac{p(x_1,x_2,\cdots,x_n,t)}{p_T(t)}
$$ 
我们注意到,$p(x_1,x_2,\cdots,x_n,t)$这是一个退化分布。一旦确定$t$之后,不是所有的样本都是自由的,$x_n $可以改写成$nt - (x_1+\cdots+x_{n-1})$。于是,
\begin{eqnarray*}
   && p(x_1,x_2,\cdots,x_n,t) \\
   &=& p(x_1,x_2,\cdots,x_{n-1},t)\\
    &=& (2\pi)^{-n/2}\exp\left\{-\frac{1}{2}\left( \sum_{i=1}^{n-1}(x_i - \mu)^2 + \left(nt - \sum_{i=1}^{n-1} x_i -\mu \right)^2 \right)\right\}\\
    &=& (2\pi)^{-n/2} \exp\left\{
    -\frac{1}{2}\left( \sum_{i=1}^{n-1} x_i^2 - 2\mu(nt - x_n) + (n-1)\mu^2 + \left(nt - \sum_{i=1}^{n-1} x_i\right)^2 - 2(x_n \mu) + \mu^2
    \right)
    \right\}\\
    &=&(2\pi)^{-n/2} \exp\left\{ -\frac{1}{2}\left( n\mu^2 - 2n \mu t + \sum_{i=1}^n x_i^2\right) \right\}
\end{eqnarray*}
因此,条件概率函数为
\begin{eqnarray*}
    p(x_1,x_2,\cdots,x_n|t) &=& \frac{p(x_1,x_2,\cdots,x_n,t)}{p(t)} \\
    &=& \frac{(2\pi)^{-n/2} \exp\left\{ -\frac{1}{2}\left( n\mu^2 - 2n \mu t + \sum_{i=1}^n x_i^2\right) \right\}}{ (2\pi/n)^{-1/2} \exp\left\{ -\frac{1}{2/n} \left(t-\mu\right)^2 \right\}} \\
    &=& \frac{(2\pi)^{-n/2}}{(2\pi/n)^{-1/2} } \exp\left\{
    -\frac{1}{2}\sum_{i=1}^n x_i^2 + \frac{n}{2}t^2
    \right\}
\end{eqnarray*}
这与参数$\mu$无关。由此得证。
\end{proof}

\section{因子分解定理}
根据以上例子,我们发现定义往往可以论证某一个统计量是充分统计量。但是难以通过定义来寻找那个统计量是充分的。以下我们有因子分解定理,可以帮助我们来寻找。
\begin{theorem}{因子分解定理}
设总体概率函数为$f(x;\theta)$,$x_1,\cdots,x_n$是样本,则$T = T(x_1,x_2,\cdots,x_n)$为充分统计量的充分必要条件是:存在两个函数$g(t,\theta)$和$h(x_1,x_2,\cdots,x_n)$使得对任意的$\theta$和任一组观测值$x_1,x_2,\cdots,x_n$,有
\begin{eqnarray*}
    f(x_1,x_2,\cdots,x_n;\theta) = g(T(x_1,x_2,\cdots,x_n),\theta)h(x_1,x_2,\cdots,x_n),
\end{eqnarray*}
其中,$g(t,\theta)$是通过统计量$T$的取值而依赖于样本的。
\end{theorem}
\begin{proof}
本定理的一般性结果的证明过程超过了本课程的内容。以下仅考虑离散随机比纳凉的证明。此时概率函数为
$$
f(x_1,x_2,\cdots,x_n;) = P(X_1 = x_1,X_2=x_2,\cdots,X_n = x_n;\theta).
$$
首先证明必要性。假定$T$是充分统计量,所以
\begin{eqnarray*}
    P(X_1 = x_1,X_2=x_2,\cdots,X_n = x_n|T=t)
\end{eqnarray*}
与$\theta$无关。我们将其记为$h(x_1,x_2,\cdots,x_n)$。令$A(t) = \left\{x_1,x_2,\cdots,x_n | T(x_1,x_2,\cdots,x_n) = t\right\}$。当样本$(x_1,x_2,\cdots,x_n) \in A(t)$时有
$$
\{T = t\} \subset \{X_1 = x_1,X_2=x_2,\cdots,X_n = x_n\},
$$
故
\begin{eqnarray*}
    P(X_1 =x_1,X_2=x_2,\cdots,X_n = x_n ) &=&  
    P(X_1 =x_1,X_2=x_2,\cdots,X_n = x_n , T= t;\theta) \\
    &=& P(X_1 =x_1,X_2=x_2,\cdots,X_n = x_n | T= t) P(T =t;\theta)\\
    &=& h(x_1,x_2,\cdots,x_n) g(t,\theta).
\end{eqnarray*}
其中$g(t,\theta) = P(T = t;\theta)$,而$h(x_1,x_2,\cdots,x_n) = P(X_1=x_1,X_2=x_2,\cdots,X_n = x_n|T=t)$与$\theta$无关。因此,必要性得证。

其次证明充分性。由于
\begin{eqnarray*}
    P(T=t;\theta) &=& \sum_{(x_1,x_2,\cdots,x_n)\in A(t)} P(X_1 = x_1,X_2=x_2,\cdots,X_n=x_n;\theta)\\
    &=&  \sum_{(x_1,x_2,\cdots,x_n)\in A(t)} g(t,\theta) h(x_1,x_2,\cdots,x_n).
\end{eqnarray*}
对任意$(x_1,x_2,\cdots,x_n)$和$t$,满足$(x_1,x_2,\cdots,x_n)\in A(t)$有
\begin{eqnarray*}
    P(X_1 = x_1,X_2=x_2,\cdots,X_n=x_n|T=t) &=& \frac{P(X_1 = x_1,X_2=x_2,\cdots,X_n=x_n,T = t;\theta)}{P(T=t;\theta)}\\
    &=& \frac{P(X_1 = x_1,X_2=x_2,\cdots,X_n=x_n;\theta)}{P(T=t;\theta)}
    \\
    &=& \frac{g(t,\theta) h(x_1,x_2,\cdots,x_n)}{ g(t,\theta)\sum_{(x_1,x_2,\cdots,x_n)\in A(t)}  h(x_1,x_2,\cdots,x_n)}\\
    &=&  \frac{h(x_1,x_2,\cdots,x_n)}{ \sum_{(x_1,x_2,\cdots,x_n)\in A(t)}  h(x_1,x_2,\cdots,x_n)}
    .
\end{eqnarray*}
该分布与$\theta$无关,这证明了充分性。
\end{proof}
\begin{remark}
    这里$T$可以是一维的,也可以是多维的。
\end{remark}

\begin{example}
设$x_1,x_2,\cdots,x_n$是来自于$b(1,\theta)$的样本。于是,$x_1,x_2,\cdots,x_n$的概率函数为
$$
P(X_1=x_1,X_2=x_2,\cdots,X_n=x_n) = \prod_{i=1}^n \theta^{x_i}(1-\theta)^{1-x_i} = \theta^{\sum_{i=1}^n x_i} (1-\theta)^{n - \sum_{i=1}^n x_i}.
$$
令$t = \sum_{i=1}^n x_i$,$g(t,\theta) = (1-\theta)^{n} \left(\theta/(1-\theta)\right)^t $且$h(x_1,x_2,\cdots,x_n) = 1$。根据因子分解定理,$t$是$\theta$的充分统计量。
\end{example}

\begin{example}
    设$x_1,x_2,\cdots,x_n$是来自于$N(\mu,\sigma^2)$的样本。 
\begin{enumerate}
    \item 若$\sigma^2 = \sigma_0^2$,其中$ \sigma_0^2$已知。$x_1,x_2,\cdots,x_n$的概率函数为
    \begin{eqnarray*}
        p(x_1,x_2,\cdots,x_n) &=
        &(2\pi\sigma_0^2)^{-n/2} \exp\left\{ - \frac{1}{2\sigma_0^2} \sum_{i=1}^n (x_i-\mu)^2\right\}\\
        &=&(2\pi\sigma_0^2)^{-n/2}  \exp\left\{ - \frac{1}{2\sigma_0^2} \left(\sum_{i=1}^n x_i^2 - 2n \bar{x}\mu + n\mu^2\right)\right\}\\
        &=& (2\pi\sigma_0^2)^{-n/2}  \exp\left\{ - \frac{1}{2\sigma_0^2} \sum_{i=1}^n x_i^2 + \frac{n}{\sigma_0^2} \bar{x}\mu -\frac{n}{2\sigma_0^2}\mu^2\right\}
    \end{eqnarray*}
令$t = \bar{x}$,$$g(t,\mu) = \exp\{n\bar{x}\mu / \sigma_0^2 - n\mu^2/(2\sigma_0^2)\},$$
和$$h(x_1,x_2,\cdots,x_n) = (2\pi\sigma_0^2)^{-n/2} \exp\{ -\sum_{i=1}^n x_i^2/(2\sigma_0^2)\}.$$根据因子分解定理,$t=\bar{x}$是充分统计量。
\item 若$\mu = 0$。$x_1,x_2,\cdots,x_n$的概率函数为
\begin{eqnarray*}
     p(x_1,x_2,\cdots,x_n) &=
        &(2\pi\sigma^2)^{-n/2} \exp\left\{ - \frac{1}{2\sigma^2} \sum_{i=1}^n (x_i)^2\right\}
\end{eqnarray*}
令$$
t = \sum_{i=1}^n x_i^2,
$$
$$g(t,\sigma^2) = (2\pi\sigma^2)^{-n/2} \exp\left\{ - \frac{1}{2\sigma^2} \sum_{i=1}^n (x_i)^2\right\}$$
和
$$
h(x_1,x_2,\cdots,x_n) = 1.
$$
根据因子分解定理,$t = \sum_{i=1}^n x_i^2$是充分统计量。

\item 令$\theta = (\mu,\sigma_0^2)'$。$x_1,x_2,\cdots,x_n$的概率函数为
    \begin{eqnarray*}
        p(x_1,x_2,\cdots,x_n) &=
        &(2\pi\sigma^2)^{-n/2} \exp\left\{ - \frac{1}{2\sigma^2} \sum_{i=1}^n (x_i-\mu)^2\right\}\\
        &=&(2\pi\sigma^2)^{-n/2}  \exp\left\{ - \frac{1}{2\sigma^2} \left(\sum_{i=1}^n x_i^2 - 2n \bar{x}\mu + n\mu^2\right)\right\}\\
        &=& (2\pi\sigma^2)^{-n/2}  \exp\left\{ - \frac{1}{2\sigma^2} \sum_{i=1}^n x_i^2 + \frac{n}{\sigma^2} \bar{x}\mu -\frac{n}{2\sigma^2}\mu^2\right\}
    \end{eqnarray*}
令$$
\bm{t} = (t_1,t_2)' = (\bar{x},\sum_{i=1}^n x_i^2)',
$$
$$
g(t,\mu,\sigma^2) = (2\pi\sigma^2)^{-n/2}  \exp\left\{ - \frac{1}{2\sigma^2} \sum_{i=1}^n x_i^2 + \frac{n}{\sigma^2} \bar{x}\mu -\frac{n}{2\sigma^2}\mu^2\right\}
$$
和
$$
h(x_1,x_2,\cdots,x_n) = 1.
$$
根据因子分解定理,$\bm{t} = (t_1,t_2)' = (\bar{x},\sum_{i=1}^n x_i^2)'$是充分统计量。
\end{enumerate}
\end{example}

以下例子是一个拓展。

\begin{example}
    总体分布为指数型分布族,即其概率函数为
    $$
    p(x;\theta) = C(\theta) \exp\left\{g(\theta) t(x)\right\}h(x).
    $$
    如果$x_1,x_2,\cdots,x_n$是样本,则$\sum_{i=1}^n t(x_i)$是充分统计量。
\end{example}
\begin{remark}
    $\sum_{i=1}^n t(x_i)$ 不仅是充分统计量,还是充分完备统计量。
\end{remark}

已经证明某个统计量是充分的。如果我们想要论证另一个统计量也是充分的,那么并不需要从定义或因子分解定理直接出发,我们有以下定理来解决这个问题。
\begin{theorem}
设统计量$T$是充分统计量,而$S$也是一个统计量。如果$T$可以表示为$S$的函数,即$T=\varphi(S)$,那么$S$也是充分统计量。
\end{theorem}

\section{习题}
\begin{enumerate}
    \item 设$x_1,x_2,\cdots,x_n$是来自几何分布
$$P(X=x) = \theta(1-\theta)^{x}$$
的样本,证明$T = \sum_{i=1}^n x_i$是充分统计量。

\item  设$x_1,x_2,\cdots,x_n$来自以下分布的样本,试给出充分统计量。
\begin{enumerate}
    \item 均匀分布$U(0,\theta)$;
    \item 均匀分布$U(\theta_1,\theta_2)$。
\end{enumerate}

\item 设$x_1,x_2,\cdots,x_n$是来自正态总体$N(\mu,\sigma^2_1)$的样本,$y_1,y_2,\cdots,y_m$是来自于另一正态总体$N(\mu,\sigma^2_2)$的样本,这两个样本相互独立,试给出$(\mu,\sigma_1^2,\sigma_2^2)$的充分统计量。

\item 设$x_1,x_2,\cdots,x_n$是来自帕累托(Pareto)分布
$$
p(x;\theta) = \theta \cdot a^{\theta} x^{-(\theta+1)}, x>a, \theta>0
$$
的样本($a>0$已知),试给出一个充分统计量。
\end{enumerate}