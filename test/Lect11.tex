\chapter{条件分布与条件期望}
\begin{introduction}
  \item Intro to Prob\quad3.5 3.6
  \item Prob $\&$ Stat\quad 3.5
\end{introduction}
\section{引导问题:公平 vs 不公平}
\begin{example}{(Simpson悖论)}
某一年北美某高校的商学院和法学院录取率如表\ref{tab:lect10_1}所示。从数据来看,在录取时该高校是否存在性别歧视?因女生的性别,而导致学生不予录取。
\begin{table}[ht]
  \caption{某高校两个学院的总录取人数\label{tab:lect10_1}}
  \centering
  \begin{tabular}{ccc}
  \toprule
    & {全部男生} 
    & {全部女生} \\
\midrule
    录取 & 209
    & 143
     \\
    未录取      & 95
    & 110
    \\
    录取率& 67.43$\%$
    & 56.52$\%$
    \\
  \bottomrule
  \end{tabular}
\end{table}

商学院和法学院实际的录取人数如表\ref{tab:lect10_2}所示。无论在法学院还是在商学院,女生的录取率都高于男生的录取率。
\begin{table}[ht]
  \caption{某高校两个学院各自的录取人数\label{tab:lect10_2}}
  \centering
  \begin{tabular}{ccccc}
  \toprule
  & \multicolumn{2}{c}{法学院} &  \multicolumn{2}{c}{商学院}\\
  \cline{2-3}\cline{4-5}
    & 男生 
    & 女生
    & 男生
    &女生\\
\midrule
    录取 & 8
    & 51
    & 201
    & 92
     \\
    未录取      & 45
    & 101
    & 50
    & 9
     \\
    录取率& 15.09$\%$
    & 33.55$\%$
    & 80.08$\%$
    & 91.09$\%$
    \\
  \bottomrule
  \end{tabular}
\end{table}
\end{example}
\begin{problem}
但是为什么从总体上来看,我们会认为女生的录取低?
\end{problem}
\begin{note}
\vspace{5cm}
\end{note}

\section{离散场合下的条件分布}
设二维离散随机变量$(X,Y)$的联合分布列为$$p_{i j}=P\left(X=x_{i}, Y=y_{j}\right), i=1,2, \cdots;\quad j=1,2, \cdots.$$
\begin{definition}
对一切使$P\left(Y=y_{j}\right)=p_{\cdot j}=\sum_{i=1}^{\infty} p_{i j}>0$的$y_{j}$,称
\begin{eqnarray*}
p_{i  |  j} &=&P\left(X=x_{i}  |  Y=y_{j}\right) \\
&=&\frac{P\left(X=x_{i}, Y=y_{j}\right)}{P\left(Y=y_{j}\right)} \\
&=&\frac{p_{i j}}{p_{\cdot j}  }
\end{eqnarray*}
为给定$Y=y_{j}$条件下$X$的条件分布列,$i=1,2,\cdots$。
\end{definition}
\begin{definition}
给定$Y=y_{j}$条件下$X$的条件分布函数为
$$F\left(x  |  y_{j}\right)=\sum_{x_{i} \leq x} P\left(X=x_{i}  |  Y=y_{j}\right)=\sum_{x_{i} \leq x} p_{i  |  j}$$
\end{definition}

\begin{example}
设随机变量$X$与$Y$独立,且$x \sim P\left(\lambda_{1}\right), Y \sim P\left(\lambda_{2}\right)$。在已知$X+Y=n$的条件下,求$X$的条件分布。
\end{example}
\begin{solution}
在例\ref{ex:lect9_1}中已经证明了$X+Y\sim P(\lambda_{1}+\lambda_{2})$。所以
\begin{eqnarray*}
P(X=k  |  X+Y=n)&=&\frac{P(X=k, X+Y=n)}{P(X+Y=n)}\\
&=&\frac{P(X=k, Y=n-k)}{P(X+Y=n)} \\
&=& \frac{P(X=k) \cdot P(Y=n-k)}{P(X+Y=n)}\\
&=&\frac{\frac{\lambda_{1}^{k}}{k !} e^{-\lambda_{1}} \cdot \frac{\lambda_{2}^{(n-k)}}{(n-k) !} e^{-\lambda_{2}}}{\frac{\left(\lambda_{1}+\lambda_{2}\right)^{n}}{n !} e^{-\left(\lambda_{1}+\lambda_{2}\right)}}\\
&=&\frac{n !}{k !(n-k) !}\left(\frac{\lambda_{1}}{\lambda_{1}+\lambda_{2}}\right)^{k}\left(1-\frac{\lambda_{1}}{\lambda_{1}+\lambda_{2}}\right)^{n-k}
\end{eqnarray*}
因此,在$X+Y=n$的条件下,$X$的条件分布为二项分布$b\left(n, \frac{\lambda_{1}}{\lambda_{1}+\lambda_{2}}\right).$
\end{solution}
\begin{example}
设在一段时间内进入某一商店的顾客人数$X$服从泊松分布$P(\lambda)$,每位顾客购买某商品的概率为$p$,并且每位顾客是否购买该种物品相互独立。求进入商店的顾客购买这种物品的人数$Y$的分布列。
\end{example}
\begin{solution}
由题可知,
$$P(X=m)=\frac{\lambda^{m}}{m !} e^{-\lambda} \quad m=0,1,2, \cdots$$
在进入商店的人数$X=m$的条件下,购买某种物品的人数$Y$的条件分布为二项分布$b(m,p)$,即
$$P(Y=k  |  X=m)=\begin{pmatrix}
m \\
k
\end{pmatrix} p^{k}(1-p)^{m-k}, \quad k=0,1, \cdots, m.$$
由全概率公式有$$\begin{aligned}
P(Y=k) &=\sum_{m=k}^{+\infty} P(X=m) \cdot P(Y=k  |  X=m) \\
&=\sum_{m=k}^{+\infty} \frac{\lambda^{m}}{m !} e^{-\lambda} \cdot \frac{m !}{k !(m-k) !} p^{k}(1-p)^{m-k} \\
&=\frac{1}{k !} e^{-\lambda}(\lambda p)^{k} \cdot \sum_{m=k}^{+\infty} \frac{(\lambda(1-p))^{m-k}}{(m-k) !} \\
&=\frac{1}{k !}(\lambda p)^{k} e^{-\lambda} \sum_{m=0}^{+\infty} \frac{(\lambda(1-p))^{m}}{m !} e^{-\lambda(1-p)} \cdot e^{\lambda(1-p)} \\
&=\frac{(\lambda p)^{k}}{k !} e^{-\lambda p} ,\quad k=0,1,2, \cdots
\end{aligned}$$
因此,$Y\sim P(\lambda p)$。
\end{solution}

\section{连续场合下的条件分布}
设二维连续随机变量$(X,Y)$的联合密度为$p(x,y)$,边际密度函数为$p_{X}(x)$和$p_{Y}(y)$。
在离散随机变量场合,其条件分布函数为$P(X\le x |  Y=y)$,但是连续随机变量取某个值的概率为零,即$P(Y=y)=0$。所以无法用条件概率直接计算$P(X\le x |  Y=y)$。一个自然的想法是,将$P(X\le x |  Y=y)$看成当$h \rightarrow 0$时$P(X\le x |  y\le Y\le y+h)$的极限,即
\begin{eqnarray*}
P(X \leq x  |  Y=y) &=&\lim _{h \rightarrow 0} P(X \leqslant x  |  y \leqslant Y \leqslant y+h) \\
&=&\lim _{h \rightarrow 0} \frac{P(X \leq x, y \leq Y \leq y+h)}{P(y \leq Y \leq y+h)} \\
&=&\lim _{h \rightarrow 0} \frac{\int_{-\infty}^{x} \int_{y}^{y+h} p(u, v) \text{d} v \text{d} u}{\int_{y}^{y+h} p(v) \text{d} v} \\
&=&\lim _{h \rightarrow 0} \frac{\int_{-\infty}^{x}\left(\frac{1}{h} \int_{y}^{y+h} p(u, v) \text{d} v\right) \text{d} u}{\frac{1}{h} \int_{y}^{y+h} P_{Y}(v) \text{d} v}
\end{eqnarray*}
当$p_{Y}(y),p(x,y)$在$y$处连续时,由积分中值定理可得\begin{eqnarray*}
\lim _{h \rightarrow 0} \frac{1}{h} \int_{y}^{y+h} p_{Y}(v) \text{d} v
&=&\lim _{h \rightarrow 0} \frac{1}{h} \cdot(y+h-y) \cdot p_{Y}(y+\delta h)=p_{Y}(y) \\
\lim _{h \rightarrow 0} \frac{1}{h} \int_{y}^{y+h} p(u, v) \text{d} v&=&\lim _{h \rightarrow 0} \frac{1}{h}(y+h-y) \cdot p(u, y+\delta h)=p(u, y)
\end{eqnarray*}
所以
$$P(X \leq x  |  Y=y)=\int_{-\infty}^{x} \frac{p(u, y)}{p_{Y}(y)} \text{d} u.$$

\begin{definition}
对一切使$p_{Y}(y)>0$的$y$,给定$Y=y$条件下$X$的条件分布函数和条件密度函数分别为
\begin{eqnarray*}
F(x  |  y)&=&\int_{-\infty}^{x} \frac{p(u, y)}{p_{Y}(y)} \text{d} u, \\
p(x  |  y)&=&\frac{p(x, y)}{p_{Y}(y)}.
\end{eqnarray*}
\end{definition}

\begin{example}
设$(X,Y)$服从二维正态分布$$N_{2}\left(\begin{pmatrix}
\mu_{1} \\
\mu_{2}
\end{pmatrix},\begin{pmatrix}
    \sigma_{1}^{2} & \rho \sigma_{1} \sigma_{2} \\
\rho \sigma_{1} \sigma_{2} & \sigma_{2}^{2}
\end{pmatrix}
\right)$$
由边际分布可知,$X$服从正态分布$N(\mu_{1},\sigma_{1}^{2})$,$Y$服从正态分布$N(\mu_{2},\sigma_{2}^{2})$。
在给定$Y=y$的条件下,$X$的边际密度函数为
\begin{eqnarray*}
 p(x |  y) &=&\frac{p(x, y)}{p_{Y}(y)} \\
&=&\frac{\frac{1}{2 \pi \sqrt{\sigma_{1}^{2} \sigma_{2}^{2}\left(1-\rho ^{2}\right)}} \exp \left\{-\frac{1}{2\left(1-\rho ^{2}\right)}\left(\frac{(x-\mu_{1})^{2}}{\sigma_{1}^{2}}-2 \rho  \frac{\left(x-\mu_{1}\right)\left(y-\mu_{2}\right)}{\sigma_{1} \sigma_{2}}+\frac{\left(y-\mu_{2}\right)^{2}}{\sigma_{2}^{2}}\right)\right\}}{\frac{1}{\sqrt{2 \pi \sigma_{2}^{2}}} \exp \left\{-\frac{1}{2 \sigma_{2}^{2}}\left(y-\mu_{2}\right)^{2}\right\}} \\
&=&\frac{1}{\sqrt{2 \pi}} \cdot \frac{1}{\sqrt{\sigma_{1}^{2}(1-\rho ^{2})}} \exp \left\{-\frac{1}{2\left(1-\rho^{2}\right)}\left(\frac{\left(x-\mu_{1}\right)^{2}}{\sigma_{1}^{2}}-2 \rho \frac{\left(x-\mu_{1}\right)\left(y-\mu_{2})\right.}{\sigma_{1} \sigma_{2}}+\frac{\rho^{2}\left(y-\mu_{2}\right)^{2}}{\sigma_{2}^{2}}\right)\right\}\\
&=&\frac{1}{\sqrt{2 \pi \sigma_{1}^{2}(1-\rho ^{2})}} \exp \left\{-\frac{1}{2 \sigma_{1}^{2}\left(1-\rho^{2}\right)}\left(x-\left(\mu_{1}+\rho \cdot \frac{\sigma_{1}}{\sigma_{2}}\left(y-\mu_{2}\right)\right)\right)^{2}\right\}
   \end{eqnarray*}
因此,在$Y=y$的条件下,$X$的条件分布为
$$ N\left(\mu_{1}+ \frac{\rho\sigma_{1}}{\sigma_{2}}\left(y-\mu_{2}\right), \sigma_{1}^{2}\left(1-\rho ^{2}\right)\right) .$$
\end{example}
\begin{remark}
在定义连续场合下的条件密度函数下,我们可以给出连续场合下的全概率公式及贝叶斯公式。因为
$$
p_Y(y)= \frac{p(x,y)}{p_Y(y)}
$$
所以,
$$
p(x,y) = p_Y(y)p_Y(y).
$$
于是,我们有
\begin{itemize}
    \item 全概率公式
    $$
    p_X(x) = \int_{-\infty}^{+\infty} p(x, y) \text{d} y =  \int_{-\infty}^{+\infty} p_{Y}(y) p(x  |  y) \text{d} y.
    $$
    \item 贝叶斯公式
    $$
    p(y|x) = \frac{p(x,y)}{p_{X}(x)} = \frac{p_Y(y)p(x|y)}{\int_{-\infty}^{+\infty} p_{Y}(y) p(x  |  y) \text{d} y}.
    $$
\end{itemize}
\end{remark}

\begin{example}
设$X\sim N(\mu,\sigma_{1}^{2})$且在给定$X = x$的条件下,$Y$的条件分布为$N(x,\sigma_{2}^{2})$,求$Y$的密度函数$p_{Y}(y)$。
\end{example}
\begin{solution}
由题可知,
\begin{eqnarray*}
    p_X(x) &=& \frac{1}{\sqrt{2\pi \sigma_1^2}} \exp\left\{-\frac{1}{2\sigma_1^2} (x-\mu)^2\right\},\\
    p(y|x) &=& \frac{1}{\sqrt{2\pi \sigma_2^2}}
    \exp\left\{-\frac{1}{2\sigma_2^2} (y-x)^2\right\}.
\end{eqnarray*}
所以,根据全概率公式可知,
\begin{eqnarray*}
P_{Y}(y)
&=&\int_{-\infty}^{+\infty}p_{X}(x)p(y  |  x) \text{d} x\\
&=&\int_{-\infty}^{+\infty} \frac{1}{\sqrt{2 \pi \sigma_{1}^{2}}} \exp \left\{-\frac{1}{2 \sigma_{1}^{2}}(x-\mu)^{2}\right\} \cdot \frac{1}{\sqrt{2 \pi \sigma_{2}^{2}}} \exp \left\{-\frac{1}{2 \sigma_{2}^{2}}(y-x)^{2}\right\} \text{d} x\\
&=&\frac{1}{2 \pi \sqrt{\sigma_{1}^{2} \sigma_{2}^{2}}} \int_{-\infty}^{+\infty} \exp \left\{-\frac{1}{2} \left(\frac{1}{\sigma_{1}^{2}} x^{2}-\frac{2 \mu}{\sigma_{1}^{2}} x+\frac{\mu^{2}}{\sigma_{1}^{2}}\right.\right.
\left.\left.+\frac{y^{2}}{\sigma_{2}^{2}}-\frac{2 y}{\sigma_{2}^{2}} x+\frac{x^{2}}{\sigma_{2}^{2}} \right)\right\} \text{d} x\\
&=& \int_{-\infty}^{+\infty} \frac{1}{\sqrt{2 \pi \frac{\sigma_{1}^{2} \sigma_{2}^{2}}{\sigma_{1}^{2}+\sigma_{2}^{2}}}} 
\exp \left\{-\frac{1}{2} \left(\left( \frac { 1 } { \sigma _ { 1 } ^ { 2 } } + \frac { 1 } { \sigma _ { 2 } ^ { 2 } } \right)
\left(x-\left(\frac{1}{\sigma_{1}^{2}}+\frac{1}{\sigma_{2}^{2}}\right)^{-1}\left(\frac{\mu}{\sigma_{1}^{2}}+\frac{y}{\sigma_{2}^{2}}\right)\right)^{2} \right)\right\}\text{d}x \\
&& \cdot \frac{1}{\sqrt{2 \pi\sigma_{1}^{2}+\sigma_{2}^{2}}}   \exp\left\{
-\left(\frac{\sigma_{1}^{2} \sigma_{2}^{2}}{\sigma_{1}^{2}+\sigma_{2}^{2}}\right)\left(\frac{\mu^{2}}{\sigma_{1}^{4}}+\frac{2 \mu y}{\sigma_{1}^{2} \sigma_{2}^{2}}+\frac{y^{2}}{\sigma_{2}^{4}}\right)+\frac{y^{2}}{\sigma_{2}^{2}}+\frac{\mu^{2}}{\sigma_{1}^{2}} \right\} \\
&=&\frac{1}{\sqrt{2 \pi\left(\sigma_{1}^{2}+\sigma_{2}^{2}\right)}} \exp \left\{-\frac{1}{2} \left(\frac{1}{\sigma_{1}^{2}+\sigma_{2}^{2}} y^{2}-\frac{2 \mu y}{\sigma_{1}^{2}+\sigma_{2}^{2}}+\frac{\mu^{2}}{\sigma_{1}^{2}+\sigma_{2}^{2}} \right)\right\}\\
&=&\frac{1}{\sqrt{2 \pi(\sigma_{1}^{2}+\sigma_{2}^{2})}} \exp \left\{-\frac{1}{2\left(\sigma_{1}^{2}+\sigma_{2}^{2}\right)}(y-\mu)^{2}\right\}
\end{eqnarray*}
因此,
$$Y\sim N(\mu,\sigma_{1}^{2}+\sigma_{2}^{2}).$$   
\end{solution}

\section{混合场合下的条件分布}
\begin{example}
    生活中,医生根据一些生理指标测量,如体温、血压等生化指标来进行医学诊断。
\end{example}

我们简化一下这个问题。令$A$是我们感兴趣的一个随机变量,而$P(A)$是事件$A$发生的概率。令$Y$是一个连续型随机变量,并假定已知条件密度函数$p(y|A)$和$p(y|\overline{A})$。

\begin{problem}
    在给定$Y$取值为$y$时,事件$A$发生的条件概率$P(A|Y=y)$是什么?如何计算?
\end{problem}

和连续场合下的条件分布有同一个问题,因为$Y$是连续型的随机变量,所以事件$\{Y = y\}$发生的概率为零。这里我们考虑$\{y \leq Y \leq y + \delta\}$,其中$\delta>0$。我们考虑
\begin{eqnarray*}
    P(A | Y= y) &\approx& P(A|y \leq Y \leq y+\delta)= \frac{P(A)P(y\leq Y\leq y+\delta | A)}{P(y \leq Y\leq y+\delta)}\\
    &\approx& \frac{P(A) p(y|A)\delta}{p(y)\delta} = \frac{P(A) p(y|A)}{p(y)}
\end{eqnarray*}
以上就是混合场景下条件分布的定义。由此,可以定义全概率公式和贝叶斯公式。
\begin{enumerate}
    \item 全概率公式为
    $$
    p_Y(y) = P(A) p(y|A) + P(\overline{A}) p(y|\overline{A}). 
    $$
    \item 贝叶斯公式为
    $$
    P(A|Y= y) = \frac{P(A)p(y|A)}{P(A)p(y|A) + P(\overline{A})p(y|\overline{A})}.
    $$
\end{enumerate}

如果已知$P(A|Y=y)$后,我们也可以推导出$p(y|A)$,即
$$
p(y|A) = \frac{p_Y(y) P(A|Y=y)}{P(A)} = \frac{p_Y(y) P(A|Y=y)}{\int_{-\infty}^{\infty} p_Y(y) P(A|Y=y) \text{d} y}.
$$

\begin{remark}
    上述公式中将随机事件$A$推广到离散型随机变量。
\end{remark}

\section{条件数学期望}
正如之前课程内容中介绍的,我们讨论过条件概率是符合概率的公理化定义。由此,根据条件概率,我们可以定义随机变量的\textbf{条件分布}。既然有分布,我们也可以定义这个分布的特征数。这里,我们首先介绍如何定义条件分布的数学期望——条件期望。
\begin{definition}{条件期望}\label{def:conditional_expectation}
条件分布的数学期望(若存在)称为条件期望,即$$E(X | Y=y)=\left\{\begin{aligned}
    &\sum_{i} x_{i} P\left(X=x_{i}  |  Y=y\right) , &\quad \text{离散场合},\\
  &\int_{-\infty}^{+\infty} x P(x  |  y) d x,&\quad \text{连续场合}.
\end{aligned}
\right.$$
\end{definition}

因为条件期望是一种期望,所以条件期望也满足期望的性质。
\begin{property}
 $E\left(a_{1} X_{1}+a_{2} X_{2}  |  Y=y\right)=a_{1} E\left(X_{1}  |  Y=y\right)+a_{2} E\left(X_{2}  |  Y=y\right)$;
\end{property}
\begin{note}
\vspace{4.5cm}
\end{note}


这里从另一个角度来看条件期望。在给定$Y=y$的条件下,$X$的条件分布会因$y$的取值不同而不同,从而导致了$E(X|Y=y)$亦是如此。所以,$E(X|Y=y)$可以看作$y$的函数。我们记$g(y) = E(X|Y=y)$。

对于随机变量$Y$,$g(Y)= E(X|Y)$是随机变量$Y$的函数。所以,条件期望$E(X|Y)$可以看作随机变量$Y$的函数,通常仍是一个随机变量。这里,我们讨论一下,其期望$E(E(X|Y))$是什么呢?

\begin{theorem}{重期望公式}
设$(X,Y)$是二维随机变量且$E(X)$存在,则$$E(X)=E(E(X |  Y)).$$
\end{theorem}
\begin{proof}
这里仅证明连续场合。

设二维随机变量$(X,Y)$的联合密度函数为$p(x,y)$,记$g(y)=E(X |  Y=y)$,则$g(X)=E(X |  Y)$,由于$p(x,y)=p(x |  Y)\cdot p_{Y}(y)$,可得
\begin{eqnarray*}
E(X) &=&\int_{-\infty}^{+\infty} \int_{-\infty}^{+\infty} x p(x, y) d x d y \\
&=&\int_{-\infty}^{+\infty} \int_{-\infty}^{+\infty} x \cdot p(x  |  y) \cdot p_{Y}(y) d x d y \\
&=&\int_{-\infty}^{+\infty} \left(\int_{-\infty}^{+\infty} x p(x |  y) dx \right) \cdot p_{Y}(y) d y \\
&=&\int_{-\infty}^{+\infty} E(X  |  Y=y) \cdot p_{Y}(y) d y \\
&=&\int_{-\infty}^{+\infty} g(y) \cdot p_{Y}(y) d y \\
&=&E[g(Y)] \\
&=&E(E(X  |  Y)).
\end{eqnarray*}
\end{proof}
类似地,我们可以定义条件方差,即$Var(X|Y)$,即
\begin{eqnarray*}
    \text{Var}(X | Y) &=& E\left ( (X-E(X |  Y))^{2} |  Y \right )\\
    &=&E(X^{2} |  Y)-\left ( E(X |  Y) \right )^{2}.
\end{eqnarray*}
类似于重期望公式,我们也可以将$X$的方差拆解成两个部分,即
$$\text{Var}(X)=E(\text{Var}(X |  Y))+\text{Var}(E(X |  Y)),$$
其中,前者可以看成组内方差,后者可以看成组间方差。

\begin{example}
    一矿工被困在有三个门的矿井里。第一个门通一坑道,沿此坑道走3小时可到达安全区;第二个门通一坑道,沿此坑道走5小时又回到原处;第三个门通一坑道,沿此坑道走7小时也回到原处。假定此矿工总是等可能地在三个门中选择一个,试求他平均要用多少时间才能到达安全区。
    \vspace{5cm}
\end{example}
\begin{solution}
如果直接求$X$的分布,$X$的可能取值为$3,5+3,7+3,5+5+3,5+7+3,\cdots$这是很困难的。

这里,我们考虑另一种解法:$Y$表示在矿井中选的门,即${Y=i}$表示选了第$i$个门。可知$$P(Y=1)=P(Y=2)=P(Y=3)=\frac{1}{3}.$$
因为选了第一个门后,$3$小时到达安全区,所以$E(X |  Y=1)=3$;因为选了第二个门后,$5$小时回到原地,所以$E(X |  Y=2)=5+E(X)$;因为选了第三个门后,$7$小时回到原地,所以$E(X |  Y=3)=7+E(X)$.
综上,
\begin{eqnarray*}
E(X) &=&E(E(X  |  Y)) \\
&=&E(X  |  Y=1) \cdot \frac{1}{3}+E(X  |  Y=2) \cdot \frac{1}{3}+E(X  |  Y=3) \cdot \frac{1}{3} \\
&=&3 \cdot \frac{1}{3}+(5+E(X)) \cdot \frac{1}{3}+(7+E(X)) \cdot \frac{1}{3} \\
&=& 5 +\frac{2}{3} E(X).
\end{eqnarray*}
解得$E(X)=15$.
\end{solution}

\begin{example}
    设$X_{1},X_{2},\cdots$为一列独立同分布的随机变量,随机变量$N$只取正整数值且$N$与${X_{n}}$独立,证明
    \begin{enumerate}
        \item $$
        E\left(\sum_{i=1}^{N} X_{i}\right)=E\left(X_{1}\right) \cdot E(N)
        $$
        \item (课后自学)$$
        \text{Var}\left(\sum_{i=1}^{N} X_{i}\right)=\operatorname{Var}\left(X_{1}\right) E(N)+\left(E\left(X_{1}\right)\right)^{2} \operatorname{Var}(N)
        $$
    \end{enumerate}
\end{example}
\begin{proof}
    \begin{enumerate}
        \item 我们先考虑第一个问题,这里与之前介绍的期望的性质有一个明显的差异——这里考虑了随机变量个随机变量之和的期望,而之前考虑的是有限个随机变量之和的期望。
        \begin{eqnarray*}
           E\left(\sum_{i=1}^{N} X_{i}\right) &=&E\left(E\left(\sum_{i=1}^{N} X_{i}  |  N\right)\right) \\
&=&\sum_{n=1}^{+\infty} E\left(\sum_{i=1}^{N} X_{i}  |  N=n\right) \cdot p_{N}(n) \\
&=&\sum_{n=1}^{+\infty} E\left(\sum_{i=1}^{n} X_{i}  |  N=n\right) p_{N}(n) \\
&=&\sum_{n=1}^{+\infty} E\left(\sum_{i=1}^{n} X_{i}\right) \cdot p_{N}(n) \\
&=&\sum_{n=1}^{+\infty} n \cdot E\left(X_{1}\right) \cdot p_{N}(n) \\
&=&E\left(X_{1}\right) \cdot E(N) 
        \end{eqnarray*}
\item 随机变量个随机变量之和的方差的证明过程由学生课后自学。
\begin{eqnarray*}
    \text{Var}\left(\sum_{i=1}^{N} X_{i}\right) &=&
    E\left(\text{Var}\left(\sum_{i=1}^{N} X_{i}  |  N\right)\right)+\text{Var}\left(E\left(\sum_{i=1}^{N} X_{i}  |  N\right)\right) \\
& =& I_{1}+I_{2}
\end{eqnarray*}
其中\begin{eqnarray*}
I_{1} &=&\sum_{n=1}^{+\infty} \text{Var}\left(\sum_{i=1}^{n} X_{i}  |  N=n\right) \cdot p_{N}(n) \\
&=&\sum_{n=1}^{+\infty} \text{Var}\left(\sum_{i=1}^{n} X_{i}\right) p_{N}(n) \\
&=&\sum_{n=1}^{+\infty} \sum_{i=1}^{n} \text{Var}\left(X_{i}\right) p_{N}(n) \\
&=&\text{Var}\left(X_{1}\right) \cdot E(N)
\end{eqnarray*}
\begin{eqnarray*}
I_{2} &=&E\left(E\left(\sum_{i=1}^{N} X_{i}  |  N\right)\right)^{2}-E^{2}\left(E\left(\sum_{i=1}^{N} X_{i}  |  N\right)\right) \\
&=&\sum_{n=1}^{+\infty}\left(E\left(\sum_{i=1}^{N} X_{i}  |  N=n\right)\right)^{2} p_{N}(n)-\left(E\left(X_{1}\right) \cdot E(N)\right)^{2} \\
&=&\sum_{n=1}^{+\infty}\left(E\left(\sum_{i=1}^{n} X_{i}  |  N=n\right)\right)^{2} p_{N}(n)-\left(E\left(X_{1}\right) \cdot E(N)\right)^{2} \\
&=&\sum_{n=1}^{+\infty}\left(n \cdot E\left(X_{1}\right)\right)^{2} p_{N}(n) \\
&=&\left(E\left(X_{1}\right)\right)^{2} E\left(N^{2}\right)-\left(E\left(X_{1}\right)\right)^{2} \cdot(E(N))^{2} \\
&=&\left(E\left(X_{1}\right)\right)^{2} \text{Var}(N)
\end{eqnarray*}
    \end{enumerate}
\end{proof}

\section{习题}
    \begin{enumerate}
        \item 一射手单发命中目标的概率为$p(0 < p < 1)$,射击进行到命中目标两次为止。设$X$为第一次命中目标所需的射击次数,$Y$为总共进行的射击次数,求$(X,Y)$的联合分布和条件分布。
        \item 随机变量$X$服从$(1,2)$上的均匀分布,在$X=x$的条件下,随机变量$Y$的条件分布是参数为$x$的指数分布,证明:$XY$服从参数为1的指数分布。
        \item 设以下所涉及的数学期望均存在,试证:
        \begin{enumerate}
            \item $E[g(X)Y|X] = g(X)E(Y|X)$;
            \item $E(XY) = E[XE(Y|X)]$;
            \item $\text{Cov}[X,E(Y|X)] = \text{Cov}(X,Y).$
        \end{enumerate}


\item 设$X$是一个连续随机变量,其密度函数为
$$
p_X(x) = \begin{aligned}
    &x/4, &1< x\leq 3\\
&0, &\text{其他}.
\end{aligned}
$$
令事件$A = \{X\geq 2\}$。
\begin{enumerate}
    \item 求$E(X)$,$P(A)$,$p_{X|A}(x)$以及$E(X|A)$;
    \item 令$Y = X^2$。求$E(Y)$和$\text{Var}(Y)$。
\end{enumerate}

\item Pat和Nat将要约会,他们所有的约会都安排在晚上9点开始。Nat总是在晚上9点准时到达。Pat非常混乱,到达的时间在晚上8点到晚上10点之间均匀分布。设$X$表示从晚上8点到Pat到达的时间之间的小时数。如果Pat在晚上9点之前到达,他们的约会将持续整整3个小时。如果Pat在晚上9点之后到达,他们约会的持续时间将在0到3-X小时之间均匀分布。约会从他们见面的时候开始。当Pat迟到时,Nat会感到生气,在Pat第二次约会迟到超过45分钟后,Nat将结束这段关系。所有约会均相互独立。
\begin{enumerate}
    \item Nat等待Pat到达的期望小时数是多少?
    \item 任一约会的期望持续时间是多少?
    \item 他们分手前的期望约会次数是多少?
\end{enumerate}
    \end{enumerate}


