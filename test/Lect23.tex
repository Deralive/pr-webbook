\chapter{似然比检验与分布拟合检验}
\begin{introduction}
  \item Prob \& Stat\quad 7.4
\end{introduction}

\section{引导问题}

\begin{instance}[(回顾:女士品茶)]
在女士品茶中,我们所考虑的原假设和备择假设分别为
$$
H_0: \theta \in \Theta_0 = \{0.5\} \quad \text{vs} \quad H_1: \theta \in \Theta_1 = \left\{ \theta > 0.5\right\}.
$$
我们发现:女士在10杯的判断中都甄别正确,记为所关心的事件$A$。
在原假设成立时,$P(A) = 1/1024$;而在备择假设成立时,$P(A) = \theta^{10}$这个概率是不确定的,因为我们并不知道$\theta$的真实值。但是我们知道,这个概率是不会超过1的。于是,我们考虑以下比值
$$
\frac{\sup_{\theta\in \Theta_1} P(A)}{\sup_{\theta\in \Theta_0} P(A)} = \frac{1}{1/1024} = 1024
$$
这个比值非常大。这说明:$\theta \in \Theta_1$的可能性要比$\theta \in \Theta_0$的可能性要大得多。所以,我们有理由认为$H_0$不成立。
\end{instance}
根据这一思想,奈曼和E·皮尔逊提出了一种名为似然比检验的检验方法。

\section{似然比检验}

\begin{definition}[似然比]
    设$x_1,x_2,\cdots,x_n$为来自密度函数为$p(x;\theta),\theta\in \Theta$的总体的样本,考虑如下检验问题:
    $$
    H_0: \theta \in \Theta_0  \quad \text{vs} \quad H_1: \theta \in \Theta_1 = \Theta- \Theta_0.
    $$
    令
    $$
    \Lambda_1(x_1,x_2,\cdots,x_n) = \frac{\sup_{\theta \in \Theta_1} p(x_1,x_2,\cdots,x_n;\theta)}{\sup_{\theta \in \Theta_0} p(x_1,x_2,\cdots,x_n;\theta)}
    $$
    和
    $$
    \Lambda(x_1,x_2,\cdots,x_n) = \frac{\sup_{\theta \in \Theta} p(x_1,x_2,\cdots,x_n;\theta)}{\sup_{\theta \in \Theta_0} p(x_1,x_2,\cdots,x_n;\theta)}
    =\max\{\Lambda_1,1\}$$
    则称统计量$\Lambda_1(x_1,x_2,\cdots,x_n)$和$\Lambda(x_1,x_2,\cdots,x_n)$为该假设的似然比(likelihood ratio)。
\end{definition}

\begin{definition}[似然比检验]
如果似然比统计量$\Lambda(x_1,x_2,\cdots,x_n)$是假设检验问题
$$
    H_0: \theta \in \Theta_0  \quad \text{vs} \quad H_1: \theta \in \Theta_1 = \Theta- \Theta_0
    $$
的检验统计量,且取其拒绝域为$W = \{\Lambda(x_1,x_2,\cdots,x_n)\geq c\}$,其中临界值$c$满足
$$
P(\Lambda(x_1,x_2,\cdots,x_n)\geq c)\leq \alpha, \forall \theta \in \Theta_0,
$$
则称此检验为显著性水平$\alpha$的似然比检验(likelihood ratio test),简称LRT。
\end{definition}

\begin{problem}
    对于两个似然比,什么时候该用$\Lambda_1$,什么时候该用$\Lambda$?
\end{problem}
\begin{solution}
    一般来说,$\Lambda_1$用于分布类型的选择,$\Lambda$用于参数假设检验。
\end{solution}
\begin{example}{(分布类型选择类的假设检验问题)}
    在研究轴承的寿命时,记轴承的寿命为$T$。在美苏冷战时期,美国使用的分布时对数正态分布(logNormal),而苏联使用的是韦布尔分布(Weibull)。那么,对于中国所生产的轴承,其寿命应服从对数正态分布,还是应服从韦布尔分布?
\end{example}



\begin{example}
设$x_{1},\cdots,x_{n}$是来自正态总体$N(\theta,\sigma _{0}^{2}),\sigma _{0}^{2}$的样本。检验问题为
$$
H_{0}: \theta=\theta_{0} \quad \text{vs} \quad H_{1}: \theta \neq \theta_{0} .$$

根据原假设和备择假设,$\Theta_0 = \{\theta_0\}$,而$\Theta_1 = (-\infty,\theta_0)\cup (\theta_0,\infty)$。于是,$\Theta = (-\infty,\infty)$。
对于参数$\theta$而言,其似然函数为
$$
l(\theta) = p(x_1,x_2,\cdots,x_n;\theta) = (2\pi \sigma_0^2)^{-n/2}\exp\left\{-\frac{1}{2\sigma_0^2} \sum_{i=1}^n(x_i-\theta)^2\right\}.
$$
对于$\theta \in \Theta_0$,
$$
\sup_{\theta\in \Theta_0} l(\theta) = l(\theta_0) = (2\pi \sigma_0^2)^{-n/2}\exp\left\{-\frac{1}{2\sigma_0^2} \sum_{i=1}^n(x_i-\theta_0)^2\right\},
$$
而对于$\theta \in \Theta$,
$$
\sup_{\theta\in \Theta} l(\theta) = l(\hat{\theta}_{\text{ML}}) = (2\pi \sigma_0^2)^{-n/2}\exp\left\{-\frac{1}{2\sigma_0^2} \sum_{i=1}^n(x_i-\bar{x})^2\right\}.
$$
于是,似然比统计量为
\begin{eqnarray*}
    \Lambda &=& \frac{\sup_{\theta\in \Theta} l(\theta)}{\sup_{\theta\in \Theta_0} l(\theta)}\\
    &=& \frac{ (2\pi \sigma_0^2)^{-n/2}\exp\left\{-\frac{1}{2\sigma_0^2} \sum_{i=1}^n(x_i-\bar{x})^2\right\}}{ (2\pi \sigma_0^2)^{-n/2}\exp\left\{-\frac{1}{2\sigma_0^2} \sum_{i=1}^n(x_i-\theta_0)^2\right\}}\\
    &=& \frac{  \exp\left\{-\frac{1}{2\sigma_0^2} \sum_{i=1}^n(x_i-\bar{x})^2\right\}}{  \exp\left\{-\frac{1}{2\sigma_0^2} \sum_{i=1}^n(x_i-\theta_0)^2\right\}}\\
    &=& \frac{  \exp\left\{-\frac{1}{2\sigma_0^2} \sum_{i=1}^n(x_i-\bar{x})^2\right\}}{  \exp\left\{-\frac{1}{2\sigma_0^2} \left(\sum_{i=1}^n(x_i-\bar{x})^2 + n(\bar{x} - \theta_0)^2\right)\right\}}\\
    &=& \exp \left\{  \frac{n}{2\sigma_0^2} \left( \bar{x} - \theta_0 \right)^2\right\}
\end{eqnarray*}
而$z$检验统计量为
$$
z = \frac{\bar{x}-\theta_0}{\sqrt{\sigma_0^2/n}}.
$$
由此,似然比统计量$\Lambda$是$z$检验统计量的绝对值的严格递增函数。易知,$\{\Lambda \geq c_1\}$等价于$\{|z|\geq c_2\}$,这里两个临界值$c_1$和$c_2$是根据显著性水平$\alpha$来确定的。因此,此时的似然比检验与双侧$z$检验完全等价。
\end{example}
\begin{remark}
    虽然难以求得似然比检验的精确分布,但是在一般条件下,其存在一个渐近分布,即$-2\ln \Lambda$服从卡方分布,其自由度为其独立参数个数。
\end{remark}
\section{分布的拟合优度检验}
\subsection{概述}
\begin{example}{(孟德尔豌豆实验)}\label{ex:chap23_pea_experiment}
    在19世纪,孟德尔按颜色与形状把豌豆分为四类:黄圆、绿圆、黄皱、绿皱。孟德尔根据遗传学原理判断这四类的比例应为$9:3:3:1$。为做验证,孟德尔在一次豌豆实验中收获了$n=556$个豌豆,其中这四类豌豆的个数分别为$315,108,101,32$。该数据是否与孟德尔提出的比例吻合?

    这个问题可以转化为分类数据的检验问题。总体被分为$r$类:$A_1,A_2,\cdots,A_r$。我们提出假设:
    $$
    H_0: A_i\text{ 所占的比率是 }p_{i0},\quad i=1,2,\cdots,r,    $$
    其中,$p_{i0}$已知,且满足$\sum_{i=1}^r p_{i0}=1$。记$x_1,x_2,\cdots,x_n$表示从该总体抽取的样本,且$n_i$表示$n$个样本中属于$A_i$的样本个数。由于当$H_0$成立时,在$n$个样本中属于$A_i$类的“期望个数”$np_{i0}$。而我们实际观测到的值为$n_i$,故当$H_0$成立时,$n_i$与$np_{i0}$应相差不大。于是,卡尔·皮尔逊提出检验统计量
    $$
    \chi^2 = \sum_{i=1}^r \frac{(n_i - np_{i0})^2}{np_{i0}}
    $$
    来衡量“期望个数”与“实际个数”间的差异。
    \end{example}
    
    \begin{theorem}
    总体被分为$r$类:$A_1,A_2,\cdots,A_r$。考虑假设:
    $$
    H_0: A_i\text{ 所占的比率是 }p_{i0},\quad i=1,2,\cdots,r,    $$
    其中,$p_{i0}$已知,且满足$\sum_{i=1}^r p_{i0}=1$。记$x_1,x_2,\cdots,x_n$表示从该总体抽取的样本,且$n_i$表示$n$个样本中属于$A_i$的样本个数。
    在$H_0$成立时,检验统计量
    $$
    \chi^2 = \sum_{i=1}^r \frac{(n_i - np_{i0})^2}{np_{i0}} \overset{L}{\rightarrow} \chi^2(r-1).
    $$
    \end{theorem}
    \begin{remark}
        根据上述定理,我们所确定的拒绝域为
        $$
        W = \{\chi^2 \geq \chi^2_{1-\alpha}(r-1)\}.
        $$
        这个检验通常称为皮尔逊$\chi^2$拟合优度检验。
    \end{remark}
    \begin{example}{(例\ref{ex:chap23_pea_experiment}续)}
        根据题意,我们有
        \begin{table}[htbp]
  \centering
  \begin{tabular}{ccccccc}
  \hline
    $X$& 黄圆$(A_{1})$ 
    & 绿圆$(A_{2})$ 
    & 黄皱$(A_{3})$ 
    & 绿皱$(A_{4})$ \\
  \hline
    概率 & $p_{10}=\frac{9}{16}$ 
    & $p_{20}=\frac{3}{16}$
    & $p_{30}=\frac{3}{16}$
    & $p_{40}=\frac{1}{16}$
     \\
     \hline
    频数& $315$ 
    & $108$
    & $101$
    & $32$
     \\
  \hline
  \end{tabular}
\end{table}

于是,该统计量为
\begin{eqnarray*}
    \chi^2 &=& \frac{\left(315 - 556 \times \frac{9}{16}\right)^2}{ 556 \times \frac{9}{16}} + 
\frac{\left(108 - 556 \times \frac{3}{16}\right)^2}{ 556 \times \frac{3}{16}} + 
\frac{\left(101 - 556 \times \frac{3}{16}\right)^2}{ 556 \times \frac{3}{16}} + 
\frac{\left(32 - 556 \times \frac{1}{16}\right)^2}{ 556 \times \frac{1}{16}}\\
&=& 0.47
\end{eqnarray*}
若取显著性水平$\alpha = 0.05$,则$\chi^2_{0.95}(3) = 7.8147 > 0.47$。所以,认为孟德尔的结论是可接受的。
    \end{example}

从似然比检验的角度也可以得到皮尔逊$\chi^2$拟合优度检验统计量。
样本的联合分布为
$$
P_{\theta}(X_1 = x_1,X_2=x_2,\cdots,X_n = x_n) = p_1^{n_1}p_2^{n_2}\cdots p_r^{n_r}=\prod_{i=1}^r p_i^{n_i}.
$$
由此求得
\begin{eqnarray*}
    \sup_{\theta \in\Theta} =P_{\theta}(X_1 = x_1,X_2=x_2,\cdots,X_n = x_n) = \prod_{i=1}^r \left(\frac{n_i}{n}\right)^{n_i},\\
     \sup_{\theta \in\Theta} =P_{\theta}(X_1 = x_1,X_2=x_2,\cdots,X_n = x_n) = \prod_{i=1}^r \left(p_{i0}\right)^{n_i}
\end{eqnarray*}
于是,似然比检验统计量为
$$
\Lambda(x_1,x_2,\cdots,x_n ) = \prod_{i=1}^r \left(\frac{n_i}{np_{i0}}\right)^{n_i}.
$$
由于
\begin{eqnarray*}
    \ln \Lambda(x_1,x_2,\cdots,x_n ) &=&\sum_{i=1}^r n_i \frac{n_i}{np_{i0}}\\
    &=& \sum_{i=1}^r ( np_{i0} + (n_i-np_{i0})) \ln \left( 1 + \frac{n_i - np_{i0}}{np_{i0}}\right)\\
    &=& \sum_{i=1}^r ( np_{i0} + (n_i-np_{i0})) \left( \frac{n_i - np_{i0}}{np_{i0}} - \frac{1}{2} \left(  \frac{n_i - np_{i0}}{np_{i0}}\right)^2 + o(n^{-2})\right)\\
    &\approx& \frac{1}{2} \sum_{i=1}^r \frac{(n_i - np_{i0})^2}{np_{i0}} + o(n^{-1}). 
\end{eqnarray*}
所以
$$
2\ln \Lambda(x_1,x_2,\cdots,x_n )  \approx \sum_{i=1}^r \frac{(n_i - np_{i0})^2}{np_{i0}} .
$$
\begin{remark}
    \begin{itemize}
        \item 在孟德尔豌豆实验中,诸$p_{i0}$都是已知的。但实际中,$p_{i0}$可能依赖于$k$个未知参数,这$k$个参数可以通过最大似然估计来得到,此时检验统计量为
        $$
        \chi^2 = \sum_{i=1}^r \frac{(n_i - n\hat{p}_{i})^2}{n\hat{p}_{i}} \overset{L}{\rightarrow} \chi^2(r-k-1).
        $$
        \item $\chi^2$检验法用于大样本场景,一般要求,各个类中的观测值均不能小于5。
    \end{itemize}
\end{remark}
\subsection{离散分布}
设$x_1,x_2,\cdots,x_n$是来自总体$F(x)$的样本,所需要检验的原假设为
$$
H_0: F(x) = F_0(x)
$$
其中,$F_0(x)$称为理论分布,可以是一个完全已知的分布,也可以是一个依赖于有限个实参数且分布形式已知的分布函数。这类问题可以用$\chi^2$拟合优度来解决。

设总体$X$为取有限或可列个值$a_1,a_2,\cdots$的离散随机变量。如有需要,可以将相邻的取值进行合并,最后分为有限个类$A_1,A_2,\cdots,A_r$,并使得样本观测值$x_1,x_2,\cdots,x_n$落入每个$A_i$内的个数$n_i$不少于5。记$P(X\in A_i) = p_i (i=1,2,\cdots,r)$。

\begin{example}
    考虑卢瑟福实验的数据。表\ref{tab:chap23_discrete_distribution_chi2_test}是卢瑟福以7.5秒为时间单位做的2608次观测得到的数据,观测的是一枚放射性$\alpha$物质在单位时间点放射的质点数。
    \begin{table}[ht]
        \centering
        \caption{卢瑟福实验数据}\label{tab:chap23_discrete_distribution_chi2_test}
        \begin{tabular}{c ccccc ccccc ccccc}
            \hline
            质点数$k$ & 0 & 1 & 2 & 3 & 4 & 5 & 6 & 7 & 8 & 9 & 10 & 11 & 12 & 13 & 14  \\
            \hline
            观察数 $n_k$ & 57 & 203 & 383 & 525 & 532 & 408 & 273 & 139 & 45 & 27 & 10 & 4 & 2 & 0 & 0\\  
             \hline
        \end{tabular}
    \end{table}

    现要检验假设
    $$
    H_0: \text{数据服从泊松分布}P(\lambda)
    $$
\end{example}
\begin{solution}
    首先,需要估计泊松分布参数$\lambda$。因为其最大似然估计为样本均值,所以
    $$
    \hat{\lambda} = \bar{x} = 3.87.
    $$
    其次,计算各个类别的概率的估计值
    $$
    \hat{p}_k = \frac{\hat{\lambda}^k}{k!} e^{-\hat{\lambda}},k=0,1,2,\cdots
    $$

    为了满足每个类出现的样本观测次数不少于5,我们将$k\geq 11$作为一类。于是,检验统计量为
    $$
    \chi^2 = \sum_{i=0}^{11} \frac{(n_{i} - n \hat{p}_{i})^2}{n \hat{p}_i} = 12.8967.
    $$

    此时,卡方分布自由度为$12-1-1 =10$。取显著性水平$\alpha = 0.05$,临界值$\chi^2_{0.95}(10) = 18.3070$,拒绝域为$W = \{ \chi^2 \geq 18.3070\}$,观测结果的$\chi^2$不落在拒绝域,因此不能拒绝$H_0$。
\end{solution}
\subsection{连续分布}

设总体$X$为连续随机变量,分布函数$F_0(x)$,一般选取$r-1$个实数$a_1<a_2<\cdots<a_{r-1}$,将实数族分为$r$个区间
$$
(-\infty, a_1],(a_1,a_2],\cdots, (a_{r-1},\infty)
$$
当观测值落入第$i$个区间内,就把它看作属于第$i$类,因此,这$r$个区间就相当于$r$个类。在$H_0$为真时,记
$$p_i = P(a_{i-1} < X \leq a_i ) = F_0(a_i) - F_0(a_{i-1}) ,i=1,2,\cdots,r $$
其中,$a_0 = -\infty$, $a_{r} = \infty$, 以$n_i$表示样本的观测值$x_1,x_2,\cdots,x_n$落入区间$(a_{i-1},a_i]$内的个数。
\begin{example}
    某工厂生产一种滚珠,现随机地抽取了50件产品,测得其直径为
    \begin{eqnarray*}
15.0, \quad 15.8, \quad 15.2, \quad 15.1, \quad 15.9, \quad
14.7, \quad 14.8, \quad 15.5, \quad 15.6, \quad 15.3\\
15.0, \quad 15.6, \quad 15.7, \quad 15.8, \quad 14.5, \quad
15.1, \quad 15.3, \quad 14.9, \quad 14.9, \quad 15.2\\
15.9, \quad 15.0, \quad 15.3, \quad 15.6, \quad 15.1, \quad
14.9, \quad 14.2, \quad 14.6, \quad 15.8, \quad 15.2\\
15.2, \quad 15.0, \quad 14.9, \quad 14.8, \quad 15.1, \quad
15.5, \quad 15.5, \quad 15.1, \quad 15.1, \quad 15.0\\
15.3, \quad 14.7, \quad 14.5, \quad 15.5, \quad 15.0, \quad
14.7, \quad 14.6, \quad 14.2, \quad 14.2, \quad 14.5
    \end{eqnarray*}
问滚珠直径是否服从正态分布?
\end{example}
\begin{solution}
    设滚珠直径为$X$,其分布函数为$F(x)$,现假设为
    $$
    H_0 : F(x) = \Phi\left(\frac{x-\mu}{\sigma}\right)
    $$
    对于此问题,我们首先需要估计$\mu$和$\sigma^2$。根据极大似然估计,
    $$
    \hat{\mu} = 15.1,\quad \hat{\sigma}^2 = 0.4379^2
    $$
    根据数据特点,我们取
    $$
    a_0 = -\infty,\quad a_1 = 14.55, \quad a_2 = 14.95, \quad a_3 = 15.35, \quad a_4 = 15.75, \quad a_5 = \infty
    $$
    各组数据个数分别为
    $$
    n_1 = 6, \quad n_2 = 11, \quad n_3 = 20, \quad n_4 = 8,\quad n_5 = 5.
    $$
    利用公式计算各组理论概率,即
    $$
    \hat{p}_i = \Phi\left(\frac{a_i-15.1}{0.4379}\right) -  \Phi\left(\frac{a_{i-1}-15.1}{0.4379}\right) ,\quad i=1,2,3,4,5
    $$
    求得
    $$
    \hat{p}_1 = 0.104559, \quad \hat{p}_2 = 0.261412, \quad \hat{p}_3 = 0.349998, \quad \hat{p}_4 = 0.215174, \quad \hat{p}_5 = 0.068857
    $$
    于是,可以计算统计量
    $$
    \chi^{2} = 2.2109
    $$
    此时,分布自由度为$5-2-1=2$。取显著性水平$\alpha =0.05$,则$\chi^2_{0.95}(2) = 5.9915>2.2109$。故不能拒绝原假设。
\end{solution}
\section{习题}
\begin{enumerate}
    \item 设$x_1,x_2,\cdots,x_n$为来自$N(\mu,\sigma^2)$的样本,试求假设$H_0: \sigma^2 = \sigma^2_0\quad \text{vs}\quad H_1: \sigma^2 \neq \sigma_0^2$的似然比检验。
    \item 投掷一颗骰子60次,结果如下
    \begin{table}[ht]
        \centering
        \begin{tabular}{c cccccc}
             \hline
             点数 & 1 & 2 & 3 & 4 & 5 & 6 \\
             \hline
             次数 & 7 & 8 & 12 & 11 & 9 & 13 \\ 
             \hline
        \end{tabular}
    \end{table}

    试在显著性水平$\alpha = 0.05$下检验这颗骰子是否均匀。
    \item 在一批灯泡中抽取300只作寿命试验,结果如下:

    \begin{table}[ht]
        \centering
        \begin{tabular}{c cccc}
             \hline
             寿命(小时) & $<100$ & $[100,200)$ & $[200,300)$ & $\geq 300$ \\
             \hline
             灯泡数 & 121 & 78 & 43 & 58\\ 
             \hline
        \end{tabular}
    \end{table}

    在显著性水平为$0.05$下,能否认为灯泡寿命服从指数分布$Exp(0.005)$?

    \item 某种配偶的后代按体格的属性分为三类,各类的数目分别是10,53,46。按照某种遗传模型其频率之比应为$p^2:2p(1-p):(1-p)^2$,问数据与模型是否相符($\alpha=0.05$)?
\end{enumerate}