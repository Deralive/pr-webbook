\chapter{假设检验}
\begin{introduction}
  \item Prob \& Stat \quad 7.1, 7.2  
\end{introduction}
\section{引导案例}
\begin{instance}[(女士品茶)]

    在20世纪初期,英国人在午后会吃下午茶。在下午茶中,奶茶是一种常见的饮品。简单来说,奶茶是由茶和奶混合而成的。某位女士声称,她能够品尝得出这杯奶茶是先加奶,还是先加茶。前者记为“MT”,后者记为“TM”。为了确定这位女士的确具有这种“分辨能力”,R. A. Fisher 在这个问题中提出了一种实验方案:他请侍者制作了10杯奶茶,可能是MT,也可能是TM,请这位女士进行品尝,并让女士对每一杯奶茶进行甄别。令$X$表示该女士甄别正确的数量。如何度量出女士是否具有真实的分辨能力?

    这里,$X$可以看作一个二项分布$b(10,p)$随机变量,用$p$来度量这位女士是否具有分辨能力。考虑一下两种情况:
    \begin{itemize}
        \item 如果女士没有甄别能力,甄别正确和甄别错误时等可能的,那么$p=0.5$。
        \item 如果女士有甄别能力,当然她仍可能犯错,那么$0.5<p\leq 1$。
    \end{itemize}
    
    如果女士在10杯的判断中都甄别正确,那么这个事件发生的可能性为
    $$
    P(X= 10)= p^{10}. 
    $$
    如何来看待这个概率值?
    \begin{itemize}
        \item 解释一:这个女士是无这种能力的, 则$p=0.5$。但$P(X=10) = {0.5}^{10} = (1024)^{-1} \approx 0.001$,说明:这位女士今天超常发挥。
        \item 解释二:这个女士是有这种能力的,则$0.5<p\leq1$。这个情况下,$P(X=10) = p^{10} $,说明:这个女士今天是正常发挥。
    \end{itemize}
    对于这两种解释,我们通常认为:无法否决解释一,但更倾向于接受解释二。这是因为解释一出现的概率太小了。
    
\end{instance}
\begin{remark}
    假设检验的根本思想是:在任何假设下,小概率事件通常是不可能发生的;一旦这个小概率事件发生,则所提出的假设不应该被接受。
\end{remark}
\begin{note}
    \vspace{5cm}
\end{note}
\section{假设检验的基本概念}
这里我们先给出假设检验的一些基础概念。
\begin{definition}
设有来自某一个参数分布族$\{F(x,\theta)|\theta \in \Theta\}$的样本$x_1,x_2,\cdots,x_n$,其中$\Theta$为参数空间。设$\Theta_0 \subset \Theta$且$\Theta_0 \neq \emptyset$。则称命题
$$
H_0 : \theta \in \Theta_0
$$
为一个假设,或原假设,或零假设(null hypothesis)。若有另一个$\Theta_1 (\Theta_1 \subset \Theta, \Theta_1 \cap \Theta_0 = \emptyset)$,则称命题
$$
H_1 : \theta \in \Theta_1 
$$
为$H_0$的对立假设或备择假设(alternative hypothesis)。于是,我们感兴趣的一对假设就是
$$
H_0: \theta \in \Theta_0 \quad \text{vs} \quad H_1: \theta \in \Theta_1
$$
其中,vs 是 versus 的缩写,即表示$H_0$对$H_1$的假设检验问题。
\end{definition}
\begin{remark}
    \begin{enumerate}
        \item $\Theta_1$的常见一种情况是$\Theta_1 = \Theta-\Theta_0$;
        \item 如果$\Theta_0$仅包含一个点,则称其为简单(simple)原假设;否则称为复杂(composite)或复合原假设。备择假设也会有简单和复杂的差别。
        \item 当原假设为简单原假设,即$H_0:\theta = \theta_0$,此时,备择假设通常有三种:
        \begin{itemize}
            \item $H_1^{'}: \theta \neq \theta_0$;
            \item $H_1^{''}: \theta < \theta_0$;
            \item $H_1^{'''}: \theta > \theta_0$;
        \end{itemize}
        称$H_0 \ \text{vs} \  H_1^{'}$为双侧假设或双边假设;称 $H_0 \  \text{vs} \  H_1^{''}$ 或 $H_0 \  \text{vs} \  H_1^{'''}$ 为单侧假设或单边假设。
    \end{enumerate}
\end{remark}
\begin{example}
    在女士品茶的例子中,我们提出的假设为
    $$
    H_0: p = 0.5 \quad \text{vs} \quad H_1: p>0.5.
    $$
    原假设时简单的,备择假设是复杂的;而且这是一个单边假设检验问题。
\end{example}

在提出一个假设检验问题(给出一对假设)之后,我们需要给出一个具体的判断规则,称这个判断规则为该假设的一个检验或者检验法则。

\begin{definition}
    给定样本$x_1,x_2,\cdots,x_n$。对于其样本空间$\Omega$,我们可以确定一组划分
    $$
    W \cap \overline{W} = \emptyset, \quad W \cup \overline{W} = \Omega.
    $$
    当$(x_1,x_2,\cdots,x_n) \in W$时,拒绝$H_0$;否则接受$H_0$。于是,称$W$为该检验的拒绝域,而称$\overline{W}$为接受域。
\end{definition}
\begin{remark} 一旦拒绝域确定了,检验的判断准则也就确定了。
\end{remark}

因为我们的检验是基于样本而确定的,所以,对于任何的检验而言,我们所作出的判断都可能犯错。检验有常见的两类错误,如表\ref{tab:lect22_1}所示。
\begin{table}[ht]
    \centering
    \caption{检验的两种错误} \label{tab:lect22_1}
    \begin{tabular}{c c |cc}
    \hline
    & & \multicolumn{2}{c}{真实情况}\\
    && $H_0$为真 & $H_1$为真\\
    \hline
      \multirow{2}{*}{判断结果}& 判$H_0$为真   &  正确 &  犯第二类错误\\
      & 判$H_1$为真   &  犯第一类错误 &  正确\\
    \hline
    \end{tabular}
  \end{table}
  同时,我们需要定义这两类错误发生概率,分别为
  \begin{itemize}
      \item 第一类错误发生的概率为$\alpha(\theta) = P_{\theta}((x_1,x_2,\cdots,x_n)\in W),\theta \in \Theta_0$.
      \item 第二类错误发生的概率为$\beta(\theta) = P_{\theta}((x_1,x_2,\cdots,x_n)\in \overline{W}),\theta \in \Theta_1$.
  \end{itemize}
\begin{remark}
    可以类比“无罪假定”,第一类错误也称为“错判”,而第二类错误也称为“漏判”。
\end{remark}
\begin{problem}
    事实上,第一类错误发生概率和第二类错误发生概率无法同时降低,那么哪一种错误值得我们在意呢?
\end{problem}

为统一地表示第一类错误发生的概率和第二类错误发生的概率,我们引入势函数或功效函数(power function)。
\begin{definition}
    设检验问题
    $$
    H_0: \theta\in \Theta_0 \quad \text{vs} \quad H_1: \theta\in\Theta_1
    $$
    的拒绝域为$W$,则样本$(x_1,x_2,\cdots,x_n)$落在拒绝域$W$内的概率称为该检验的势函数,记为
    $$
    g(\theta) = P_{\theta}((x_1,x_2,\cdots,x_n) \in W), \theta \in \Theta=\Theta_0 \cup \Theta_1.
    $$
\end{definition}
\begin{remark}
    势函数
    $$
    g(\theta) = \left\{
    \begin{aligned}
    & \alpha(\theta), & \theta\in\Theta_0\\
    & 1-\beta(\theta), & \theta\in\Theta_1\\
    \end{aligned}
    \right.
    $$
\end{remark}
\begin{note}
    \vspace{6cm}
\end{note}

\begin{example}\label{ex:chap22_one_sample_mean_z}
    设$x_1,x_2,\cdots,x_n$是来自正态分布$N(\mu,\sigma_0^2)$的样本,其中$\sigma_0^2$是已知的。我们需要检验以下假设
    $$
    H_0: \mu = \mu_0 \quad \text{vs} \quad H_0: \mu > \mu_0.    $$
    在上述假设中,我们想要知道总体均值$\mu$是否大于给定的值$\mu_0$。因为样本均值$\bar{x} = \frac{1}{n}\sum_{i=1}^n x_i$是总体均值$\mu$的一个合理估计,所以我们需要依据样本均值来推断哪种假设更可能成立。以下有两种情况:
    \begin{itemize}
        \item 第一种情况:$\bar{x} \leq \mu_0$,于是,我们没有任何证据来支持备择假设。
        \item 第二种情况:$\bar{x} > \mu_0$,于是,我们能够直接下结论——支持备择假设吗?
    \end{itemize}
    在第二种情况下,我们仍无法直接接受备择假设。这是因为如果只是大一点,这可能是由于样本的随机性造成的。于是,我们需要考虑一个问题:样本均值$\bar{x}$多大,才能接受备择假设。也就是说,我们构造一个拒绝域为
    $$
    \{\bar{x} \geq c\}
    $$
    其中,$c$指的是临界值,怎么来确定这个$c$是另一个问题。

    为了确定$c$,我们需要计算第一类错误发生的概率,即
    \begin{eqnarray*}
        P\left(\bar{x} > c |H_0\right) &= &P\left(\frac{\bar{x}-\mu_0}{\sqrt{\sigma_0^2/n}} > \frac{c-\mu_0}{\sqrt{\sigma_0^2/n}} |H_0\right) \\
        &=& 1-\Phi\left(\frac{c-\mu}{\sqrt{\sigma_0^2/n}}\right)
    \end{eqnarray*}
    其中第二个等式成立是因为$\bar{x}\sim N(\mu,\sigma_0^2)$且在$H_0$成立时,$\mu=\mu_0$。

    记
    $$
    \alpha_{\mu_0}(c) = 1-\Phi\left(\frac{c-\mu_0}{\sqrt{\sigma_0^2/n}}\right).
    $$
    我们发现$\alpha_{\mu_0}(c)$是关于$c$的减函数。为控制住第一类错误率,我们事先给出一个很小的值$\alpha>0$,这个值$\alpha$被称为显著性水平,也就是说,$\alpha_{\mu_0}(c)\leq \alpha$。我们可以通过这个不等式将临界值$c$反解出来,即
    $$
    c = \mu_0 + z_{1-\alpha} \sqrt{\sigma_0^2/n}.
    $$
    因此,我们所构造的拒绝域为
    \begin{eqnarray*}
         W &=& \left\{(x_1,x_2,\cdots,x_n): \bar{x} \geq  \mu_0 + z_{1-\alpha} \sqrt{\sigma_0^2/n}\right\} \\
         &=& \left\{(x_1,x_2,\cdots,x_n): \frac{\bar{x}-\mu_0}{\sqrt{\sigma_0^2/n}} \geq   z_{1-\alpha} \right\} .
    \end{eqnarray*}
    
\end{example}
\begin{remark}
    这个检验也被称为$z$检验,其中$$
    z = \frac{\bar{x}-\mu_0}{\sqrt{\sigma_0^2/n}}
    $$
    是$z$检验的检验统计量。
\end{remark}
\newpage
根据例\ref{ex:chap22_one_sample_mean_z},我们进一步论证:第一类错误发生的概率$\alpha(\mu)$和第二类错误发生的概率$\beta(\mu)$是无法同时减小。
\begin{example}{(例题\ref{ex:chap22_one_sample_mean_z}续)}

    因为我们所构造的拒绝域为$$W = \{(x_1,x_2,\cdots,x_n): \bar{x} \geq  c\} $$
    所以接受域为$$\overline{W} = \{(x_1,x_2,\cdots,x_n): \bar{x} <  c\} .$$
    于是,对于$\mu > \mu_0$,第二类错误率为
    \begin{eqnarray*}
    \beta_{\mu}(c) =   P_{\mu}\left( \bar{x} < c|H_1\right) 
 = \Phi\left(\frac{c-\mu}{\sqrt{\sigma_0^2/n}}\right),
    \end{eqnarray*}
    其中,在$H_1$成立时,$\bar{x}\sim N(\mu,\sigma_0^2)$。
    我们发现$\beta_{\mu}(c)$是关于$c$的增函数。由此,
    \begin{itemize}
        \item $\alpha_{\mu_0}(c)$减小 $\Rightarrow c$增加 $\Rightarrow \beta_{\mu}(c)$增加。
        \item $\beta_{\mu}(c)$减小 $\Rightarrow c$减小 $\Rightarrow \alpha_{\mu_0}(c)$增加。
    \end{itemize}
   综上,第一类错误发生的概率和第二类错误发生的概率不可同时减小。
\end{example}

\begin{definition}{显著性检验}
    对检验问题$$
H_0: \theta\in \Theta_0 \quad \text{vs} \quad H_1: \theta\in\Theta_1,
    $$
    如果一个检验满足对任意的$\theta\in\Theta_0$,都有
    $$
    g(\theta) = \alpha(\theta) \leq \alpha
    $$
    则称该检验是显著性水平为$\alpha$的显著性检验,简称水平为$\alpha$的检验。
\end{definition}

\section{单个总体正态分布下的假设检验问题}
\subsection{方差已知时,$\mu$的检验}
\begin{example}
    设$x_1,x_2,\cdots,x_n$是来自正态分布$N(\mu,\sigma_0^2)$的样本,其中$\sigma_0^2$是已知的。我们需要检验以下假设
    $$
    H_0: \mu \leq \mu_0 \quad \text{vs} \quad H_0: \mu > \mu_0.    $$
    求显著性水平为$\alpha$的检验。
\end{example}
\begin{solution}
    因为备择假设依旧为$\mu > \mu_0$,所以,我们构造拒绝域为
    $$
    W = \left\{\bar{x} \geq c\right\}.
    $$
    考虑第一类错误发生的概率为
    \begin{eqnarray*}
        \alpha_{\mu}(c) &=& P_{\mu}(\bar{x} \geq c|H_0) \\
        &=& 1-\Phi\left(\frac{c-\mu}{\sqrt{\sigma_0^2/n}}\right) , \mu \leq \mu_0.
    \end{eqnarray*}
    注意到$ \alpha_{\mu}(c)$是关于$\mu$的增函数。我们要求对任意$\mu\leq \mu_0$,有
    $$
    \alpha_{\mu}(c) \leq \alpha.
    $$
    只需要$\max_{\mu \leq \mu_0}\alpha_{\mu}(c)= \alpha$,等价于$\alpha_{\mu_0}(c)\leq \alpha$。于是可以解得
    $$
    c = \mu_0 + z_{1-\alpha}\sqrt{\sigma_0^2/n}.
    $$
    由此,所构造的拒绝域仍为
    $$
     W = \{(x_1,x_2,\cdots,x_n): \bar{x} \geq  \mu_0 + z_{1-\alpha} \sqrt{\sigma_0^2/n}\}.
    $$
\end{solution}
\begin{remark}
    假设$
    H_0: \mu = \mu_0 \quad \text{vs} \quad H_0: \mu > \mu_0$和$
    H_0: \mu \leq \mu_0 \quad \text{vs} \quad H_0: \mu > \mu_0$
    我们构造的水平为$\alpha$的显著性检验是相同的,拒绝域均为$W = \left\{ \bar{x} \geq  \mu_0 + z_{1-\alpha} \sqrt{\sigma_0^2/n} \right\}$。
\end{remark}
\subsection{方差未知时,$\mu$的检验}
\begin{example}
    设$x_1,x_2,\cdots,x_n$是来自正态分布$N(\mu,\sigma)$的样本,其中$\sigma$是未知的。我们需要检验以下假设
    $$
    H_0: \mu \leq \mu_0 \quad \text{vs} \quad H_0: \mu > \mu_0.    $$
    求显著性水平为$\alpha$的检验。
\end{example}
\begin{solution}
    我们构造的拒绝域为
    $
    W = \left\{\bar{x} \geq c\right\}.
    $
    考虑第一类错误发生的概率为
    \begin{eqnarray*}
        \alpha_{\mu_0}(c) &=& P_{\mu}(\bar{x} \geq c|H_0) \\
        &=& P_{\mu}\left(\frac{\bar{x} - \mu}{\sqrt{\hat{\sigma}^2/n}} \geq \frac{c - \mu}{\sqrt{\hat{\sigma}^2/n}}|H_0\right)\\
        &=& 1- P\left(\frac{\bar{x} - \mu_0}{\sqrt{\hat{\sigma}^2/n}}\leq \frac{c - \mu_0}{\sqrt{\hat{\sigma}^2/n}}\right)
    \end{eqnarray*}
    其中, $\hat{\sigma}^2 = (n-1)^{-1}\sum_{i=1}^n (x_i - \bar{x})^2$且检验统计量为
    $$
    \frac{\bar{x} - \mu_0}{\sqrt{\hat{\sigma}^2/n}} \sim t(n-1).
    $$
    为了$ \alpha_{\mu_0}(c) = \alpha$,可解得
    $$
    c = \mu_0 + t_{1-\alpha}(n-1)\sqrt{\hat{\sigma}^2/n}.
    $$
    由此,所构造的拒绝域仍为
    $$
     W = \{(x_1,x_2,\cdots,x_n): \bar{x} \geq   \mu_0 + t_{1-\alpha}(n-1)\sqrt{\hat{\sigma}^2/n}\}.
    $$
\end{solution}
\begin{remark}
    这个检验是单样本的$t$检验,这是因为检验统计量服从$t$分布。
\end{remark}
\newpage
上述介绍的检验都是单边检验,以下我们给一个双边检验的例子。
\begin{example} \label{ex:chap22_one_sample_mean_t}
    设$x_1,x_2,\cdots,x_n$是来自正态分布$N(\mu,\sigma)$的样本,其中$\sigma$是未知的。我们需要检验以下假设
    $$
    H_0: \mu = \mu_0 \quad \text{vs} \quad H_0: \mu \neq \mu_0.    $$
    求显著性水平为$\alpha$的检验。
\end{example}
\begin{solution}
    我们构造的拒绝域为
    $$
    W = \left\{\bar{x} \leq c_1\right\} \cup \left\{\bar{x} \geq c_2\right\}.
    $$
    考虑第一类错误发生的概率为
    \begin{eqnarray*}
        \alpha_{\mu_0}(c) &=& P_{\mu_0}\left(\left\{\bar{x} \leq c_1\right\} \cup \left\{\bar{x} \geq c_2\right\}\right) \\
        &=& P_{\mu_0}\left(\left\{\bar{x} \leq c_1\right\} |H_0\right) + P\left( \left\{\bar{x} \geq c_2\right\}\right) \\
        &=& P_{\mu_0}\left(\frac{\bar{x} - \mu_0}{\sqrt{\hat{\sigma}^2/n}} \leq \frac{c_1 - \mu_0}{\sqrt{\hat{\sigma}^2/n}}\right) + P_{\mu_0}\left(\frac{\bar{x} - \mu_0}{\sqrt{\hat{\sigma}^2/n}} \geq \frac{c_2 - \mu_0}{\sqrt{\hat{\sigma}^2/n}}\right)\\
        &=& P\left( t \leq \frac{c_1 - \mu_0}{\sqrt{\hat{\sigma}^2/n}} \right) + 1 - P\left( t \leq \frac{c_2 - \mu_0}{\sqrt{\hat{\sigma}^2/n}}\right).
    \end{eqnarray*}
    为了$ \alpha_{\mu_0}(c) = \alpha$,我们将$\alpha$拆成两部分,即
    $$
    \left\{
    \begin{aligned}
       & P\left( t \leq \frac{c_1 - \mu_0}{\sqrt{\hat{\sigma}^2/n}} \right) = \alpha/2\\
        & 1 - P\left( t \leq \frac{c_2 - \mu_0}{\sqrt{\hat{\sigma}^2/n}}\right) = \alpha/2
    \end{aligned}
    \right.
    $$
    于是解得
    \begin{equation*}
        \begin{aligned}
       & c_1 = \mu_0 - t_{1-\alpha/2}(n-1)\sqrt{\hat{\sigma}^2/n} \\
       & c_2 = \mu_0 + t_{1-\alpha/2}(n-1)\sqrt{\hat{\sigma}^2/n} .
    \end{aligned}
    \end{equation*}
    由此,所构造的拒绝域仍为
    $$
     W = \{(x_1,x_2,\cdots,x_n): \bar{x} \leq \mu_0 - t_{1-\alpha/2}(n-1)\sqrt{\hat{\sigma}^2/n} \text{ 或 }  \bar{x} \geq \mu_0 + t_{1-\alpha/2}(n-1)\sqrt{\hat{\sigma}^2/n}\}.
    $$
\end{solution}
\subsection{假设检验与区间估计之间的关系}
在例题\ref{ex:chap22_one_sample_mean_t}中,我们构造的拒绝域为
 $$
     W = \{\bar{x} \leq \mu_0 - t_{1-\alpha/2}(n-1)\sqrt{\hat{\sigma}^2/n} \text{ 或 }  \bar{x} \geq \mu_0 + t_{1-\alpha/2}(n-1)\sqrt{\hat{\sigma}^2/n}\}.
    $$
则接受域为
$$
\overline{W} = \left\{  \mu_0 - t_{1-\alpha/2}(n-1)\sqrt{\hat{\sigma}^2/n} \leq \bar{x} \leq \mu_0 + t_{1-\alpha/2}(n-1)\sqrt{\hat{\sigma}^2/n} \right\}.
$$
根据上述区间,并进行变换,可以解得
$$
\left\{
\begin{aligned}
    & \mu_0 \leq \bar{x}  + t_{1-\alpha/2}(n-1)\sqrt{\hat{\sigma}^2/n} \\
    & \mu_0 \geq \bar{x}  - t_{1-\alpha/2}(n-1)\sqrt{\hat{\sigma}^2/n} 
\end{aligned}
\right.
$$
从$\mu_0$的角度来看,$\mu_0$的区间为
$$
[\bar{x}  - t_{1-\alpha/2}(n-1)\sqrt{\hat{\sigma}^2/n},\bar{x}  + t_{1-\alpha/2}(n-1)\sqrt{\hat{\sigma}^2/n}].
$$
\begin{example}
    设$x_1,x_2,\cdots,x_n$来自于一个正态分布$N(\mu,\sigma^2)$,其中$\sigma^2$未知。求$\mu$的置信水平为$1-\alpha$的置信区间。
\end{example}
\begin{solution}
因为$\mu$的点估计为$\hat{\mu} = \bar{x}$,其分布为$N(\mu,\sigma^2/n)$,而$\hat{\sigma}^2 = (n-1)^{-1}\sum_{i=1}^n (x_i-\bar{x})^2$。所构造的枢轴量为
$$
\frac{\bar{x} - \mu}{\sqrt{\hat{\sigma}^2/n}} \sim t(n-1).
$$
所以,$\mu$的置信水平为$1-\alpha$的置信区间为
$$
[\bar{x}  - t_{1-\alpha/2}(n-1)\sqrt{\hat{\sigma}^2/n},\bar{x}  + t_{1-\alpha/2}(n-1)\sqrt{\hat{\sigma}^2/n}]
$$
\end{solution}
\begin{remark}
    区间估计也可以解决假设检验问题。以显著性水平为$\alpha$的双边检验问题为例,如果所构造的置信水平为$1-\alpha$置信区间能够盖住$\theta_0$,那么我们可以通过变换确定样本落在接受域$\overline{W}$中,从而接受原假设。
\end{remark}

\section{两个总体正态分布下的假设检验问题}
\subsection{方差已知时,均值差的检验}
\begin{example}
    设$x_1,x_2,\cdots,x_m$来自正态总体$N(\mu,\sigma_1^2)$的样本,$y_1,y_2,\cdots,y_n$来自正态总体$N(\mu,\sigma_2^2)$的样本。两样本相互独立。我们想要检验
    $$
    H_0: \mu_1 \leq \mu_2 + l \quad \text{vs}\quad H_1: \mu_1 > \mu_2 + l.
    $$
    如果$\sigma_1^2,\sigma_2^2$已知,那么求水平为$\alpha$的显著性检验。
\end{example}
\begin{solution}
    令$\theta = \mu_1- \mu_2 -l$。原本的假设等价于
    $$
    H_0: \theta  \leq 0 \quad \text{vs}\quad H_1: \theta > 0.
    $$
    通常我们对$\theta$的点估计为
    $$
    \hat{\theta} = \bar{x} -\bar{y} -l.
    $$
    在原假设$H_0$下,取$\theta = 0$,有
    $$
   \bar{x} -\bar{y}-l \sim N\left(0,\frac{\sigma_1^2}{m} + \frac{\sigma_2^2}{n} \right) .
    $$
    所以,经标准化后,可得检验统计量为
    $$\frac{\bar{x} -\bar{y}-l}{\sqrt{\frac{\sigma_1^2}{m} + \frac{\sigma_2^2}{n}}}.$$
    因此,水平为$\alpha$的显著性检验的拒绝域为
    $$
    W = \left\{
    \frac{\bar{x} -\bar{y}-l}{\sqrt{\frac{\sigma_1^2}{m} + \frac{\sigma_2^2}{n}}} \geq z_{1-\alpha}
    \right\}.
    $$
\end{solution}
\subsection{方差未知时,均值差的检验}
\begin{example}
    设$x_1,x_2,\cdots,x_m$来自正态总体$N(\mu,\sigma_1^2)$的样本,$y_1,y_2,\cdots,y_n$来自正态总体$N(\mu,\sigma_2^2)$的样本。两样本相互独立。
    我们想要检验
    $$
    H_0: \mu_1 \leq \mu_2 + l \quad \text{vs}\quad H_1: \mu_1 > \mu_2 + l.
    $$
    如果$\sigma_1^2 = \sigma_2^2 =\sigma^2$未知,那么求水平为$\alpha$的显著性检验。
\end{example}
\begin{solution}
    令$\theta = \mu_1- \mu_2 -l$。原本的假设等价于
    $$
    H_0: \theta  \leq 0 \quad \text{vs}\quad H_1: \theta > 0.
    $$
    通常我们对$\theta$的点估计为
    $$
    \hat{\theta} = \bar{x} -\bar{y} -l.
    $$
    在原假设$H_0$下,取$\theta = 0$,有
    $$
   \bar{x} -\bar{y}-l \sim N\left(0,\sigma^2\left(\frac{1}{m} + \frac{1}{n} \right)\right) .
    $$
    因为$\sigma^2$是冗余参数,这里我们用和方差进行估计,即
    $$
    s_w^2 = \frac{(m-1)\sum_{i=1}^m (x_i -\bar{x})^2 +(n-1)\sum_{i=1}^n (y_i -\bar{b})^2 }{m+n-2}.
    $$
    所以,经标准化后,可得检验统计量为
    $$\frac{\bar{x} -\bar{y}-l}{s_w\sqrt{\frac{1}{m} + \frac{1}{n}}} \sim t(m+n-2).$$
    因此,水平为$\alpha$的显著性检验的拒绝域为
    $$
    W = \left\{
    \frac{\bar{x} -\bar{y}-l}{s_w\sqrt{\frac{1}{m} + \frac{1}{n}}} \geq t_{1-\alpha}(m+n-2)
    \right\}.
    $$
\end{solution}
\begin{remark}
    这个检验是两样本独立$t$检验。
\end{remark}
\subsection{成对数据的检验}
\begin{example}
  为了比较两种谷物种子的优劣,特选取10块土质不全相同的工地,并将每块分为面积相同的两部分,分别种植这两种种子,施肥与田间管理在20块小块土地上都是一样,表\ref{tab:lect22_2}是各小块上的单位产量。
  \begin{table}[ht]
  \centering
  \caption{成对数的数据}\label{tab:lect22_2}
\begin{tabular}{l ccccc  ccccc} 
\hline
{土地} & 1  & 2  & 3  & 4  & 5  & 6  & 7  & 8  & 9  & 10\\ \hline
{种子1 $(x)$} & 23 & 35 & 29 & 42 & 39 & 29 & 37 & 34 & 35 & 28  \\ \hline
{种子2 $(y)$} & 30 & 39 & 35 & 40 & 38 & 34 & 36 & 33 & 41 & 31 \\ \hline
{差 $d=x-y$}   & -7 & -4 & -6 & 2  & 1  & -5 & 1  & 1  & -6 & -3 \\ \hline
\end{tabular}
\end{table}
假定单位产量服从等方差的正态分布,试问:两种种子的平均单位产量在显著性水平$\alpha = 0.05$上有无差异差异?
\end{example}
\begin{solution}
假定$x_1,x_2,\cdots,x_n$来自正态分布$N\left(\mu_{1}, \sigma^{2}\right)$和$y_1,y_2,\cdots,y_n$来自正态分布$N\left(\mu_{2}, \sigma^{2}\right)$且两样本独立。我们需要检验的问题为
$$
H_0: \mu_1 = \mu_2 \quad \text{vs} \quad H_1: \mu_1 \neq \mu_2.
$$
首先,我们考虑两样本独立$t$检验。由于检验统计量为
$$
t_1 = \frac{\bar{x}-\bar{y}}{s_w \sqrt{1/n+1/n}} = -1.1937
$$
其中,$\bar{x}=33.1$,$\bar{y}=35.7$,$s_1^2 = 33.2111$,$s_y^2 = 14.2333$,$s_w^2 = 23.7222$。且$t$分布的$1-\alpha/2$分位数为
$$
t_{0.975}(18) = 2.1009 > |t_1|
$$
所以,无法拒绝原假设,即认为两种种子的单位产量平均值没有显著差别。

但是重新看一看数据,我们发现在同一地块上不同种子的产量的差,大于零的数的绝对值比较小,但小于零的数的绝对值比较大,这让我们认为这两种种子的产量是有差异的,但与两样本独立$t$检验的结论不一致。这是因为土地之间的差异性比较大,而在比较种子时,选取了同一个地块上分别验证了两种种子的产量。基于此,我们计算其差,得到$d_i,i=1,2,\cdots,n$。由于两种种子的产量均服从正态分布,其差仍服从正态分布$N(\mu_d,\sigma_d^2)$,$\sigma_d^2$未知。我们需要检验的问题也转换为
$$
H_0: \mu_d= 0 \quad \text{vs} \quad H_1: \mu_d \neq 0 
$$
因为方差未知,我们采用单样本$t$检验。检验统计量为
$$
t_2 = \frac{\bar{d}}{\sqrt{s_d^2/n}} = -2.3475,
$$
而$t$分布的$1-\alpha/2$分位数为
$$
t_{0.975}(9) = 2.2622 < |t_2|.
$$
因此,在显著性水平为$0.05$时,我们发现种子$y$的产量与种子$x$的产量是不同的,且种子$y$的产量更高。
\end{solution}
\section{$p$值}
在假设检验中,实际问题中需要考虑如何选取显著性水平$\alpha$更为合适?这里我们利用一个简单的假设检验来比较在不同显著性水平的取值下的检验。

\begin{example}{(例题\ref{ex:chap22_one_sample_mean_z}续)}

    正态分布$N(\mu,\sigma_0^2)$下,$\sigma_0^2$已知,我们需要检验$$
    H_0: \mu = \mu_0 \quad \text{vs} \quad H_1: \mu > \mu_0
    $$
    基于样本$x_1,x_2,\cdots,x_n$,我们所构造的拒绝域为
    $$
    \{ \bar{x} \geq \mu_0 + z_{1-\alpha} \sqrt{\sigma_0^2/n} \} = \left\{ \frac{\bar{x} - \mu_0}{\sqrt{\sigma_0^2/n}} \geq z_{1-\alpha}\right\}.
    $$
    如果根据真实的数据,我们可以计算检验统计量$u = 2.25 $,那么在不同的显著性水平$\alpha$下,我们得到的结论是不同的,如表\ref{tab:lect22_3}。
    \begin{table}[ht]
        \centering
        \caption{不同显著性水平下拒绝域的对比}\label{tab:lect22_3}
        \begin{tabular}{ccc}
        \hline
            $\alpha$ & $z_{1-\alpha}$ & 结论\\
             \hline
            $10^{-4}$ & $3.72$ & 接受\\
             $10^{-3}$ & $3.09$ & 接受\\
              $0.01$ & $2.33$ & 接受\\
               $0.025$ & $1.96$ & 拒绝\\
                $0.05$ & $1.64$ & 拒绝\\
                 $0.1$ & $1.28$ & 拒绝\\
             \hline
        \end{tabular}
    \end{table}
\end{example}
我们发现存在一个$\tilde{\alpha}$值,恰好使得$z_{1-\tilde{\alpha}} = 2.25$,自然$\tilde{\alpha} = \Phi(-2.25)$,这就是我们定义的$p$值。

\begin{definition}
    在一个假设检验问题中,利用样本观测值能够作出拒绝原假设的最小显著性水平称为检验的$p$值。
\end{definition}
\begin{remark}
    有了$p$值之后,我们不需要预先设置显著性水平$\alpha$的具体取值,直接反馈$p$值即可。
\end{remark}
根据$p$值,也可以做出假设检验的结论:
\begin{itemize}
    \item 如果$p \leq \alpha$,则在显著性水平$\alpha$下拒绝$H_0$;
    \item 如果$p>\alpha$,则在显著性水平$\alpha$下接受$H_0$。
\end{itemize}


\newpage
\section{习题}
\begin{enumerate}
    \item 设$x_1,x_2,\cdots,x_n$是来自$N(\mu,1)$的样本,考虑如下假设检验问题
$$
H_0:\mu = 2 \quad \text{vs} \quad H_1: \mu = 3
$$
若检验由拒绝域为$W = \{ \bar{x} \geq 2.6 \}$确定。

\begin{enumerate}
    \item 当$n=20$时求检验犯两类错误的概率;
\item 如果要使得检验犯第二类错误的概率$\beta \leq 0.01$,$n$最小应取多少?
\item 证明:当$n\rightarrow \infty$时,$\alpha \rightarrow 0$,$\beta \rightarrow 0$。
\end{enumerate}

\item  设$x_1,x_2,\cdots,x_{20}$是来自$0-1$总体$b(1,p)$的样本,考虑如下检验问题:
$$
H_0 : p = 0.2 \quad \text{vs} \quad H_1 : p \neq 0.2
$$
取拒绝域为$W = \{ \sum_{i=1}^{20} x_i \geq 7 \text{或} \sum_{i=1}^{20} x_i \leq 1 \}$,
\begin{enumerate}
    \item 求$p = 0,0.1,0.2,\cdots,0.9,1$时的势并由此画出势函数的图;
\item 求在$p = 0.05$时犯第二类错误的概率。
\end{enumerate}

\vspace{1cm}
以下第3至第7题是应用题。答题要求:
\begin{description}
    \item[(1)] 在应用题中,需要分别采用临界值法和$p$值法;
    \item[(2)] 无特殊指定,显著性水平为$\alpha = 0.05$。
\end{description}
\vspace{0.5cm}

\item 化肥厂用自动包装机包装化肥,每包的质量服从正态分布,其平均质量为100kg,标准差为1.2kg。某日开工后,为了确定这天包装机工作是否正常,随机抽取$9$袋化肥,称得质量(单位:kg)如下:
$$
99.3 \quad 98.7 \quad 100.5 \quad 101.2 \quad 
98.3 \quad 99.7 \quad 99.5 \quad 102.1 \quad 100.5
$$

设方差稳定不变,问这一天包装机的工作是否正常(取$\alpha = 0.05$)?

\item 考察一鱼塘中的与的含汞量,随机地选取10条鱼,测得各条鱼的含汞量(单位:mg)为
$$
0.8 \quad 1.6 \quad 0.9 \quad 0.8 \quad 1.2 \quad 0.4 \quad 0.7 \quad 1.0 \quad 1.2 \quad 1.1
$$
设鱼的含汞量服从正态分布$N(\mu,\sigma^2)$。试检验假设$H_0: \mu \leq 1.2\quad \text{vs}\quad H_1: \mu > 1.2$(取$\alpha=0.1$)。

\item 一药厂生产一种新的止痛片,厂方希望验证服用新药片后至开始起作用的时间间隔较原有止痛片至少缩短一半,因此厂方提出需检验假设
$$
H_0: \mu_1 = 2\mu_2 \quad \text{vs} \quad H_1: \mu_1 > 2\mu_2
$$
此处,$\mu_1,\mu_2$分别是服用原有止痛片和服用新止痛片后至开始起作用的时间间隔的总体的均值。设两总体均为正态分布且方差分别为已知值$\sigma_1^2,\sigma_2^2$,现分别在两总体中取一样本$x_1,\cdots,x_n$和$y_1,\cdots,y_m$,设两个样本独立。试给出上述假设检验问题的检验统计量,拒绝域,并给出$p$值的计算公式。

\item  对冷却到$-0.72^{\circ}$C的样品用A,B两种测量方法测量其融化到$0^{\circ}$C时的潜热,数据如下:
\begin{eqnarray*}
    \text{方法} A&:& 79.98, 80.04, 80.02, 80.04, 80.03, 80.03,  80.04, 79.97, 80.05, 80.03, 80.02, 80.00, 80.02\\
    \text{方法} B&:& 80.02, 79.94, 79.98, 79.97, 80.03,79.95, 79.97, 79.97
\end{eqnarray*}

假设它们服从正态分布,方差相等。在显著性水平$\alpha = 0.05$时,检验两种测量方法的平均性能是否相等?

\item 为了比较测定活水中氯气含量的两种方法,特在各种场合收集到8个污水水样,每个水样均用这两种方法测定氯气含量(单位:mg/l),具体数据如下:

\begin{table}[ht]
\centering
\begin{tabular}{cccc}
\hline
水样号 & 方法一($x$) & 方法二($y$)& 差($d = x-y$) \\
\hline
 1 & 0.36 & 0.39 & -0.03\\
 2 & 1.35 & 0.84 &  0.51 \\
 3 & 2.56 & 1.76 & 0.80 \\ 
 4 & 3.92 & 3.35 & 0.57 \\ 
 5 & 5.35 & 4.69 & 0.66  \\
 6 & 8.33 & 7.70 & 0.63  \\
 7 & 10.70 & 10.52 & 0.18 \\
 8 & 10.91 & 10.92 & -0.01 \\
\hline
\end{tabular}
\end{table}

设总体为正态分布。试比较两种测定方法是否有显著差异。
\end{enumerate}


