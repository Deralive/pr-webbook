\chapter{一元随机变量函数的分布}
\begin{introduction}
  \item Intro to Prob\quad2.3
  \item Prob $ \& $ Stat\quad 2.6
\end{introduction}

\section{引导问题}
我们已经介绍过几种常见的随机变量分布。在本讲中我们将介绍在已经一个随机变量的概率分布时,求这个随机变量函数的分布。这里我们先给出这个问题的定义。
\begin{problem}
如果$X$是一个已知分布的随机变量,且我们知道某种函数(或者某种映射关系)的具体形式$g: \mathcal{R} \rightarrow \mathcal{R} $,所以,只要$g(x)$不是一个常值函数,$g(X)$通常仍是一个随机变量。因此,我们想要知道$Y = g(X)$的概率分布是什么?如何求?
\end{problem}


\section{离散随机变量函数的分布}
已知$X$为一个离散型随机变量,其分布列为
\begin{table}[htbp]
 % \caption{$X$的分布列\label{tab:color thm}}
  \centering
  \begin{tabular}{c|cccccc}

    $X$
    & $x_{1}$
    & $x_{2}$
    & $\cdots$
    & $x_{n}$
    & $\cdots$\\
  \midrule
    $P$ & $p(x_{1})$
    & $p(x_{2})$
    & $\cdots$
    & $p(x_{n})$
    & $\cdots$\\

  \end{tabular}
\end{table}
欲求$Y=g(X)$的分布?
\begin{enumerate}
    \item $Y = g(X)$也是一个离散随机变量;
    \item $Y$的分布列为
    \begin{table}[htbp]
 % \caption{$Y$的分布列\label{tab:color thm}}
  \centering
  \begin{tabular}{c|cccccc}
    $Y$
    & $g(x_{1})$
    & $g(x_{2})$
    & $\cdots$
    & $g(x_{n})$
    & $\cdots$\\
  \midrule
    $P$ & $p(x_{1})$
    & $p(x_{2})$
    & $\cdots$
    & $p(x_{n})$
    & $\cdots$\\
  \end{tabular}
\end{table}
\item 如果$g(x_{1}),g(x_{2}),\cdots,g(x_{n}),\cdots$中有某些值相等时,则把那些相等的值分别合并,并把对应的概率相加即可。
\end{enumerate}

\begin{example}
倘若$X$的分布列为
\begin{table}[ht]
    \centering
    \begin{tabular}{c|ccccc}
         $X$& $-2$ & $-1$ & $0$ & $1$ & $2$ \\
         \hline
         $P$& $0.2$ &  $0.1$ &  $0.1$ &  $0.3$ & $0.3$\\
    \end{tabular}
\end{table}

于是,$Y=X^{2}+X$的分布列为

    \centering
    \begin{tabular}{c|ccc}
         $Y$&  $0$ & $2$ & $6$ \\
         \hline
         $P$&   $0.2$ &  $0.5$ & $0.3$\\
    \end{tabular}
\end{example}
\begin{note}
\vspace{3cm}
\end{note}

\section{连续随机变量函数的分布}
\subsection{$g(X)$是一个离散型随机变量}
倘若$Y=g(X)$是一个离散型随机变量,我们只需要将$Y$的所有可能取值一一列出,在球$Y$的取各个可能值的概率。
\begin{example}
    若$X$服从正态分布$N(\mu,\sigma^2)$,则
    $$
    Y=\left\{\begin{matrix}
    0&,X<\mu \\
    1&,X\ge \mu
    \end{matrix}\right.
    $$
    那么,$Y$服从二点分布$b(1,0.5)$。
\end{example}
\subsection{$g(\cdot)$是严格单调函数}
当$g(x)$是一个关于$x$的严格单调函数,我们有以下定理。
\begin{theorem}\label{thm:rv_transform}
    设$X$是连续随机变量,其密度函数为$p_{X}(x)$。$Y=g(X)$是另一个连续随机变量。若$y=g(x)$严格单调,其反函数$h(y)$有连续导函数,则$Y=g(X)$的密度函数为
    $$p_{Y}(y)=\left\{\begin{aligned}
    p_{X}\left [ h(y) \right ]\left | h'(y) \right |  &,a<y<b \\
    0&,\text{其他}\end{aligned}\right.$$
    其中,$a=\min\left \{ g(-\infty ),g(+\infty ) \right \} ,b=\max\left \{ g(-\infty ),g(+\infty ) \right \}$
    \end{theorem}
    \begin{proof}
    不妨设$g(x)$是一个严格单调增函数,这时它的反函数$h(y)$也是严格单调增函数,且$h'(y)>0$。

    首先,考虑$Y$的取值范围。记$a = \min\{g(-\infty),g(\infty)\}$,$b = \max\{g(-\infty),g(\infty)\}$。于是,$Y=g(X)$仅在区间$(a,b)$取值。
    其次,考虑$Y$的分布函数和密度函数。
    \begin{itemize}
        \item 若$y\leq a$,那么$P(Y\leq y) = 0$,则$p_Y(y) = 0$;
        \item  若$y\geq b$,那么$P(Y\leq y) = 1$,则$p_Y(y) = 0$;
        \item 若$a< y<b$,那么$Y$的分布函数为
        \begin{eqnarray*}
        F_Y(y) &=& P(Y \leq y) = P(g(X) \leq y ) \\
        &=& P(h(g(X)) \leq h(y)) \\
        &=& P(X \leq h(y)) \\
        &=& \int_{a}^{h(y)} p_X(x) \text{d}x 
        \end{eqnarray*}
        $Y$的密度函数为
        \begin{eqnarray*}
            p_Y(y) &=& \frac{\text{d} }{\text{d} y} F(y)= \frac{\text{d} }{\text{d} y} P(Y\leq y) \\
            &=& \frac{\text{d} }{\text{d} y}\int_{a}^{h(y)} p_X(x) \text{d}x \\
            &=& p_X(h(y)) \cdot h'(y).
        \end{eqnarray*}
    \end{itemize}
    类似地,因为$g(x)$是严格单调减函数,其反函数$h(y)$也是严格单调减函数,所以$h'(y)<0$。因此,结论中$h'(y)$需要绝对值符号。
    \end{proof}
    \begin{note}
        \vspace{5cm}
    \end{note}
\begin{example}
    设随机变量$X$服从正态分布$N(\mu ,\sigma^{2})$,则当$a \neq 0$时,有$Y=aX+b\sim N(a\mu+b,a^{2}\sigma^{2})$.
\end{example}
\begin{solution} 以下我们从$a>0$和$a<0$两个方面来证明。
    \begin{enumerate}
        \item 若$a>0$,$y=g(x)=ax+b$是严格增函数。仍在$(-\infty,+\infty)$上取值,其反函数$x=h(y)=\frac{y-b}{a}$。由上述定理可知,
        \begin{eqnarray*}
            p_{Y}(y)&=&p_{X}\left ( h(y) \right )\left | h'(y) \right |  \\
            &=&\frac{1}{\sqrt{2\pi \sigma ^{2}} } \exp\left\{-\frac{1}{2\sigma ^{2}}(\frac{y-b}{a} -\mu)^{2} \right\}\cdot \frac{1}{a} \\
            &=&  \frac{1}{\sqrt{2\pi (a\sigma )^{2}} } \exp\left\{-\frac{1}{2(a\sigma) ^{2}}(y-a\mu-b)^{2} \right\}.
        \end{eqnarray*}
    因此,$Y\sim N((a\mu+b,a^{2}\sigma^{2})$。
    \item 若$a<0$,证明结果类似,学生可以在课后进行补充。
    \begin{note}
        \vspace{5cm}
    \end{note}
    \end{enumerate}
\end{solution}

\begin{example}
    设随机变量$X\sim N(\mu,\sigma^{2})$,则$Y=e^{X}$的概率密度函数为
    $$p_{Y}(y)=\left\{\begin{matrix}
    \frac{1}{\sqrt{2 \pi} y \sigma} e ^{ -\frac{(\ln y-\mu)^{2}}{2 \sigma^{2}} } &, \quad y>0 \\
    0&, \quad y \leqslant 0
    \end{matrix}\right.$$
\end{example}
\begin{solution}因为$y=g(x)=e^{x}$是严格单调递增函数,它仅在$(0,+\infty)$上取值,其反函数$x=h(y)=lny$,而$h'(y)=\frac{1}{y}$,根据定理可知,
\begin{eqnarray*}
    p_{Y}(y)&=&p_{X}\left ( h(y) \right )\left | h'(y) \right |  \\
    &=&\frac{1}{\sqrt{2\pi \sigma ^{2}} } \exp\left\{-\frac{1}{2\sigma ^{2}}(lny -\mu)^{2} \right\}\cdot \frac{1}{y} \\
    &=&\frac{1}{\sqrt{2\pi }y\sigma  } \exp\left\{-\frac{1}{2\sigma ^{2}}(lny -\mu)^{2}  \right\}
\end{eqnarray*}
\end{solution}
\begin{remark}
 这个分布称为对数正态分布$\omega \left(\mu, \sigma^{2}\right)$。
    \end{remark}

\begin{example}
    设$X\sim Ga(\alpha,r)$,则当$k>0$时,有$Y=kX\sim Ga(\alpha,\frac{r}{k})$。
\end{example}
\begin{solution}
    因为$k>0$,所以$y=kx$是严格增函数,它仍在$(0,+\infty)$上取值,其反函数$x=\frac{y}{k}$。
    \begin{enumerate}
        \item 当$y<0$时,$p_{Y}(y)=0$;
        \item 当$y\ge0$时,我们根据上述定理有
    \begin{eqnarray*}
    p_{Y}(y)&=&p_{X}\left(\frac{y}{k}\right) \cdot \frac{1}{k} \\
    &=&\frac{\lambda^{\alpha}}{\Gamma(\alpha)}\left(\frac{y}{k}\right)^{\alpha-1} e^{-\lambda \frac{y}{k}} \cdot \frac{1}{k} \\
    &=&\frac{\left(\frac{\lambda}{k}\right)^{\alpha}}{\Gamma(\alpha)} y^{\alpha-1} e^{-\frac{\lambda}{k} y}
    \end{eqnarray*}
    即$Y\sim Ga(\alpha,\frac{\lambda}{k})$。
    \end{enumerate}
\end{solution}
\begin{corollary}
若随机变量$X$的分布函数$F_{X}(x)$为严格单调增的连续函数,其反函数$F_{X}^{-1}(y)$存在,则$Y=F_{X}(X)$服从$(0,1)$上的均匀分布$U(0,1)$。
\end{corollary}
\begin{proof}
    首先注意到$Y=F_{X}(X)$是一个随机变量,于是,我们需要求其分布函数。由于根据分布函数的有界性,分布函数$F_{X}(x)$仅在$[0,1]$区间上取值,故当$y<0$时,因为$\left \{ F_{X}(X)\leq y \right \} $是不可能事件,所以$$F_{Y}(y)=P(Y \leqslant y)=P\left(F_{X}(X) \leqslant y\right)=0$$
    当$y\geq 1$时,因为$\left \{ F_{X}(X)\le y \right \} $是必然事件,所以$$F_{Y}(y)=P(Y \leqslant y)=P\left(F_{X}(X) \leqslant y\right)=1.$$
    当$0\leq y<1$时,有
    \begin{eqnarray*}
           F_{Y}(y) &=&P(Y \leqslant y) \\
    &=&P\left(F_{X}(X) \leqslant y\right) \\
    &=&P\left(F_{X}^{-1}\left(F_{X}(X)\right) \leqslant F_{X}^{-1}(y)\right) \\
    &=&P\left(X \leqslant F_{X}^{-1}(y)\right) \\
    &=&F_{X}\left(F_{X}^{-1}(y)\right) \\
    &=&y
    \end{eqnarray*}
    综上所述,$Y=F_{X}(X)$的分布函数为$$F_{Y}(y)=\left\{
    \begin{aligned}
   & 0 &,y<0 \\
   & y &,0 \leq y<1 \\
   & 1 &,y \geqslant 1
    \end{aligned}\right.$$
    因此,$Y\sim U(0,1)$。
\end{proof}
 \begin{remark}
 \begin{enumerate}
     \item  任一个连续随机变量$X$都可通过其分布函数$F(x)$与均匀分布随机变量$U$有关联。
     \item $X\sim {Exp}(\lambda)$,其分布函数$F(x)=1-e^{-\lambda x}$。于是,
     $$U=1-e^{-\lambda x} \Rightarrow x=\frac{1}{\lambda} \ln \frac{1}{1-u}.$$
     这表明了由均匀分布$U(0,1)$的随机数$u_{i}$可得指数分布${Exp}(\lambda)$的随机数$$x_{i}=\frac{1}{\lambda} \cdot \ln \frac{1}{1-u_{i}} \quad i=1,2, \cdots, n, \cdots,$$
     这是Monte Carlo法的基础。
 \end{enumerate}
\end{remark}
    
\subsection{$g(\cdot)$是其他特殊形式}
\begin{example}
    $X\sim N(0,1)$,求$Y=X^{2}$的密度函数。
\end{example}
\begin{solution}
    先求$Y$的分布函数$F_{Y}(y)$。由于$Y=X^{2}\ge 0$,故当$y\leq 0$时,有$F_{Y}(y)=0$,从而$P_{Y}(y)=0$.
   
    当$y>0$时,有$$F_{Y}(y)=P(Y \leqslant y)=P\left(X^{2} \leqslant y\right)=P(-\sqrt{y} \leqslant X \leqslant \sqrt{y})=2 \Phi(\sqrt{y})-1$$
    因此,$Y$的分布函数为
    $$F_{Y}(y)=\left\{\begin{array}{cc}
    2 \Phi(\sqrt{y})-1 & ,y>0 \\
    0 & , y \leq 0
    \end{array}\right.$$
    再用求导的方式求出$Y$的密度函数
    $$p_{Y}(y)=\left\{\begin{array}{cc}
    \varphi(\sqrt{y}) \cdot y^{-\frac{1}{2}} & ,y>0 \\
    0 & , y \leqslant 0
    \end{array}=\left\{\begin{array}{cc}
    \frac{1}{\sqrt{2 \pi}} y^{-\frac{1}{2}} e^{-\frac{y}{2}} & ,y>0 \\
    0 & ,y \leqslant 0
    \end{array}\right.\right.$$
    称$Y$服从自由度为 $1$的卡方分布,记$\chi ^{2}(1)$。
\end{solution}
\begin{remark}
   可以发现,这个分布也是伽马分布,即 $Ga(\frac{n}{2},\frac{1}{2})=\chi ^{2}(n)$。
    \end{remark}
\section{习题}
\begin{enumerate}
    \item 已知随机变量$X$的密度函数为
$$
 p(x) = \frac{2}{\pi}\cdot \frac{1}{e^x+e^{-x}},     -\infty < x < +\infty
$$
试求随机变量$Y = g(X)$的概率分布,其中
$$
 g(x) = \left. 
	 \begin{cases}
	 -1, & \text{当 } x < 0,\\
	 1, & \text{当 } x \geq 0 .
	 \end{cases} 
	 \right.
$$

\item 设圆的直径服从区间$(0,1)$上的均匀分布。求圆的面积的密度函数。

\item 设随机变量$X$的密度函数为
$$p_X(x) =\left\{ 
	 \begin{aligned}
	      \frac{3}{2}x^2, & -1 < x < 1,\\
	 0, & \text{其他}.
	 \end{aligned}
	\right.
$$
试求下列随机变量的分布:
\begin{enumerate}
\item $Y_1 = 3X$;
\item $Y_2 = 3-X$; 
\item $Y_3 = X^2$。
\end{enumerate}

\item 设$X$为随机变量,其取值范围为0到9,取值概率相等均为1/10。
\begin{enumerate}
    \item 求随机变量$Y = X$ mod$(3)$的分布列;
    \item 求随机变量$Y = 5$ mod$(X+1)$的分布列。
\end{enumerate}

\end{enumerate}