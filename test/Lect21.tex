\chapter{区间估计}
\begin{introduction}
  \item Prob $\&$ Stat\quad 6.6
\end{introduction}
\section{区间估计的概念}
设$x_{1},x_{2},\cdots,x_{n}$是样本。我们想要找到两个统计量$\hat{\theta}_{L}=\hat{\theta}_{L}(x_{1},\cdots,x_{n})$和$\hat{\theta}_{U}=\hat{\theta}_{U}(x_{1},\cdots,x_{n})$,$\hat{\theta}_{L}<\hat{\theta}_{U}$。于是,所构造的一个区间$[\hat{\theta}_{L},\hat{\theta}_{U}]$为$\theta$的一个区间估计。

\begin{problem}
    一个合适的区间估计应该有什么要求?
\end{problem}
\begin{note}
    \vspace{3cm}
\end{note}

因为样本具有随机性,所以,$\left[ \hat{\theta}_{L},\hat{\theta}_{U} \right]$是一个随机区间。但待估参数$\theta$是一个未知常数。我们通常要求区间$\left[ \hat{\theta}_{L},\hat{\theta}_{U} \right]$盖住$\theta$的概率
$$
P(\hat{\theta}_{L}\leq \theta\leq \hat{\theta}_{U})=P\left ( \left \{ \hat{\theta}_{L}\le \theta \right \}\cap \left \{ \hat{\theta}_{U}\le \theta \right \}   \right ) 
$$
尽可能大。
\begin{definition}
设$\theta$是总体的一个参数,其参数空间为$\Theta$,$x_1,x_2,\cdots,x_n$是来自该总体的样本,对给定的一个$\alpha(0<\alpha<1)$,假设有两个统计量$\hat{\theta}_L = \hat{\theta}_L(x_1,x_2,\cdots,x_n)$和$\hat{\theta}_U = \hat{\theta}_U(x_1,x_2,\cdots,x_n)$,若对任意的$\theta \in  \Theta$,有
$$
P(\hat{\theta}_{L}\le \theta\le \hat{\theta}_{U}) \geq 1-\alpha
$$
则称随机区间$\left[ \hat{\theta}_{L},\hat{\theta}_{U} \right]$为$\theta$的置信水平为$1-\alpha$的置信区间,或简称$\left[ \hat{\theta}_{L},\hat{\theta}_{U} \right]$是$\theta$的$1-\alpha$置信区间,$\hat{\theta}_{L}$和$ \hat{\theta}_{U}$分别称为$\theta$的(双侧)置信下限和置信上限。
\end{definition}
\begin{remark}
    \begin{enumerate}
        \item 若对给定的$\alpha(0<\alpha<1)$,对任意的$\theta \in  \Theta$,有
$$
P(\hat{\theta}_{L}\le \theta\le \hat{\theta}_{U}) = 1-\alpha
$$
则称$\left[ \hat{\theta}_{L},\hat{\theta}_{u} \right]$是$\theta$的$1-\alpha$同等置信区间;
\item 若对给定的$\alpha(0<\alpha<1)$,对任意的$\theta \in  \Theta$,有
$$
P(\hat{\theta}_{L}\le \theta) \geq 1-\alpha
$$
则称$\hat{\theta}_{L}$为$\theta$的置信水平为$1-\alpha$的(单侧)置信下限;
\item 若对给定的$\alpha(0<\alpha<1)$,对任意的$\theta \in  \Theta$,有
$$
P(\hat{\theta}_{L}\le \theta) = 1-\alpha
$$
则称$\hat{\theta}_{L}$为$\theta$的置信水平为$1-\alpha$的(单侧)同等置信下限;
\item 若对给定的$\alpha(0<\alpha<1)$,对任意的$\theta \in  \Theta$,有
$$
P(\theta\leq \hat{\theta}_{U}) \geq 1-\alpha
$$
则称$\hat{\theta}_{U}$为$\theta$的置信水平为$1-\alpha$的(单侧)置信上限;
\item 若对给定的$\alpha(0<\alpha<1)$,对任意的$\theta \in  \Theta$,有
$$
P(\theta\leq \hat{\theta}_{U}) = 1-\alpha
$$
则称$\hat{\theta}_{U}$为$\theta$的置信水平为$1-\alpha$的(单侧)同等置信上限;
\end{enumerate}
\end{remark}
\section{枢轴量法}
\begin{problem}
    如何求$1-\alpha$置信区间?
\end{problem}
\begin{definition}[枢轴量]
    若$G(x_1,x_2,\cdots,x_n,\theta)$是样本和待估参数$\theta$的函数,而$G$的分布不依赖于未知参数,则称$G$为枢轴量。
\end{definition}
以下我们介绍一种构造$1-\alpha$置信区间的通法——枢轴量法。
\begin{enumerate}
    \item 构造一个枢轴量$G$;
    \item 适当地选择两个常数$c$和$d$,使对给定的$\alpha(0<\alpha<1)$,有
    $$
    P(c\leq G\leq d) = 1-\alpha.
    $$
    如果$G$是一个离散型分布,取大于等于号。
    \item 假如能将$c\leq G\leq d$进行不等式变形化为$\hat{\theta}_L\leq \theta\leq \hat{\theta}_U$,则有
    $$
    P(\hat{\theta}_L\leq \theta\leq \hat{\theta}_U) = 1-\alpha.
    $$
    \item 由此,$\left[\hat{\theta}_L,\hat{\theta}_U\right]$是$\theta$的$1-\alpha $同等置信区间。
\end{enumerate}

\begin{example}
    若总体分布为$N(\mu,\sigma_0^2)$,其中$\mu$是待估参数,而$\sigma_0^2$是已知的。$x_1,x_2,\cdots,x_n$是样本。于是,我们介绍如何利用枢轴量法来构造$\mu$的区间估计。因为$\mu$的点估计为$$\hat{\mu} = \bar{x} = \frac{1}{n}\sum_{i=1}^n x_i.$$
    我们知道,$\bar{x}$的分布是$N(\mu,\sigma_0^2/n)$。对其进行标准化,
    $$
    G = \frac{\bar{x}-\mu}{\sqrt{\sigma_0^2/n}} \sim N(0,1).
    $$
    于是,$G$就是我们所构造的枢轴量。于是,存在两个常数$c_1$和$c_2$,满足
    $$
    P(c_1\leq G\leq c_2) = 1-\alpha.
    $$
    虽然$(c_1,c_2)$的取法有无数种,但是最简单的取法为$c_1= z_{\alpha/2}$和$c_2 = z_{1-\alpha/2}$。所以,
$$
z_{\alpha/2} \leq \frac{\bar{x}-\mu}{\sqrt{\sigma_0^2/n}}\leq z_{1-\alpha/2},
$$
可以变形为
$$
\bar{x}- \frac{\sigma_0}{\sqrt{n}}u_{1-\alpha/2} \leq \mu \leq \bar{x}- \frac{\sigma_0}{\sqrt{n}}u_{\alpha/2} = \bar{x} + \frac{\sigma_0}{\sqrt{n}}u_{1-\alpha/2}
$$
因此,$\mu$的$1-\alpha$同等置信区间为
$$
\left[\bar{x}- \frac{\sigma_0}{\sqrt{n}}u_{1-\alpha/2},\bar{x} + \frac{\sigma_0}{\sqrt{n}}u_{1-\alpha/2}  \right].
$$
\end{example}
\begin{remark}
    区间$\left[\bar{x}- \frac{\sigma_0}{\sqrt{n}}u_{1-\alpha/2},\bar{x} + \frac{\sigma_0}{\sqrt{n}}u_{1-\alpha/2}  \right]$是最短的。
\end{remark}



\begin{example}
若总体分布为$N(\mu,\sigma^2)$,其中$\mu$是待估参数,而$\sigma^2$也是未知的。$x_1,x_2,\cdots,x_n$是样本。于是,我们介绍如何利用枢轴量法来构造$\mu$的区间估计。因为$\mu$的点估计为$$\hat{\mu} = \bar{x} = \frac{1}{n}\sum_{i=1}^n x_i.$$
我们知道,$\bar{x}$的分布是$N(\mu,\sigma^2/n)$。对其进行标准化,
    $$
     \frac{\bar{x}-\mu}{\sqrt{\sigma^2/n}} \sim N(0,1).
    $$
    除了待估计的参数之外,还有一个未知参数$\sigma^2$,称其为冗余参数。对于冗余参数$\sigma^2$,我们用其估计代替参数,即
    $$
     G = \frac{\bar{x}-\mu}{\sqrt{\hat{\sigma}^2/n}} = \frac{\frac{\bar{x}-\mu}{\sqrt{\sigma^2/n}}}{\sqrt{\frac{(n-1)\hat{\sigma}^2}{\sigma^2}/(n-1)}} \sim t(n-1).
    $$
    于是,$G$就是我们所构造的枢轴量。于是,存在两个常数$c_1$和$c_2$,满足
    $$
    P(c_1\leq G\leq c_2) = 1-\alpha.
    $$
    取$c_1 = t_{\alpha/2}(n-1)$和$c_2 = t_{1-\alpha/2}(n-1)$。所以,
$$
t_{\alpha/2}(n-1) \leq \frac{\bar{x}-\mu}{\sqrt{\hat{\sigma}^2/n}}\leq t_{1-\alpha/2}(n-1),
$$
可以变形为
$$
\bar{x}- \frac{\hat{\sigma}}{\sqrt{n}}t_{1-\alpha/2}(n-1) \leq \mu \leq \bar{x} + \frac{\hat{\sigma}}{\sqrt{n}}t_{1-\alpha/2}(n-1) 
$$
因此,$\mu$的$1-\alpha$同等置信区间为
$$
\left[\bar{x}- \frac{\hat{\sigma}}{\sqrt{n}}t_{1-\alpha/2}(n-1),\bar{x} + \frac{\hat{\sigma}}{\sqrt{n}}t_{1-\alpha/2}(n-1) \right].
$$
\end{example}

\begin{problem}
    置信水平为$1-\alpha$指的是什么?
\end{problem}
\begin{note}
    在《概率论与数理统计教程》第300页例6.6.1中,我们取$\alpha = 0.1$以及$n=10$的情形。假定真实的分布为正态分布$N(\mu,\sigma^2)$,其中$\mu = 15$,方差$\sigma^2 = 4$。我们从这个正态分布里抽一组样本量为10的样本,记为$x_{m,1},x_{m,2},\cdots,x_{m,10}$。基于样本,我们可以计算样本均值$\bar{x}_m = \frac{1}{n}\sum_{i=1}^n x_{m,i}$,样本方差$
    s_{m}^2 = \frac{1}{n-1}\sum_{i=1}^n (x_{m,i}-\bar{x}_m)^2$。同时,根据置信水平为$1-\alpha=0.9$和样本量$n=10$,来确定分位数
    $$
    t_{1-\alpha/2}(n-1) = t_{0.95}(9)=1.8331.
    $$
    根据这组样本,我们可以构造一个置信区间。重复这个行为100次,于是可以得到100个区间。
    对这100个区间进行统计,可以发现其中91个区间是包含真值$\mu=15$的,而这个比例$91\%$与我们所构造的置信水平差不多。

    这就是置信水平的一种理解。
   \begin{note}
        \vspace{3cm}
   \end{note}
   
    \end{note}

\begin{example}
若总体分布为$U(0,\theta)$,其中$\theta$是待估参数。现有$x_1,x_2,\cdots,x_n$是样本。我们想要得到$\theta$的$1-\alpha$置信区间。
\end{example}
\begin{solution}
我们知道$\theta$的点估计为$x_{(n)}$,而且我们知道$$
\frac{x_{(n)}}{\theta} \sim Be(n,1).
$$
于是,$x_{(n)}/\theta$就是我们所构造的枢轴量。我们希望找到两个常数$c_!$和$c_2$使得
$$
P(c_1 \leq \frac{x_{(n)}}{\theta} \leq c_2) = 1-\alpha. 
$$
一旦可以确定$c_1$和$c_2$,我们可以得到$\theta$的$1-\alpha$置信区间为
$$
\left[\frac{x_{(n)}}{c_2}, \frac{x_{(n)}}{c_1}\right].
$$
显然满足条件的$(c_1,c_2)$个数是无穷的。这里我们考虑最短的置信区间。因为我们知道,对于一个随机变量$X\sim Be(n,1)$,其密度函数为
$$
p(x) = nx^{n-1}, 0<x<1.
$$
于是,
$$
P(X\leq c) = \int_{0}^c nx^{n-1}\text{d}x = c^{n}.
$$
我们令$c_1^n = \alpha_1$,$c_2^n = 1-(\alpha-\alpha_1)$。我们要求$(c_1,c_2)$使得
$$
\frac{1}{c_1} - \frac{1}{c_2}
$$
达到最小。
所以,我们构造了一个函数
$$
l(\alpha_1) = \alpha_1^{-1/n} - (1-\alpha+\alpha_1)^{-1/n}, 0<\alpha_1<\alpha
$$
我们可以证明$l(\alpha_1)$是一个单调递减的函数,则$l(\alpha_1)$在$\alpha_1 = \alpha$处取到最小值。于是,
$$
c_1 = \alpha^{1/n},\quad c_2 = 1.
$$
因此,$\theta$的$1-\alpha$置信区间为
$$
\left[ x_{(n)}, x_{(n)} \cdot \alpha^{-1/n} \right].
$$
\end{solution}


\begin{example}
若总体分布为$N(\mu,\sigma^2)$,其中$\sigma^2$是待估参数,而$\mu$也是未知的。$x_1,x_2,\cdots,x_n$是样本。这里我们考虑$\sigma^2$的$1-\alpha$置信区间。首先考虑$\sigma^2$的点估计为$$
s^2 = \frac{1}{n-1}\sum_{i=1}^{n} (x_i - \bar{x})^2
$$
其分布为
$$
\frac{(n-1)s^2}{\sigma^2} \sim \chi^2(n-1).
$$
很自然地我们想要找到两个常数$c_1$和$c_2$使得
$$
P(c_1\leq \frac{(n-1)s^2}{\sigma^2}\leq c_2 ) =1-\alpha
$$
当然,我们可以类似于例题1.3的做法求解最短的区间,但是我们也可以取
$$
c_1 = \chi_{\alpha/2}^{2}(n-1), \quad c_2 = \chi_{1-\alpha/2}^{2}(n-1).
$$
虽然这个区间不是最短的,但是它是最简单的取法。于是,$\sigma^2$的$1-\alpha$置信区间为
$$
\left[ \frac{(n-1)s^2}{\chi_{1-\alpha/2}^{2}(n-1)},  \frac{(n-1)s^2}{\chi_{\alpha/2}^{2}(n-1)}\right].
$$
这个区间被称为等尾置信区间。
\end{example}

以上介绍例子大多都是在正态总体假定下的置信区间估计的构造方法。这些方法还可以应用于大样本的情形中。
\begin{example}
若总体分布为$b(1,\theta)$,其中$\theta$是待估参数。$x_1,x_2,\cdots,x_n$是样本。当样本量$n$比较大时,我们想要得到$\theta$的$1-\alpha$置信区间。

首先考虑$\theta$的点估计,即
$$
\hat{\theta} = \bar{x} = \frac{1}{n}\sum_{i=1}^n x_i.
$$
因为样本量$n$比较大,这里我们可以考虑$\bar{x}$的渐近分布,即
$$
\frac{\bar{x} - \theta}{\sqrt{\theta(1-\theta)/n}} \sim  AN(0,1).
$$
因此,$\frac{\bar{x} - \theta}{\sqrt{\theta(1-\theta)/n}} $就是我们所构造的(渐近)枢轴量。类似于正态总体下的置信区间,我们取$\pm z_{1-\alpha/2}$,有
$$
P\left( - z_{1-\alpha/2} \leq \frac{\bar{x} - \theta}{\sqrt{\theta(1-\theta)/n}}  \leq  z_{1-\alpha/2} \right)= 1-\alpha.
$$
于是,我们需要从上述等式的左边将$\theta$反解出来。
\begin{enumerate}
    \item 第一种想法:通过解一个一元二次不等式,从而得到$\theta$的置信区间。具体来说,
    $$
    \frac{|\bar{x} - \theta|}{\sqrt{\theta(1-\theta)/n}}  \leq  z_{1-\alpha/2}
    $$
    \vspace{3cm}
    \item 第二种想法:实际上我们注意到分母中$\theta(1-\theta)/n$是$\bar{x}$也可以看作一种冗余参数,很自然的想法是我们用估计来代替。在样本量大的情形下,$\bar{x}$可以近似$\theta$,所以
    $$
    \frac{\theta(1-\theta)}{n} \approx  \frac{\bar{x}(1-\bar{x})}{n}.
    $$
    这样可以很容易反解出$\theta$,从而得到$\theta$的$1-\alpha$置信区间
    $$
    \left [ \bar{x}-z_{1-\frac{\alpha }{2} }\sqrt{\frac{\bar{x}(1-\bar{x})}{n} }  ,\bar{x}+z_{1-\frac{\alpha }{2} }\sqrt{\frac{\bar{x}(1-\bar{x})}{n} } \right ] .
    $$
\end{enumerate}
\end{example}


\begin{example}
现有两个独立总体$N(\mu_1,\sigma_1^2)$和$N(\mu_2,\sigma_2^2)$。设$x_1,x_2,\cdots,x_m$是来自$N(\mu_1,\sigma_1^2)$的样本,而$y_1,y_2,\cdots,y_n$是来自$N(\mu_2,\sigma_2^2)$的样本。
考虑不同$\theta$以及它们的$1-\alpha$置信区间。
\begin{enumerate}
    \item $\theta = \mu_1 - \mu_2$。
    \begin{enumerate}
        \item 若$\sigma_1^2$和$\sigma_2^2$已知。由于$\theta$的点估计为
        $
        \hat{\theta} = \bar{x} - \bar{y}
        $
        其分布为
        $$
        \hat{\theta} = \bar{x} - \bar{y} \sim N\left(\theta, \frac{\sigma_1^2}{m} + \frac{\sigma_2^2}{n}\right).
        $$
        于是,枢轴量为
        $$
        G = \frac{\hat{\theta} - \theta}{ \sqrt{\frac{\sigma_1^2}{m} + \frac{\sigma_2^2}{n}}} \sim N(0,1).
        $$
        所以,$\theta$的$1-\alpha$置信区间为
        $$
        \left[
        (\bar{x}-\bar{y}) - \sqrt{\frac{\sigma_1^2}{m} + \frac{\sigma_2^2}{n}} z_{1-\alpha},
        (\bar{x}-\bar{y}) + \sqrt{\frac{\sigma_1^2}{m} + \frac{\sigma_2^2}{n}} z_{1-\alpha}
        \right]
        $$
        \item 若$\sigma_1^2= \sigma_2^2 = \sigma^2$未知。令
        $$
        s_w^2 = \frac{(m-1)s_1^2 + (n-1)s_2^2}{m+n-2}.
        $$
        则$(m+n-2)s_w^2 / \sigma^2$的分布为$\chi^2(m+n-2)$。
        于是,我们可以构造枢轴量
        $$
        \frac{\hat{\theta} -\theta }{s_w\sqrt{\frac{1}{m}+\frac{1}{n}} } \sim t(m+n-2)
        $$
        所以,$\theta$的$1-\alpha$置信区间为
        $$
        \bar{x}-\bar{y} \pm s_w\sqrt{\frac{1}{m}+\frac{1}{n}} t_{1-\alpha/2}(m+n-2)
        $$
    \end{enumerate}
    \item $\theta = \sigma_1^2/\sigma_2^2$。我们考虑$\theta$的点估计为$s_1^2/s_2^2$,则$$
    \frac{s_1^2}{s_2^2}/\theta \sim F(m-1,n-1).
    $$
    所以,$\theta$的$1-\alpha$置信区间为
    $$
    \left[
     \frac{s_1^2}{s_2^2}/ F_{1-\alpha/2}(m-1,n-1),
     \frac{s_1^2}{s_2^2}/ F_{\alpha/2}(m-1,n-1)
    \right].
    $$
\end{enumerate}
\end{example}

\section{习题}
\begin{enumerate}
    \item 假设人体身高服从正态分布,今抽测甲、乙两地区18岁至25岁女青年身高数据如下:甲地抽取10名,样本均值1.64米,样本标准差0.2米;乙地区抽取10名,样本均值1.62米,样本标准差0.1米。求
    \begin{enumerate}
        \item 两样本总体方差比的置信水平为95\%的置信区间;
        \item 两样本总体均值差的置信水平为95\%的置信区间;
    \end{enumerate}

\item 在一批货物中随机抽取80件,发现有11件不合格品,试求这批货物的不合格品率的置信区间为0.90的置信区间。

\item 总体$X\sim N(\mu,\sigma^2),\sigma^2$已知,问样本量$n$取多大时才能保证$\mu$的置信水平为$95\%$的置信区间的长度不大于$k$。


\item $0.50,1.25,0.80,2.00$是取自总体$X$的样本,已知$Y = \ln X$服从正态分布$N(\mu,1)$.
\begin{enumerate}
    \item 求$\mu$的置信水平为$95\%$的置信区间;
    \item 求$X$的数学期望的置信水平为$95\%$的置信区间;
\end{enumerate}

\item  设总体$X$的密度函数为
$$
p(x;\theta) = e^{-(x-\theta)} I_{\{ x>\theta \}}, -\infty < \theta < \infty
$$
$x_1,x_2,\cdots,x_n$为抽自此总体分布的简单随机样本。 
\begin{enumerate}
    \item 证明:$x_{(1)} - \theta$的分布与$\theta$无关,并求出此分布;
    \item 求$\theta$的置信水平的$1-\alpha$置信区间。
\end{enumerate}
\end{enumerate}