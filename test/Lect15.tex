\chapter{总体与样本}

\begin{introduction}
  \item Prob $\&$ Stat\quad5.1
\end{introduction}

\section{引导问题:数据的价值}
\begin{problem}
当下,没有人能够忽略数据的价值。在后续课程中,我们需要思考以下三个问题:
\begin{enumerate}
    \item 数据是什么?
    \item 如何分析数据?
    \item 得到怎样的结论?
\end{enumerate}
\end{problem}

\section{总体与个体}

在统计学中,我们将研究对象的全体称为\textbf{总体},构成总体的每个成员称为\text{个体}。

\begin{example}[(总统选举)]
若有一个“乌托邦”的国家,通过民意来选举总统。目前有两位总统候选人:$1$号候选人和$2$号候选人。假定每位公民只能从这两位候选人中进行投票,且只能投一票。每一位公民的投票结果就是\textbf{个体},而所有公民的投票结果就是\textbf{总体}。总体就是由$\{1,2\}$构成的一个集合。我们关心的是哪一位候选人当选?本质上就是总体中1出现的数量和2出现的数量,哪一个多?在唱票前,没有人能够确定谁能够当选。于是,在结果揭晓之前,我们并不能知道这个具体的数值。

事实上,在考虑总体时,我们只关心$1$号候选人能否当选,由此可以构造一个随机变量。也就是说,$1$号候选人能否当选可以看作一个伯努利分布的随机变量$b(1,p)$。
\end{example}
\begin{remark}
    在本课程中,我们将总体视作一种分布来考虑的。
    
    通常而言,总体有不同的区分方式,如下:
    \begin{itemize}
        \item 一维总体 vs 多维总体;
        \item 无限总体 vs 有限总体。
    \end{itemize}
\end{remark}
\section{样本}
\begin{example}[(总统选举 - 续)]
    在唱票之前,我们可以采访部分投票公民,从而得知他们的投票结果。假设共采访了$n$位公民,得到了他们的投票结果,记为$x_1,x_2,\cdots,x_n$。在统计学中,称$x_1,x_2,\cdots,x_n$为\textbf{样本},也是我们所观测到的数据。
\end{example}
\begin{remark}
    一方面,样本既表示从总体中随机抽取的,抽取前无法预知其取值,所以,样本是随机变量;另一方面,样本在抽取以后经观测就有确定的观测值,所以,样本又是一组数值。这就是样本所谓的两重性。

    这两重性质是\textbf{同等重要的}。在数据分析中,需要得到具体的结果并进行合理的结论,主要依赖于样本的可观测性;而如何进行分析,需要事先确定数据分析的方法,这主要依赖于样本的随机性。
\end{remark}



\begin{problem}
    怎样的样本才能够帮助我们研究总体?
\end{problem} 

\begin{example}[(总统选举 - 续)]
假定在乌托邦里有两个城镇:$T_1,T_2$,两个城镇里各有10个百姓。通过唱票结果发现:因为候选人$A$来自于城镇$T_1$,他们中有7名会为候选人$A$投票,剩下的投票给了候选人$B$;而候选人$B$来自于城镇$T_2$,他们中9人会为$B$投票,只有1人投票给了$A$。如果我们只调研城镇$T_1$的公民,很自然会认为$A$当选,但实际结果相反。
\end{example}

虽然从总体中抽取样本有不同的抽法,但是我们希望能够通过样本来对总体作出较为可靠的推断,我们希望所抽出的样本能够很好地代表总体。需要对样本提出以下两个要求:
\begin{itemize}
    \item \textbf{代表性}:要求总体中每一个个体都有同等机会被选入样本,这表示每一个样本$x_i$与总体$X$具有相同的分布。
    \item \textbf{独立性}:要求样本中每一个的取值不影响其他取值,这表示$x_1,x_2,\cdots,x_n$相互独立。
\end{itemize}
这就是我们常说的“简单随机抽样”。

由此,我们常常会这样描述一个统计问题。设总体$X$具有分布函数$F(x)$,$x_1,x_2,\cdots,x_n$为取自该总体的容量为$n$的样本,则样本的联合分布函数为
$$
F(x_1,x_2,\cdots,x_n) = \prod_{i=1}^n F(x_i).
$$

\section{补充案例}

\begin{example}[(测量)]
    每天早上起床我们可以多次测量自己的身高。如果使用精度高的测量仪器来进行数据采集,可以得到一系列数据:166.01厘米,165.98厘米,165.89厘米,166.11厘米,165.96厘米,$\cdots$。假设我们测量了100万次,那么共有100万个数据。于是,我们可以得到一个分布。这里我们可以认为自己的身高是一个总体,其有一个均值,表示我们真实的高度,而每次测量是存在一些偏差。所以,我们定义所测量的数据为
    $$
    x = \mu + \varepsilon
    $$
    这个式子是所有测量的最简单结构式,放之四海皆准。其中,$x$表示观测值(可以观测到的),$\mu$表示真值(永远未知),$\varepsilon$表示误差(需要假定)。

    为了认识$\mu$,我们需要对误差进行一些假定,才能对$\mu$做出合理的推断。
    \begin{itemize}
        \item 假定一:$\varepsilon \sim N(0,\sigma^{2})$,其中$\sigma^{2}$是未知的。在这个假定中我们利用高斯分布(正态分布)$ N(\mu,\sigma^{2})$来刻画测量数据。高斯分布由两个参数唯一确定,其中,$\mu$是我们感兴趣的参数,称为目标参数;而$\sigma^{2}$是我们不感兴趣的参数,称为讨厌参数(或冗余参数)。
        \item 假定二:$\varepsilon \sim N(0,\sigma^{2}_0)$,其中$\sigma^{2}_0$是已知的。
        \item 假定三:$\varepsilon$服从一个以$0$为中心的对称分布,即$x$服从一个以$\mu$为中心的对称分布。这里我们并未对数据的分布给出一个具体的假定。 
    \end{itemize}
\end{example}
\begin{remark}
    与假设一和假设二中参数总体的假定不同,假设三中是非参数总体的假定。
\end{remark}

\begin{example}
    $1979$年$4$月$17$日,日本《朝日新闻》曾刊登过这样的一条消息:美国人喜欢购买日本索尼工厂生产的彩电,而不愿购买设在美国加州的索尼工厂生产的彩电。而美国本土生产的电视机出厂合格率为$100\%$,日本产的合格率只有$99.73\%$.这究竟是什么原因呢?美国一家咨询公司采用统计抽样的方法,对此进行了专题调查分析。结果发现,两地生产彩电质量特征的概率分布不同。

    在美国的工厂中,采用门柱法来进行产品质量管理;而在日本的工厂中,采用$3$-$\sigma$质量管理策略:将产品划分优、良和合格三类,严格控制每一类产品在市场的占比。基于$3$-$\sigma$质量管理策略,日本产品的优良率明显高于美国本土生产的产品。所以更受到美国消费者的青睐。
\end{example}
